\documentclass[12pt]{article}
\usepackage[left = 1in, right = 1in, top = 1in, bottom = 1in]{geometry}
\usepackage{textcomp}
\usepackage{gensymb}
\usepackage{paralist}
\usepackage{cancel}
\usepackage{enumitem}
\usepackage{amsmath}
\usepackage{amssymb}
\usepackage{amsthm}
\usepackage{tkz-euclide}
\usepackage{hyperref}
\usepackage{esdiff}
\usepackage{parskip}
\usepackage{accents}
\usepackage{xcolor}

\usetikzlibrary{arrows.meta,positioning}

\newtheoremstyle{customstyle}
  {8pt} % Space above (adjust as needed)
  {0pt} % Space below (adjust as needed)
  {} % Body font
  {} % Indent amount
  {\bfseries} % Theorem head font
  {. } % Punctuation after theorem head
  {0pt} % Space after theorem head
  {} % Theorem head spec
\theoremstyle{customstyle}
\newtheorem{theorem}{Theorem}[section]
\newtheorem{exercise}{Exercise}[section]
\newtheorem{claim}[theorem]{Claim}
\newtheorem{prop}[theorem]{Proposition}
\newtheorem{corollary}[theorem]{Corollary}
\newtheorem{lemma}[theorem]{Lemma}
\newtheorem{definition}[theorem]{Definition}
\newtheorem{question}{Question}
\newtheorem{subquestion}{Part}[question]

\def\definitionautorefname{Definition}
\def\corollaryautorefname{Corollary}

\newenvironment{nonproof}{\par $\cancel {\text{\textit{Proof}}}.$}{\hfill$\cancel\square$}

\def\contra{\tikz[baseline, x=0.22em, y=0.22em, line width=0.032em]\draw (0,2.83)--(2.83,0) (0.71,3.54)--(3.54,0.71) (0,0.71)--(2.83,3.54) (0.71,0)--(3.54,2.83);}
\newenvironment{answer}{\par\noindent\textit{Answer.}}{\par}

\renewcommand{\CancelColor}{\color{red}}
\renewcommand{\Re}{\operatorname{Re}}
\renewcommand{\Im}{\operatorname{Im}}
\renewcommand{\bar}{\overline}
% \renewcommand{\vec}[1]{\undertilde{\mathrm{#1}}}
% \renewcommand{\vec}[1]{\undertilde{#1}}
% \renewcommand{\vec}[1]{\underline{#1}}
\renewcommand{\vec}[1]{\mathbf{#1}}
\newcommand{\unitvec}[1]{\hat{\vec{#1}}}
\newcommand{\sol}{$\operatorname{sol}^{\simeq}$}
\newcommand{\Arg}{\operatorname{Arg}}
\newcommand{\Log}{\operatorname{Log}}
\newcommand{\vecspan}{\operatorname{span}}
\newcommand{\id}[1]{\operatorname{id}_{#1}}
\newcommand{\indic}[1]{i_{#1}}
\newcommand{\Sym}{\operatorname{Sym}}
\newcommand{\Isom}{\operatorname{Isom}}
\newcommand{\di}{\mathrm{d}}
\newcommand{\gen}[1]{\langle{#1}\rangle}

\definecolor{applegreen}{rgb}{0.55, 0.71, 0.0}
\definecolor{ufogreen}{rgb}{0.24, 0.82, 0.44}


\begin{document}
% Lectured by Angela Capel Gueras (ac2722)
% Office: BO.14, Department of Applied Mathematics and Theoretical Physics

\section{Complex Numbers}

\subsection{Definition}

We construct $\mathbb{C}$ by adding $i$ to $\mathbb{R}$, where $i^{2} = -1$.
Then, any $z \in \mathbb{C}$ has the form $z = x + iy$, where $x,y \in \mathbb{R}$.
We define the real part, $x = \Re(z)$, and the imaginary part, $y = \Im(z)$.

\subsection{Properties}

\begin{compactenum}[i)]
\item Addition: $z_{1} \pm z_{2} = (x_{1} \pm x_{2}) + i(y_{1} \pm y_{2})$.
\item Multiplication: $z_{1}z_{2} = (x_{1}y_{1} - x_{2}y_{2}) + i(x_{1}y_{2} + x_{2}y_{1})$.
\end{compactenum}
Note that both addition and multiplication are associative and commutative.
\begin{compactenum}[i)]
\setcounter{enumi}{2}
\item Identity: The group $(\mathbb{C},+)$ is an abelian group with identity $0$.
\item Inverse: The \emph{inverse of $z$} is given by
    \[
    z^{-1} = \frac{x - iy}{x^{2} + y^{2}}
    \]
    and it satisfies $z \cdot z^{-1} = 1$.
\end{compactenum}

Thus, $(\mathbb{C}^{*}, \cdot)$ is an abelian group with identity $1$,
where $\mathbb{C}^{*} = \mathbb{C} - \{0\}$.

Note that the distributive property is also satisifed, where
\[
    (z_{1}+z_{2})z_{3} = z_{1}z_{2} + z_{2}z_{3}
\]
\begin{compactenum}[i)]
\setcounter{enumi}{4}
\item Complex Conjugate: For any $z = x + iy$, 
    the \emph{complex conjugate of $z$}, denoted $\bar z$, or $z^{*}$, is equal to $x - iy$.
\end{compactenum}
Thus, we also have $\Re(z) = (z + \bar z) / 2$, and $\Im(z) = (z - \bar z) / 2i$.
We have rules
\begin{compactitem}
\item $\bar {\bar z} = z$
\item $\bar{z_{1} + z_{2}} = \bar z_{1} + \bar z_{2}$
\item $\bar {z_{1}z_{2}} = \bar z_{1} \bar z_{2}$
\end{compactitem}

\begin{compactenum}[i)]
\setcounter{enumi}{5}
\item Modulus: For any $z = x + iy$, we define the \emph{modulus of $z$} by $|z|$, or $r$, by
    a real and non-negative number such that
    \[
    |z|^{2} = x^{2} + y^{2}
    \]
\item Argument: The \emph{argument} of a complex number $z = x + iy \ne 0$
    is a real number, denoted by $\theta = \arg(z)$, such that
    \[
        z = r(\cos \theta + i \sin \theta)
    \]
    Which is the \emph{polar form} of $z$. We can verify
    that $\tan\theta = y/x$.
\end{compactenum}

Note that if $\theta$ is an argument of $z$, so is $\theta + 2\pi n$ for $n \in \mathbb{Z}$.
To make it unique, we restrict $-\pi < \theta \le \pi$.
This value of $\theta$ is the \emph{principal value}.

Remarks:
\begin{compactenum}[(1)]
\item $\mathbb{R} \subset \mathbb{C}$, since for $a \in \mathbb{R}$ we have
    $a = a + i 0 \in \mathbb{C}$.
\item A complex number $0 + ib$ is said to be a \emph{pure imaginary number}.
\item The representation of a complex number in terms
    of its real and imaginary parts is unique.
\end{compactenum}

\subsubsection*{More Properties / Consequences}
\begin{compactenum}[(i)]
\setlength{\parskip}{4pt}
\item $(\mathbb{C},+,\cdot)$ is a field.
\item \emph{Fundamental Theorem of Algebra}:

    A polynomial of degree $n$ with coefficients in $\mathbb{C}$
    can be written as the product of $n$ linear factors:

    \[
        \begin{aligned}
            p(z) &= c_n z^{n} + \cdots + c_{0} \\
                &= c_n (z - \alpha _1)\cdots(z - \alpha_2)
        \end{aligned}
    \]
    
    where $c_i \in \mathbb{C}$, and $c_n \ne 0$, with roots $\alpha_i \in \mathbb{C}$.
    Thus, $p(z) = 0$ has at least one root,
    and $n$ roots counted with multiplicity.
\item The modulus satisifes the following properties:
    \begin{itemize}
        \item $|z_{1}z_{2}| = |z_{1}||z_{2}|$
        \item $|z_{1}| + |z_{2}| \le |z_{1}| + |z_{2}|$
        \item $|z_{1} - z_{2}| \ge \left||z_{1}| - |z_{2}|\right|$
    \end{itemize}
\end{compactenum}

\subsubsection*{De Moivre's Theorem}

\begin{lemma}
    If $z_{1} = r_{1}(\cos \theta_{1} + i \sin \theta_{1})$,
    and $z_{2} = r_{2}(\cos \theta_{2} + i \sin \theta_{2})$,
    then $z_{1}z_{2} = r_{1}r_{2}(\cos (\theta_{1} + \theta_{2}) + i \sin (\theta_{1} + \theta_{2})$.
\end{lemma}

\begin{proof}
    Multiplying naively gives
    \[
    z_{1}z_{2} = r_{1}r_{2}(
        (\cos\theta_{1} \cos\theta_{2} - \sin\theta_{1}\sin\theta_{2})
        +i(\cos\theta_{1} \sin\theta_{2} + \cos\theta_{2} \sin\theta_{1})
    )
    \]
    Applying the addition rule of $\sin$ and $\cos$, the result is obtained.
\end{proof}

\begin{theorem}[De Moivre's]
    \label{thm:de_moivre}
    For any $\theta \in \mathbb{R}$, $n \in \mathbb{Z}$, we have
    \[
        (\cos\theta + i\sin\theta )^{n} = \cos(n\theta ) + i\sin(n\theta ).
    \]
\end{theorem}
\begin{proof}
    The proof is trivial and left as an exercise to the reader.
\end{proof}

\setcounter{subsection}{3}
\subsection{Exponential and Trigonometric Functions}

\subsubsection{Exponential Functions}

We define the exponential function on $z \in \mathbb{C}$ by
\[
    \exp(z) = e^{z} = \sum\limits_{n=0}^{\infty}\frac{1}{n!}z^{n}.
\]
The function only exists when the series converges,
and it does for all $z \in \mathbb{C}$.
A proof will not be provided for that result.

Properties:
\begin{compactitem}
\item $e^{z}e^{w} = e^{z+w}$, for all $z,w \in \mathbb{C}$.
\item If $z \in \mathbb{R}$, then $\exp(z)$ reduces to usual exponential.
\item $e^{0} = 1$.
\item $(e^{z})^{n} = e^{nz}$.
\end{compactitem}

\subsubsection{Trigonometric Functions}

We have
\begin{align*}
    \cos(z) &= \frac{1}{2}(e^{iz} + e^{-iz}) \\
            &= \frac{1}{2}\left(
            \sum\limits_{n=0}^{\infty}\frac{1}{n!}(iz)^{n}
            +\sum\limits_{n=0}^{\infty}\frac{1}{n!}(-iz)^{n} \right) \\
            &= \sum\limits_{n=0}^{\infty}(-1)^{n}\frac{z^{2n}}{(2n)!}
\end{align*}

Similarly, 
\begin{align*}
    \sin(z) &= \frac{1}{2i}(e^{iz} - e^{-iz}) \\
            &= \frac{1}{2i}\left(
            \sum\limits_{n=0}^{\infty}\frac{1}{n!}(iz)^{n}
            -\sum\limits_{n=0}^{\infty}\frac{1}{n!}(-iz)^{n} \right) \\
            &= \sum\limits_{n=0}^{\infty}(-1)^{n}\frac{z^{2n+1}}{(2n+1)!}
\end{align*}

If $z \in \mathbb{R}$, this clearly reduces to the usual trigonometric functions.

It is also possible to notice that
\[
e^{iz} = \cos z + i\sin z,
\]
in particular, if $z = \pi$, then
\begin{equation*}
    e^{i\pi} = -1
\end{equation*}
which is \emph{Euler's Identity}.

\begin{corollary}
    $e^{z} = 1 \iff z = 2\pi ni$,
    where $n \in \mathbb{Z}$.
\end{corollary}
\begin{proof}
    Let $z = x+iy$, then 
    $e^{z} = e^{x}e^{iy} = e^{x}(\cos y+i\sin y) 
    = e^{x}\cos y + ie^{x} \sin y = 1$.
    Then, $e^{x}\sin y = 0$, so $\sin y = 0$, so $y = n\pi$. 
    Similarly, $e^{x}\cos y = 1$,
    substituting gives $e^{x}(\pm 1) = 1$, 
    and since $e^{x}$ is positive, $\cos y = 1$, 
    so $y = 2n\pi$.
    Thus, $e^{x} = 1$, so $x = 0$, and $z = 2\pi n i$.
\end{proof}

Finally, we can write $z = r(\cos\theta + i\sin\theta) = re^{i\theta }$,

\subsubsection{Roots of Unity}

Let $z = re^{i\theta }$, and let for some $N \in \mathbb{N}$ that $z^{N} = 1$.
Then we have
\[
z^{N} = r^{N}e^{i\theta N}.
\]
Trivially then we have $r = 1$, $\theta N= 2\pi n$, so
\[
    z = e^{\frac{2\pi i}{N}n}
\]

\subsection{Logarithm and Complex Powers}

We say, if $z \in \mathbb{C}$ and $z \ne 0$, that $w = \log z$ iff $e^{w} = z$.
By definition, $\log$ is the inverse of $\exp$. 
Note that $\log$ here means the natural logarithm.

Importantly, this is a many-to-one relationship.
\[
r = re^{i\theta } = e^{\log r}e^{i\theta } = e^{\log r + i\theta },
\]
so
\[
\log z = \log r + i\theta  = \log r + i \arg(z)
\]
for any $\theta  \in \Arg(z)$.
Thus, if $r + i\theta $ is a log of $z$, then so is $r + i(\theta + 2\pi)$.
To make it unique, we simply use the principal value of $\arg$,
as in $-\pi < \theta  \le \pi$. 
In contexts where we use all values, we write $\Log(z)$.

\subsubsection*{Complex Powers}

We define $z$ to the power of $\alpha $ to be
\begin{align*}
    z^{\alpha } &= e^{\alpha \Log z}\\
                &= e^{\alpha(\log r + i\theta + i 2n\pi)}
\end{align*}
This is multivalued due to the multi-valued nature of $\Log$.

Examples:
\begin{compactitem}
\item $\log(i) = \log(e^{i\pi / 2}) = i \left(\frac{\pi}{2} + 2\pi n\right)$.
\item $\log(1 + i) = \log\sqrt{2} + i\left(\frac{\pi}{4} + 2\pi n\right)$.
\item $(1 + i)^{1/2} = \sqrt[4]{2} + i\left(\frac{\pi}{8} + \pi n\right)$.
\end{compactitem}

Note that $(e^{a})^{b}$ is not necessarily equal to $e^{ab}$.

\subsection{Lines and Circles}

\subsubsection*{Lines}

Lines are defined by a point $z_{0} \in \mathbb{C}$ and a direction $\omega \in \mathbb{C}$,
and given by
\[
z = z_{0} + \lambda \omega
\]
where $\lambda \in \mathbb{R}$.
Taking conjugates, we have $\bar z = \bar z_{0} + \lambda \bar \omega$,
so $\bar \omega z - \omega \bar z = \bar \omega z_{0} - \omega \bar z_{0}$.

\subsubsection*{Circles}

Circles are defined by a center $c \in \mathbb{C}$ and a radius $\rho > 0$,
and given by
\[
z = c + \rho e^{i\theta}
\]
where $\theta \in \mathbb{R}$. Note that $|z - c| = \rho$.

\section{Vectors}

A vector is specified by a positive magnitude and a direction in a space.
We can represent a vector as a directed line segment between two points $A$ and $B$,
denoted $\vec v = \overrightarrow{AB}$.
Of we choose $O$ as origin, then point $A$ has \emph{position vector}
$\vec a = \overrightarrow{OA}$.

\subsection{Vector Spaces}

\begin{definition}[Vector Space]
    A vector space over a field $F$ is a set $V$
    together with two binary operations:
    vector addition $+ : V \times V \to V$,
    and scalar multiplication $\times : F \times V \to V$,
    satisfying the following axioms:
    \begin{compactitem}
    \item $V$ forms an abelian group under vector addition.
        we write $\vec{0}$ for the identity in this group.
    \item Compatibility of vector addition and scalar multiplication:
        $a(b\vec{v}) = (ab)\vec{v}$.
    \item Write $1$ for the multiplicative identity in $F$, we need $1\vec{v} = \vec{v}$.
    \item Scalar multiplication distributes over vector addition: $a(\vec{u} + \vec{v}) = a\vec{u} + a\vec{v}$.
    \item and vice versa: $(a + b)\vec{v} = a\vec{v} + b\vec{v}$.
    \end{compactitem}
\end{definition}

If $F = \mathbb{R}$, the vector space is \emph{real}.

% \begin{definition}[Vector Space]
%     A vector space over $\mathbb{R}$ or $\mathbb{C}$
%     is a collection of vectors $\vec v \in V$,
%     together with two operations:
%     addition of two vectors, and multiplication by a scalar (from $\mathbb{R}$ or $\mathbb{C}$, respectively).
% \end{definition}

% To add vectors, join vectors end-to-end.

\begin{center}
    \begin{tikzpicture}[scale=1.5]
        \draw (0,0) coordinate (O) node[below left] {O};
        \draw[->] (O) -- node[pos=0.5, below] {$\vec a$} (0:3) coordinate (A) node[below right] {A};
        \draw[->] (O) -- node[pos=0.5, above left] {$\vec b$} (60:2) coordinate (B) node[above left] {B};
        \draw[dashed] (A) -- ++(B) coordinate (C) node [above right] {C};
        \draw[dashed] (B) -- ++(A);
        \draw[->] (O) -- node[pos=0.5, above] {$\vec c$} (C);
    \end{tikzpicture}
\end{center}

% Note that this definition is close, commutative, associative, 
% has an identity and has an inverse.
% Thus, $(V,+)$ forms an abelian group.

% Scalar multiplication distributes over vector addition, over scalar addition,
% is associative with scalar multiplication, and has an identity of $1$.

\begin{definition}
    A \emph{unit vector} has length $1$.
    We may denote it $\unitvec v$.
\end{definition}

\begin{definition}
    Given vectors $\vec a, \vec b$ and scalars $\alpha ,\beta \in \mathbb{R}$,
    then $\alpha \vec a + \beta \vec b$ is a \emph{linear combination} of $\vec a$ and $\vec b$.
\end{definition}

This definition trivially applies to linear combinations of more than two vectors.

\begin{definition}
    The \emph{span} of a collection of vectors
    is the set formed by their linear combinations,
    $\vecspan\{\vec a, \vec b\}=\{\alpha \vec a + \beta \vec b : \alpha,\beta \in \mathbb{R}\}$
\end{definition}

We say $\vec{a}$ and $\vec{b}$ are parallel,
denoted $\vec{a}\parallel\vec{b}$, if there exists non-zero scalar $\lambda $
such that $a = \lambda b$.
We write $\vec a \nparallel \vec b$ otherwise.
If $\vec a \nparallel \vec b$, then $\vecspan\{\vec a, \vec b\}$ is a plane.

Under these definitions, $\mathbb{R}^{n}$ is a vector space,
where addition is addition by component, and scalar multiplication.

\subsection{Dot Product}

The dot / scalar / inner product,
$\vec{a}\cdot\vec{b}$,
is a binary operation from the vectors
to the scalar space ($\mathbb{R}$ or $\mathbb{C}$).

It is "defined by" $\vec{a}\cdot\vec{b} = |\vec{a}||\vec{b}|\cos\theta$.
If either $\vec{a}$ or $\vec{b}$ is zero, we define $\vec{a}\cdot\vec{b}=0$.

Properties:
\begin{compactitem}
\item The dot product has $\vec{a}\cdot\vec{a}=|\vec{a}|^{2}\ge 0$,
    with equality iff $\vec{a}=\vec{0}$,
\item Associative with scalar multiplication, as in
    $(\lambda \vec{a})\cdot\vec{b}=\lambda (\vec{a}\cdot\vec{b}) = \vec{a}\cdot(\lambda \vec{b})$,
\item Commutative,
\item Distributive over vector addition.
\end{compactitem}

We say that $\vec{a}$ and $\vec{b}$ are \emph{orthogonal},
denoted $\vec{a}\perp\vec{b}$, iff $\vec{a}\cdot\vec{b}=0$.
Note that $\vec{0}$ is thus orthogonal to every vector.

\begin{definition}
    In a real vector space $\mathcal{V}$,
    an inner product is a map $\cdot : V \times V \to \mathbb{R}$
    that satisfies:
    \begin{compactenum}[(i)]
    \item Commutativity: $\vec{a}\cdot\vec{b} = \vec{b}\cdot\vec{a}$.
    \item Linearity in the second argument: $\vec{x}(\lambda \vec{y}+\mu \vec{z}) = \lambda \vec{x}\cdot\vec{y}+\mu \vec{x}\cdot\vec{z}$.
    \item Positive definiteness: $\vec{x}\cdot\vec{x} \ge 0$ with equality iff $\vec{x}=\vec{0}$.
    \end{compactenum}
    It is denoted $\vec{x}\cdot\vec{y}$ or $\langle\vec{x}|\vec{y}\rangle$.
\end{definition}

\begin{definition}
    In a real vector space with an inner product,
    the norm of a vector $\vec{a}$ is $|\vec{a}| = \sqrt{\vec{a}\cdot\vec{a}}$.
\end{definition}

\begin{theorem}[Cauchy-Schwarz Inequality]
    For all $\vec{x},\vec{y} \in \mathbb{R}^{n}$, we have $|\vec{x}\cdot \vec{y}| \le |\vec{x}||\vec{y}|$.
    \begin{proof}
        Consider the expression $|\vec{x} - \lambda \vec{y}|^{2}$, where $\lambda \in \mathbb{R}$.
        Then, mlearly, we have
        \begin{align*}
            |\vec{x} - \lambda \vec{y}|^{2} &\ge 0 \\
            (\vec{x} - \lambda \vec{y})\cdot(\vec{x} - \lambda \vec{y}) &\ge 0\\
            |\vec{x}|^{2} + \lambda^{2}|\vec{y}|^{2} - 2\lambda \vec{x}\cdot\vec{y} &\ge 0\\
            (|\vec{y}|^{2})\lambda ^{2} + (-2\vec{x}\cdot\vec{y})\lambda + |\vec{x}|^{2} & \ge 0\\
        \end{align*}
        Hence we have a quadratic inequality in $\lambda $, with a positive first coefficient.
        Solving for the roots we have
        \[
        \lambda = \frac{2\vec{x}\cdot\vec{y} \pm \sqrt{4(\vec{x}\cdot\vec{y})^{2} - 
        4|\vec{x}|^{2}|\vec{y}|^{2}}}
        {2|\vec{y}|^{2}}.
        \]
        Since the equality holds for all $\vec{x},\vec{y}$, there can
        never be two roots, so by consider the discriminant we have
        \[
        4(\vec{x}\cdot\vec{y})^{2} - 4|\vec{x}|^{2}|\vec{y}|^{2} \le 0.
        \]
        Trivially rearranging we have
        \[
        |\vec{x}\cdot\vec{y}|^{2} \le \left(|\vec{x}||\vec{y}|\right)^{2},
        \]
        so taking square roots on both sides we have $|\vec{x}\cdot\vec{y}| \le |\vec{x}||\vec{y}|$.
    \end{proof}
\end{theorem}
Thus, the inequality holds for all possible scalar products
on any real vector space.

Note that equality holds iff $\vec{x} = \lambda \vec{y}$ or 
$\vec{y} = \lambda \vec{x}$ for some $\lambda  \in \mathbb{R}$.

We can now define $\theta $ by $\vec{x}\cdot\vec{y} = |\vec{x}||\vec{y}|\cos\theta $,
since Cauchy-Schwarz guarantees that $-1 \le \cos\theta \le 1$.
Note that this technically has multivalued $\theta $,
which shouldn't matter since $\theta $ only interacts through trig.

\begin{theorem}[Triangle Inequality]
    $|\vec{x} + \vec{y}| \le |\vec{x}| + |\vec{y}|$
\end{theorem}
\begin{proof}
    \begin{align*}
        |\vec{x}+\vec{y}|^{2} &= (\vec{x} + \vec{y})\cdot(\vec{x} + \vec{y})\\
                              &= |\vec{x}|^{2} + |\vec{y}|^{2} + 2\vec{x}\cdot\vec{y}\\
                              &\le |\vec{x}|^{2}+|\vec{y}|^{2}+2|\vec{x}||\vec{y}|\\
                              &= (|\vec{x}| + |\vec{y}|)^{2}
    \end{align*}
    Thus, $|\vec{x}+\vec{y}| \le |\vec{x}| + |\vec{y}|$.
\end{proof}

\subsection{Orthonormal Bases}

\begin{definition}
    For any set of vectors $\{\vec{e}_i\}$, they are orthonormal iff
    we have $e_i \cdot e_j = \delta _{ij}$. 
\end{definition}
This immediately implies $|e_i| = 1$ and $e_i \cdot e_j = 0$ if $i \ne j$,
i.e. every vector is normalized and they are pairwise orthogonal.

\begin{definition}
    A set of vectors $\{e_i\}$ is a basis of a vector space
    if every vector can be written as a unique linear combination of $\{e_i\}$.
    A basis that is orthonormal is an orthonormal basis.
\end{definition}

Consider $\mathbb{R}^{3}$, and consider vectors $\vec{e}_1, \vec{e}_2, \vec{e}_3$ 
orthonormal to each other. 
This is equivalent to a cartesian axes along these directions.

This forms a basis of $\mathbb{R}^{3}$, specifically an orthonormal basis.

By definition, for any $\vec{a} \in \mathbb{R}^{3}$, we have $\vec{a} = a_i \vec{e}_i$.
We will use row vector and 
column vector notation to write $\vec{a}$ using its components.

Using the inner product in $\mathbb{R}^{3}$, we have
\begin{align*}
    \vec{a}\cdot\vec{b} &= (a_i \vec{e}_i) \cdot (b_j \vec{e}_j)\\
                    &= a_i b_j (\vec{e}_i \cdot \vec{e}_j)\\
                    &= a_i b_j \delta _{ij}\\
                    &= a_i b_i
\end{align*}
In particular, we have $\vec{a}\cdot\vec{a} = |\vec{a}|^{2} = a_i a_i$, 
which is the Pythagorean theorem.

Specifically in $\mathbb{R}^{3}$, we define the 
canonical basis by have any orthonormal basis,
and denoting them $\vec{i}, \vec{j}, \vec{k}$. By definition,
$\vec{i} = (1,0,0)$, $\vec{j} = (0,1,0)$, and $\vec{k} = (0,0,1)$.

\subsection{Vector Product}

The vector product is a product between vectors,
only on $\mathbb{R}^{3}$.

It is defined geometrically by 
$\vec{a}\times\vec{b} = |\vec{a}||\vec{b}|\sin(\theta) \unitvec{n}$,
where $\unitvec{n}$ is a unit vector perpendicular to both $\vec{a}$ and $\vec{b}$.
Note that two such vectors exist, and we will use the right-handed convention.
We may also write $\vec{a}\wedge\vec{b}$ for $\vec{a}\times\vec{b}$.

When $\vec{a}$ and $\vec{b}$ are parallel,
$\unitvec{n}$ is not defined, but
$\sin\theta = 0$, so we define $\vec{a}\times\vec{b} = \vec{0}$.
If $\vec{a}$ or $\vec{b}$ are zero,
then $\theta $ is not defined, but $|\vec{a}|$ or $|\vec{b}|$ is zero,
so we define $\vec{a}\times\vec{b} = \vec{0}$.

Properties:
\begin{compactitem}
\item $\vec{a}\times\vec{b} = -\vec{b}\times\vec{a}$.
\item $\vec{a}\times\vec{a} = \vec{0}$.
\item the only cases when $\vec{a}\times\vec{b} = \vec{0}$ are the special cases.
\item $(\lambda \vec{a})\times\vec{b} = \lambda (\vec{a}\times\vec{b}) = \vec{a}\times(\lambda \vec{b})$.
\item $\vec{a}\times(\vec{b}+\vec{c}) = \vec{a}\times\vec{b} + \vec{a}\times\vec{c}$.
\item $\vec{a}\cdot(\vec{a}\times\vec{b}) = \vec{b}\cdot(\vec{a}\times\vec{b}) = 0$.
\end{compactitem}

Geometrically, the product $\vec{a}\times\vec{b}$ is the
vector area of the parallelogram defined by 
$\vec{0}$, $\vec{a}$, $\vec{a}+\vec{b}$, and $\vec{b}$,
since $|\vec{a}\times\vec{b}| = |\vec{a}||\vec{b}|\sin\theta $.

\end{document}
