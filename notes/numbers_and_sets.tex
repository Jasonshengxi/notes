\documentclass[12pt]{article}
\usepackage[left = 1in, right = 1in, top = 1in, bottom = 1in]{geometry}
\usepackage{textcomp}
\usepackage{gensymb}
\usepackage{paralist}
\usepackage{cancel}
\usepackage{enumitem}
\usepackage{amsmath}
\usepackage{amssymb}
\usepackage{amsthm}
\usepackage{tkz-euclide}
\usepackage{hyperref}
\usepackage{esdiff}
\usepackage{parskip}
\usepackage{accents}
\usepackage{xcolor}

\usetikzlibrary{arrows.meta,positioning}

\newtheoremstyle{customstyle}
  {8pt} % Space above (adjust as needed)
  {0pt} % Space below (adjust as needed)
  {} % Body font
  {} % Indent amount
  {\bfseries} % Theorem head font
  {. } % Punctuation after theorem head
  {0pt} % Space after theorem head
  {} % Theorem head spec
\theoremstyle{customstyle}
\newtheorem{theorem}{Theorem}[section]
\newtheorem{exercise}{Exercise}[section]
\newtheorem{claim}[theorem]{Claim}
\newtheorem{prop}[theorem]{Proposition}
\newtheorem{corollary}[theorem]{Corollary}
\newtheorem{lemma}[theorem]{Lemma}
\newtheorem{definition}[theorem]{Definition}
\newtheorem{question}{Question}
\newtheorem{subquestion}{Part}[question]

\def\definitionautorefname{Definition}
\def\corollaryautorefname{Corollary}

\newenvironment{nonproof}{\par $\cancel {\text{\textit{Proof}}}.$}{\hfill$\cancel\square$}

\def\contra{\tikz[baseline, x=0.22em, y=0.22em, line width=0.032em]\draw (0,2.83)--(2.83,0) (0.71,3.54)--(3.54,0.71) (0,0.71)--(2.83,3.54) (0.71,0)--(3.54,2.83);}
\newenvironment{answer}{\par\noindent\textit{Answer.}}{\par}

\renewcommand{\CancelColor}{\color{red}}
\renewcommand{\Re}{\operatorname{Re}}
\renewcommand{\Im}{\operatorname{Im}}
\renewcommand{\bar}{\overline}
% \renewcommand{\vec}[1]{\undertilde{\mathrm{#1}}}
% \renewcommand{\vec}[1]{\undertilde{#1}}
% \renewcommand{\vec}[1]{\underline{#1}}
\renewcommand{\vec}[1]{\mathbf{#1}}
\newcommand{\unitvec}[1]{\hat{\vec{#1}}}
\newcommand{\sol}{$\operatorname{sol}^{\simeq}$}
\newcommand{\Arg}{\operatorname{Arg}}
\newcommand{\Log}{\operatorname{Log}}
\newcommand{\vecspan}{\operatorname{span}}
\newcommand{\id}[1]{\operatorname{id}_{#1}}
\newcommand{\indic}[1]{i_{#1}}
\newcommand{\Sym}{\operatorname{Sym}}
\newcommand{\Isom}{\operatorname{Isom}}
\newcommand{\di}{\mathrm{d}}
\newcommand{\gen}[1]{\langle{#1}\rangle}

\definecolor{applegreen}{rgb}{0.55, 0.71, 0.0}
\definecolor{ufogreen}{rgb}{0.24, 0.82, 0.44}


\begin{document}
% Lectured by Dr Zoe Wyatt (zw253)

\section{Introduction}

Expect:
\begin{compactitem}
    \item precise definitions
    \item rigorous proofs
    \item foundational questions
\end{compactitem}

We start with assumptions called \emph{axioms}.

A \emph{statement} is a sentence that can have a true or false value.
A \emph{proof} is a sequence of true statements without logical gaps
establishing some conclusion.

\begin{compactitem}
\item show they are true
\item gain insight into why they are true
\item the proof might be cool
\end{compactitem}

\subsection{Number Systems}

Define $\mathbb{N}$ as the set of natural numbers, $\{1,2,3,\cdots\}$ (note the lack of $0$).
Define $\mathbb{Z},\mathbb{Q},\mathbb{R}$ to be 
integer, rational, and real sets respectively.

A real number is \emph{algebraic} if it is the
root of some polynomial with integer coefficients.
Non-algebraic numbers are called \emph{transcendental}.

The existence of a transcendental number was shown in 1844.

\subsection{Some Proofs and Nonproofs}

\begin{claim}
    For all positive integers $n$, $n^{3} - n$ is always a multiple of $3$.
\end{claim}
\begin{proof}
    Let $n \in \mathbb{N}$, we have $n^{3} - n = n(n^{2} - 1) = (n - 1) \times n \times (n+1)$.
    One of the three consecutive integers $n-1,n,n+1$ must be a multiple of $3$.
    Thus, the product $n^{3} - n$ must also be a multiple of $3$.
\end{proof}

Note the little box given for free by \LaTeX, on the right side of the page.
That denotes the end of the proof.

\begin{claim}
    For any positive integer $n$, if $n^{2}$ is even then so is $n$.
\end{claim}
\begin{nonproof}
    Given $n \in \mathbb{N}$, we can write $n = 2k$ where $k \in \mathbb{N}$.
    We have $n^{2} = 4k^{2} = 2(2k^{2})$, which is even.
\end{nonproof}

\begin{proof}
    Suppose on the contrary that $n^{2}$ is even so $n = 2k+1$ for some $k \in \mathbb{N}$.
    Then we have,
    \[
    \begin{aligned}
        n^{2} = (2k+1)^{2} &= 4k^{2} + 4k + 1 \\
                     &= 4(k^{2} + k) + 1.
    \end{aligned}
    \]
    which is odd, contradicting the assumption that $n^{2}$ is even.
    \contra
\end{proof}

\begin{claim}
    The \sol{} to $x^{2} - 5x + 6 = 0$ is $x = 2$ or $x = 3$.
\end{claim}
\begin{proof}
    If $x = 2$ or $x = 3$,
    then $x - 2 = 0$ or $x - 3 = 0$.
    So $(x-2)(x-3) = x^{2} - 5x + 6 = 0$.

    If $x^{2}-5x+6=0$, then
    $(x-2)(x-3)=0$,
    so either $x-2=0$ or $x-3=0$,
    which yields $x=2$ or $x=3$.
\end{proof}

Alternatively, we can write
\begin{proof}
    \begin{align*}
        &x=2 \quad \text{or} \quad x=3\\
        \iff &x-2=0 \quad\text{or}\quad x-3=0 \\
        \iff &(x-2)(x-3)=0\\
        \iff &x^{2}-5x+6=0
    \end{align*}
\end{proof}

\begin{claim}
    Every positive real number is greater than or equal to $1$.
\end{claim}
\begin{nonproof}
    Let $r$ be the least positive real.
    Either $r = 1$ or $r < 1$ or $r > 1$.

    If $r < 1$, then $0 < r^{2} < r$, 
    so $r^{2}$ is a smaller positive real. \contra

    If $r > 1$, then $0 < \sqrt{r} < r$,
    so $\sqrt{r}$ is a small positive real. \contra
    
    Thus, $r = 1$.
\end{nonproof}

The problem lies in the nonexistence of a least positive real.
The moral is that
\begin{center}
    \underline{Every claim must be justified}.
\end{center}

\subsection{Combining Claims}

The truth of assertions like $A \land B$ and $A \lor B$ depend on
the truth of $A$ and $B$, as summarised in the \emph{truth table}:

\begin{table}[h]
    \centering
    \begin{tabular}{ |c|c|c|c|c|c|c| }
        \hline
        A & B & $A \land B$ & $A \lor B$ & $\neg A$ & $A \Rightarrow B$ \\
        \hline
        F & F & F & F & T & T\\
        F & T & F & T & T & T\\
        T & F & F & T & F & F\\
        T & T & T & T & F & T\\
        \hline
    \end{tabular}
    \caption{The truth table of various logical constructs.}
\end{table}

Note, for example, that $\neg (A \land B)$ is equivalent to
$(\neg A) \lor (\neg B)$, by comparing truth tables.
Similarly, $A \Rightarrow B$ is equivalent to $(\neg A) \lor B$,
and hence $B \lor (\neg A)$,
and hence to $(\neg B) \Rightarrow (\neg A)$.
This is called the \emph{contrapositive}.

\subsubsection*{Negating Quantifiers}

A claim may involve "quantifiers" like $\forall$ or $\exists$.
$\neg(\forall x, A(x))$ is equivalent to $\exists x, \neg A(x)$.
Similarly, $\neg(\exists x, A(x)) \iff \forall x, \neg A(x)$.

\section{Sets, Functions, and Relations}

\subsection{Sets}

A \emph{set} is a collection of mathematical objects.
For example, $\mathbb{R}$, $\mathbb{N}$, $\{1,5,9\}$, $(-2,3]$.

Two important facts required for a set are that 
\begin{compactenum}
\item the order of elements in the set is immaterial, and
\item each element in the set occurs only once.
\end{compactenum}

For instance, $\{1,3,7\} = \{1,7,3\}$, and $\{3,4,4,8\} = \{3,4,8\}$.

\begin{definition}
    Two sets are equal if they have the same elements. 
    That is, $A = B$ iff $\forall x, x \in A \iff x \in B$.
\end{definition}
There is only one \emph{empty set} $\emptyset$ i.e. the set with no elements.

\begin{definition}
    A set $B$ is a \emph{subset} of $A$, written $B \subseteq A$, or $B \subset A$,
    if every element of $B$ is an element of $A$, or equivalently $\forall x \in B, x \in A$.
    $B$ is said to be a \emph{proper subset of A} if $B \subseteq A$ and $B \ne A$,
    sometimes written $B \subsetneq A$.
\end{definition}

Note that $A = B$ iff $A \subseteq B$ and $B \subseteq A$.


For example, $\{u \in \mathbb{N} : n\text{ is prime}\} = \{2,3,5,7,11,\cdots\}$.

\begin{definition}
    For $A,B$ sets, their \emph{union} $A \cup B$ is
    given by $\{x : x \in A \lor x \in B\}$.
    Their \emph{intersection} $A \cap B$ is defined to be
    $\{x : x \in A \land x \in B\}$.
    We say $A$ and $B$ are \emph{disjoint} if $A \cap B = \emptyset$.
\end{definition}

Note that we can view intersection as a special case of subset selection,
in that $A \cap B = \{x \in A : x \in B\}$.

\begin{definition}
    For $A,B$ sets, their \emph{set difference} $A \setminus B$ is defined by $\{x \in A : x \notin B\}$.
\end{definition}
Note that set difference is non-commutative, whereas $\cap$ and $\cup$ 
are commutative and associative.
Interestingly, $\cup$ and $\cap$ are distributive over each other, in that
\begin{align*}
    A \cup (B \cap C) = (A \cup B) \cap (A \cup C),\\
    A \cap (B \cup C) = (A \cap B) \cup (A \cap C).
\end{align*}
Also,
\begin{align*}
    A \setminus (B \cap C) = (A \setminus B) \cup (A \setminus C),\\
    A \setminus (B \cup C) = (A \setminus B) \cup (A \setminus C).
\end{align*}

Let's prove one of them.
\begin{theorem}
    If $A,B,C$ are sets, we have $A \cap (B \cup C) = (A \cap B) \cup (A \cap C)$.
\end{theorem}
\begin{proof}
    Let $x \in A \cap (B \cup C)$.
    Then we have $x \in A$ and $x \in B \cup C$,
    which means that $x \in A$ and also either $x \in B$ or $x \in C$.
    If $x \in B$, since $x \in A$, we have $x \in A \cap B$.
    Similarly, if $x \in C$, we have $x \in A \cap C$.
    Thus, $x \in A \cap B$ or $x \in A \cap C$, so $x \in (A \cap B) \cup (A \cap C)$.
    Thus, we have $A \cap (B \cup C) \subseteq (A \cap B) \cup (A \cap C)$.

    Conversely, if $x \in (A \cap B) \cup (A \cap C)$,
    then either $x \in A \cap B$ or $x \in A \cap C$.
    If $x \in A \cap B$, then we have both $x \in A$ and $x \in B$,
    so $x \in B \cup C$, and therefore $x \in A \cap (B \cup C)$.
    Similarly, if $x \in A \cap C$, then
    $x \in B \cup C$ anyway, and $x \in A \cap (B \cup C)$.
    Thus, we have $(A \cap B) \cup (A \cap C) \subseteq A \cap (B \cup C)$.

    Finally, $(A \cap B) \cup (A \cap C) = A \cap (B \cup C)$.
\end{proof}

\begin{definition}
    If $A_{1},A_{2},A_{3},\cdots$ are sets, then
    \begin{align*}
        \bigcap_{n=1}^{\infty} A_n &= A_{1} \cap A_{2} \cap A_{3} \cap \cdots\\
                                   &= \{x : x \in A_n \ \forall n \in \mathbb{N}\}
    \end{align*}
    and similarly,
    \begin{align*}
        \bigcup_{n=1}^{\infty} A_n &= A_{1} \cup A_{2} \cup A_{3} \cup \cdots\\
                                  &= \{x : x \in A_n \text{ for some } n \in \mathbb{N}\}
    \end{align*}
\end{definition}
Note that these definitions do not utilize any limits.
This definition can be trivially applied to
a collection of sets $A_i$ indexed by $i \in I$,
this time denoted $\bigcap\limits_{i \in I} A_i$.

\begin{definition}
    Given sets $A$ and $B$, we can form their \emph{cartesian product} by
    \[
        A \times B = \{(a,b) : a \in A, b \in B\}.
    \]
    which is the set of \emph{ordered} pairs $(a,b)$ with $a \in A$ and $b \in B$.
\end{definition}

Note that $(a,b)$ can be defined by $\{\{a,b\},a\}$. 
Similarly, we can have cartesian product of more, using ordered $n$-ples.

\begin{definition}
    The \emph{power set} of a set $X$, denoted $\mathcal{P}(X)$,
    is the set of all subsets of $X$,
    \[
    \mathcal{P}(X) = \{Y : Y \subseteq X\}
    \]
\end{definition}
For example, if $X=\{a,b\}$, we have $\mathcal{P}(X) = \{\emptyset,\{a\},\{b\},X\}$.

All subset selections must be confined, in that
$\{x : P(x)\}$ must be specified by $\{x \in A : P(x)\}$.
In fact, suppose we have $X=\{x : x\text{ is a set}, x \notin x\}$.
We ask if $X \in X$. If so, then $X \notin X$ is false,
so $X \notin X$. But if not, then $X \notin X$ is true,
so $X \in X$. Thus, two contradictions appear.
This is known as \emph{Russell's Paradox}.

Another deduction we can make from the paradox is that
there is no \emph{universal set}, a set $Y$ that contains
all mathematical objects, i.e. $\forall x, x \in Y$.
If such a set exists, then a contradiction occurs
via Russell's Paradox.

Thus, to guarantee that a set exists,
it should be obtained via known sets.

\begin{definition}
    Write $\mathbb{N}_{0} = \mathbb{N} \cup \{0\} = \{0,1,2,3,\cdots\}$.
    Given $n \in \mathbb{N}_{0}$, we can say a set $A$
    has \emph{size} $n$ if we can write $A = \{a_{1},a_{2},\cdots,a_n\}$,
    with the elements $a_i$ distinct.
\end{definition}

\begin{definition}
    We say $A$ is \emph{finite} if there exists $n \in \mathbb{N}_{0}$
    such that $A$ has size $n$.
    Otherwise, we say that $A$ is \emph{infinite}.
\end{definition}

\subsection{Functions}

Informally, given sets $A,B$, a function $f$ from $A$ to $B$,
is a "rule" that assigns to every $x \in A$ a unique $f(x) \in B$.
More formally,

\begin{definition}
    A \emph{function} from $A$ to $B$, denoted $f: A \to B$ is a subset $f \subseteq A \times B$
    such that for all $x \in A$, 
    there is a unique $y \in B$ such that $(x, y) \in f$.
    If $(x,y) \in f$, we write $f(x) = y$, or $x \mapsto y$.
\end{definition}

Exmaples:
\begin{compactenum}[(1)]
\item $f: \mathbb{R} \to \mathbb{R}$, where $x \mapsto x^{2}$, is a function.
\item $f: \mathbb{R} \to \mathbb{R}$, where $x \mapsto x^{-1}$, is not a function, 
    since it is not defined at $0$.
\item $f: \mathbb{R} \to \mathbb{R}$, where $x \mapsto \pm\sqrt{|x|}$, is not a function,
    since it is multi-valued.
\item $f: \mathbb{R} \to \mathbb{R}$ yielding $1$ if $x \in \mathbb{Q}$ and $0$ otherwise,
    is a function.
\end{compactenum}

\begin{definition}
    Given $f: A \to B$, we say $A$ is the \emph{domain of $f$}
    and $B$ is the \emph{range of $f$}, or codomain of $f$.
\end{definition}

\begin{definition}
    Given $f: A \to B$ and $x,y \in A$ such that $f(x) = y$,
    we say $y$ is the \emph{image} of $x$,
    and $x$ is a \emph{preimage} of $y$.
\end{definition}

Note that images are unique but preimages are not. 
For example, if $f(x) = x^{2}$, then
the image of $6$ is $36$, but $36$ has two preimages, $6$ and $-6$.

\begin{definition}
    Given $f: A \to B$ and $X \subseteq A$,
    the \emph{image of $X$ under $f$} is
    \[
        f(X) = \{f(x) : x \in X\} = \{b \in B : f(x) = b \text{ for some } x \in X\}.
    \]
\end{definition}

Here is where some more stuff about functions
should've been defined, like surjection, injection,
and bijection, if I didn't wake up 40 minutes late.

Define the indicator function $\indic A$:
\[
i_A(x) = \begin{cases}
    1 & x \in A\\
    0 & \text{otherwise}\\
\end{cases}
\]
It has the following properties:
\begin{compactenum}[(i)]
\item $\indic A = \indic B \iff A = B$.
\item $\indic{A\cap B} = \indic A \indic B$.
\item $\indic{X\setminus A} = 1 - \indic A$.
\item $\indic {A\cup B} = \indic A + \indic B - \indic {A \cap B}$.
\end{compactenum}


\section{Integers and Counting}

\section{Elementary Number Theory}
\section{Real Numbers}
\section{(Un)Countability}

\appendix

\newpage
\section{Notation}

We write $A \implies B$ for "if A then B".
We write \sol{} for a solution, as in the \sol{} of a polynomial, or $3$ is a \sol{}.

If $A$ and $B$ are assertions,
we can (but usually don't) write $A \land B$ for "A and B",
and $A \lor B$ for "A or B".
We can similarly write $\neg A$ for "not A".

We write $x \in A$ if $x$ is an element of the set $A$, and $x \notin A$ if not.

If $A$ is a set and $P$ is a property of (some) elements of $A$,
we can write $\{x \in A : P(x)\}$ for the subset of $A$ comprising
of the elements $x$ for which $P(x)$ holds true.

We use the notation $\id X : X \to X$ for the identity function on the set $X$,
and $\indic A : X \to \{0,1\}$ for the indicator function of $A \subseteq X$.

\end{document}
