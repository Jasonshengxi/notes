\documentclass[12pt]{article}
\usepackage[left = 1in, right = 1in, top = 1in, bottom = 1in]{geometry}
\usepackage{paralist}
\usepackage{cancel}
\usepackage{enumitem}
\usepackage{amsmath}
\usepackage{amssymb}
\usepackage{amsthm}
\usepackage{tkz-euclide}
\usepackage{hyperref}
\usepackage{esdiff}
\usepackage{parskip}
\usepackage{accents}
\usepackage{xcolor}

\usetikzlibrary{arrows.meta,positioning}

\newtheoremstyle{customstyle}
  {8pt} % Space above (adjust as needed)
  {0} % Space below (adjust as needed)
  {} % Body font
  {} % Indent amount
  {\bfseries} % Theorem head font
  {. } % Punctuation after theorem head
  {0pt} % Space after theorem head
  {} % Theorem head spec
\theoremstyle{customstyle}
\newtheorem{theorem}{Theorem}[section]
\newtheorem{corollary}{Corollary}[theorem]
\newtheorem{lemma}{Lemma}[section]
\newtheorem{definition}{Definition}[section]

\def\definitionautorefname{Definition}
\def\corollaryautorefname{Corollary}

\newenvironment{nonproof}{\par \textit{Nonproof:}}{\hfill$\cancel\square$}

\def\contra{\tikz[baseline, x=0.22em, y=0.22em, line width=0.032em]\draw (0,2.83)--(2.83,0) (0.71,3.54)--(3.54,0.71) (0,0.71)--(2.83,3.54) (0.71,0)--(3.54,2.83);}

\renewcommand{\Re}{\operatorname{Re}}
\renewcommand{\Im}{\operatorname{Im}}
\renewcommand{\bar}{\overline}


\begin{document}
% Lectured by Henry Wilton (hjrw2)

\section{Examples and Definitions}

You can think of groups in two ways:
It is either something to do with algebra,
or something to do with symmetry.

We will begin with the symmetry point of view.

\subsection{Symmetry}

An equilateral triangle has at least two different kinds of symmetry.
It has \emph{rotational} and \emph{reflective} symmetry,
and most importantly, the identity symmetry.

\begin{figure}[h]
    \centering
    \begin{tikzpicture}[scale=3]
        \draw (90:1) coordinate (A) 
            -- (210:1) coordinate (B) -- (330:1) coordinate (C) -- cycle;
    \end{tikzpicture}
    \caption{The "identity" symmetry}
\end{figure}

Combining all symmetries, there are 6 different symmetries 
on an equilateral triangle:
\begin{compactenum}
    \item identity,
    \item rotate by $120\degree$ anticlockwise,
    \item rotate by $240\degree$ anticlockwise
    \item 5. 6. reflections along the three axes.
\end{compactenum}

\begin{exercise}
    How many symmetries does a regular $n$-gon have?
\end{exercise}
\begin{answer}
    the number of symmetries of a regular $n$-gon is $2n$.
\end{answer}

Now we begin to consider the \emph{composition} of symmetries,
or, from an algebraic standpoint, the \emph{product} of symmetries.

Say we compose a rotation by $120\degree$ anticlockwise and
a flip around the vertical axis. This is a reflection around 
the axis of positive slope.

Some important features to notice from this are:
\begin{compactitem}
\item Symmetries can be composed, 
    i.e. symmetries are closed under composition.
\item There is an identity symmetry that leaves everything the same.
\item Every symmetry has an inverse. Specifically, in this case:
    \begin{compactitem}
    \item the inverse of the identity is itself.
    \item the inverse of a rotation is the opposite rotation.
    \item the inverse of a reflection is itself.
    \end{compactitem}
\item Composition of symmetries is \emph{associative},
    as in $a(bc) = (ab)c$, but for symmetries.
\end{compactitem}
Additionally, there is one important non-feature: 
symmetries are not always commutative. 
It is trivial to check by multiplying a rotation by a reflection,
then the other way around.

But isn't this whole drawing thing exhausting (especially for me)?
Let's go to the algebra side.

\subsection{Algebra}

\begin{definition}
    A binary operation on a set $X$ is
    a function $f : X \times X \to X$.
    It is also denoted $\cdot$ for multiplication.
    We write $a \cdot b$ for $f(a,b)$.
\end{definition}

Note that this definition includes closure,
in that for any $x,y \in X$, we have $f(x,y) \in X$.

\begin{definition}
    A group is a triple $(G,\cdot,e)$ where
    $G$ is a set, $\cdot$ is a binary operation on $G$,
    and $e \in G$, which satisfies the following four axioms:
    \begin{compactenum}[(i)]
    \item Closure: for all $a,b \in G$, we have $a \cdot b \in G$.
    \item Associativity: for all $a,b,c \in G$, we have $(a \cdot b) \cdot c = a \cdot (b \cdot c)$.
    \item (Right) Identity: for all $a \in G$, we have $a \cdot e = a$.
    \item (Right) Inverse: for all $a \in G$, there exists $a^{-1} \in G$ s.t. $a \cdot a^{-1} = e$.
    \end{compactenum}
\end{definition}

Note that the closure axiom is indeed redundant,
by the definition of a \emph{binary operation on G}.

Example: Notice that the set of symmetries of 
the equilateral triangle forms a group.

We can also think of this definition as
encompassing \emph{algebra with one operation}.

\begin{exercise}
    $(\mathbb{Z},+,0)$ forms a group.
\end{exercise}
\begin{proof}
    Closure and identity and associativity are trivial.
    The inverse of $x \in \mathbb{Z}$ is $-x$, in that $x + (-x) = 0$.
\end{proof}

\begin{theorem}
    \label{thm:right_is_left}
    Let $(G,\cdot,e)$ be a group, and $a,b,b'e' \in G$. We have:
    \begin{compactenum}[(i)]
    \item If $a \cdot b = e$ then $b \cdot a = e$. \\
        (i.e. right inverses are left inverses)
    \item If $e \cdot a = a$. \\
        (i.e. right identities are left identities)
    \item If $a,b,b'$ are s.t. $a \cdot b = e = a \cdot b'$,
        then $b = b'$. \\
        (i.e. inverses are unique)
    \item If $a \cdot e' = a$ then $e' = e$.\\
        (i.e. the identity is unique).
    \end{compactenum}
\end{theorem}

The proof is given in \autoref{sec:proof_of_right_is_left}.
Since the inverse is unique, we denote $b \in G$ s.t. $a \cdot b = e$ as $a^{-1} = b$.
We have then that
\[
    a \cdot a^{-1} = e = a^{-1} \cdot a.
\]
It is also clear from that equation that $a = (a^{-1})^{-1}$.
Now the "to the power of $-1$" is defined, let us also define the rest.
\begin{definition}
    We define the powers of $a$, such that
    \begin{compactitem}
    \item $a^{0} = e$,
    \item $a^{n} = a^{n-1} \cdot a$ for any $n \in \mathbb{N}$.
    \item $a^{-n} = (a^{-1})^{n}$ for any $n \in \mathbb{N}$.
    \end{compactitem}
\end{definition}
Thus, we have defined $a^{n}$ for all $n \in \mathbb{Z}$.

\begin{exercise}
    Show that, for all $a \in G$ and $m,n \in \mathbb{Z}$, we have
    \[
        a^{m+n} = a^{m}\cdot a^{n} 
        \quad \text{and} \quad
        (a^{m})^{n} = a^{mn}
    \]
\end{exercise}
\begin{nonproof}
    TODO
\end{nonproof}

Note that much alike how it is not necessarily true that
$a \cdot b = b \cdot a$, it is also not necessarily true that
$(a \cdot b)^{-1} = a^{-1} \cdot b^{-1}$. Instead, it is always true that
\begin{align*}
    (a \cdot b) \cdot (b^{-1} \cdot a^{-1})
    &= a \cdot b \cdot b^{-1} \cdot a^{-1}\\
    &= a \cdot e \cdot a^{-1}\\
    &= a \cdot a^{-1}\\
    &= e.
\end{align*}
Thus, we have
\[
    (a \cdot b)^{-1} = b^{-1} \cdot a^{-1}.
\]
\begin{definition}
    An \emph{abelian} group $(G,\cdot,e)$ is such that
    for any $a,b \in G$, we have
    $a \cdot b = b \cdot a$.
\end{definition}

For example, let $G = \{e\}$, define $\cdot$ by $e \cdot e = e$,
then $(G,\cdot,e)$ is an abelian group, and 
is generally known as the \emph{trivial} group.

For some less trivial examples,
the groups $(\mathbb{Z},+,0)$, $(\mathbb{Q},+,0)$, $(\mathbb{R},+,0)$, $(\mathbb{C},+,0)$
are all abelian groups, and $x^{-1}$ is defined by $-x$.

Things that are not groups include:
$(\mathbb{Z},\times,1),(\mathbb{N},\times,1),(\mathbb{Q},\times,1)$, 
but for differing reasons.
Specifically, $(\mathbb{Q} \setminus \{0\}, \times, 1)$ is an abelian group,
and remains a group if $\mathbb{Q}$ is replaced by $\mathbb{R}$ or $\mathbb{C}$.

\begin{definition}
    The \emph{order} of a group $(G,\cdot,e)$ is the 
    number of elements in $G$,
    denoted by $|G|$ or $\#G$.
    If $|G|$ is finite then $(G,\cdot,e)$ is a \emph{finite group}.
\end{definition}

For any $n \in \mathbb{N}$, define $C_n = \{z \in \mathbb{C} : z^{n} = 1\}$, then $(C_n,\times,1)$ is an abelian group.

Let $Z_n = \{0,1,\cdots,n-1\}$, define $a +_{n} b$ to be the remainder
of $a + b$ when divided by $n$.

\begin{exercise}
    Show that $(Z_n, +_{n}, 0)$ is an abelian group.
\end{exercise}
\begin{proof}
    TODO
\end{proof}

\subsection*{Symmetric Groups}

\begin{definition}
    Let $X,Y$ be sets. A \emph{bijection} is a
    map $f:X\to Y$ that has an \emph{inverse} $g:Y\to X$,
    such that $f \circ g = \id Y$ , and $g \circ f = \id X$.
    A bijection from $X$ to itself is called a \emph{permutation}.
\end{definition}

\begin{definition}
    $\Sym(X)$ is defined to be the set of permutations of a set $X$.
\end{definition}

\begin{lemma}[Composition is associative]
    Consider the following maps: $W \stackrel f{\to} X \stackrel g\to Y \stackrel h\to Z$. Then $(h \circ g) \circ f = h \circ (g \circ f)$.
\end{lemma}
\begin{proof}
    Let arbitrary $w \in W$,
    $(h\circ g)\circ f (w) = (h\circ g)(f(w)) = h(g(f(w))) = h(g \circ f(w)) = h\circ(g\circ f)(w)$.
\end{proof}

\begin{prop}
    $(\Sym(X),\circ,\id X)$ is a group, 
    called the symmetric group of $X$, denoted $S_X$.
\end{prop}
\begin{proof}
    Associativity follows from the lemma,
    inverse follows by definition.
\end{proof}

\begin{definition}
    We write $S_n = \Sym(X)$ where $X = \{1,2,\cdots,n\}$.
\end{definition}

Note that $|S_n| = n!$.

\subsection{Subgroups}

\begin{definition}
    Let $G$ be a group and $H \subseteq G$, with the following properties:
    \begin{compactitem}
    \item $e \in H$,
    \item $ab \in H$ for all $a,b \in H$,
    \item $a^{-1} \in H$ for all $a \in H$.
    \end{compactitem}
    Then we say $H$ is a \emph{subgroup} of $G$, denoted $H \le G$.
\end{definition}

Trivially, since $G$ is a group, associativity of
the operation in $H$ is given, so $H$ forms a group under $\cdot_G$, 
with the same identity $e$.

For example, trivially, $G$ is a subgroup of itself, and $\{e\}$ is a subgroup of $G$.
The subgroup $\{e\}$ is called the \emph{trivial subgroup}.
A subgroup that is not either of those is called a \emph{proper} subgroup.
Trivially, we also have $\mathbb{Z} \le \mathbb{Q} \le \mathbb{R} \le \mathbb{C}$.

Define $n\mathbb{Z}$ to be $\{nk : k \in \mathbb{Z}\}$. We have $n\mathbb{Z} \le \mathbb{Z}$,
since the identity element $0 = 0n \in n\mathbb{Z}$,
the sum of two elements $nk_{1} + nk_{2} = n(k_{1}+k_{2}) \in n\mathbb{Z}$,
and the inverse of an element $-nk = n(-k) \in n\mathbb{Z}$.

\begin{prop}
    If $H \le \mathbb{Z}$, then $H = n\mathbb{Z}$ for some $n \in \mathbb{N}_0$.
\end{prop}
\begin{proof}
    If $H = \{0\}$, then $H = 0\mathbb{Z}$, and we are done.
    If $H \ne \{0\}$, then there exists $a \in H$ such that $a \ne 0$.
    if $a > 0$, there exists a positive integer in $H$.
    if $a < 0$, then $-a \in H$ has $-a > 0$, which is a positive integer in $H$.
    Thus, let $n \in H \setminus \{0\}$ be the smallest positive element of $H$.

    By induction, $nk \in H$ for $k \in \mathbb{N}$.
    But taking the inverse for each element, we have $(-nk) = n(-k) \in H$ for $k \in \mathbb{N}$.
    Thus, $n\mathbb{Z} \le H$.

    Suppose $n\mathbb{Z} \ne H$, so there exists $x \in H$ st $x \ne n\mathbb{Z}$.
    Divide $x$ by $n$ and taking remainders, we have
    $x = nq + n$ for some $q \in \mathbb{Z}$ and $0 < r < n$.
    But we have $r = x - nq$, noting that
    $x \in H$ and $nq \in n\mathbb{Z} \le H$.
    Thus, $r \in H$. However, $0 < r < n$,
    which is a smaller positive element of $H$. \contra
\end{proof}

\begin{prop}
    For any family of subgroups $H_i \le G$,
    \[
        \bigcap_i H_i = \{a \in G : a \in H_i \text{ for all } i\} \le G
    \]
\end{prop}
\begin{proof}
    For partial proof, see Example Sheet 1, Question 2.
    The entire proof is trivial after that.
\end{proof}

\begin{definition}
    Let $X$ be a subset of group $G$. Then
    we define $\gen X$ to be
    \[
        \gen X = \bigcap_{X \le H \le G} H
    \]
    And we call this the \emph{subgroup generated by $X$}.
\end{definition}
Intuitively, $\gen X$ is the smallest subgroup
containing $X$.
If $G = \gen X$, we say that $X$ \emph{generates} $G$,
or that $X$ is a \emph{generating set} for $G$.

\begin{prop}
    If $X$ generates $G$, that means that every $g \in G$
    can be written as
    \[
    g = x_{1}^{\pm 1}x_{2}^{\pm 1}\cdots x_{n}^{\pm 1}
    \]
    for some $n \ge 0$, where $x_i$ not necessarily distinct have $x_i \in X$.
\end{prop}
\begin{proof}
    See Example Sheet 1, Question 9.
\end{proof}

\subsection{Geometric Examples}

Let $\mathbb{C}$ be the 2D plane, 
equipped with the usual notion of distance.

\begin{tikzpicture}
    \draw[->] (-1,0) -- (6,0) node [right] {Re};
    \draw[->] (0,-1) -- (0,6) node [above] {Im};
    \draw (5, 1) coordinate (w) 
        node[draw, circle, fill=black, inner sep=0.03cm] {} 
        node[right] {$w = w_{1} + iw_{2}$};
    \draw (1, 5) coordinate (z) 
        node[draw, circle, fill=black, inner sep=0.03cm] {}
        node[above right] {$z = z_{1} + iz_{2}$};
    \draw (w) -- (z);
\end{tikzpicture}

\begin{definition}
    For any subset $X \subseteq \mathbb{C}$,
    an \emph{isometry} of $X$ is a bijection
    $f : X \to X$ that preserves distance,
    in that for any $z,w \in X$,
    $|z-w| = |f(z)-f(w)|$.
\end{definition}

Since any isometry is a bijection of a set $X$,
it is an element of the symmetric group $\Sym(X)$.

\begin{prop}[Isometry Groups in $\mathbb{C}$]
    Let $X \subseteq \mathbb{C}$, The set of isometries of $X$,
    $\Isom(X)$, is a subgroup of $\Sym(X)$.
    In particular, $\Isom(X)$ is a group.
\end{prop}
\begin{proof}
    Let $f,g \in \Isom(X)$, and $x,y \in X$. 
    We will now check the axioms:
    \begin{compactenum}
    \item The identity of $\Sym(X)$ is $\id X$, 
        which has $|\id X(x) - \id X(y)| = |x - y|$ by definition.
        Thus, $\id X \in \Isom(X)$.
    \item Since $f,g$ are isometries, we have
        $|f \circ g(x) - f\circ g(y)| = |f(g(x)) - f(g(y))| = |g(x) - g(y)| = |x - y|$.
        Thus, $f\circ g$ is an isometry.
    \item We have
        $|f^{-1}(x) - f^{-1}(y)| = |f(f^{-1}(x)) - f(f^{-1}(y))| = |x - y|$,
        so $f^{-1}$ is an isometry.
    \end{compactenum}
    Thus, all axioms of subgroups are satisfied.
\end{proof}

\begin{definition}[Dihedral Groups]
    Let $X_n\subseteq \mathbb{C}$, where $n \ge 3$, be the vertices of the
    regular $n$-gon with vertices $\{\exp(2\pi i k / n) : k = 0,\cdots,n-1\}$.
    \begin{center}
        \begin{tikzpicture}[scale=3]
            \draw[->] (-1.5, 0) -- (1.5, 0) node[below] {Re};
            \draw[->] (0, -1.5) -- (0, 1.5) node[right] {Im};
            \draw (0,0) circle[radius=1cm];
            \foreach \x in {0,...,5}
            {
                \draw ({360/6 * \x}:1) -- ({(\x + 1)*360/6}:1);
                \draw ({360/6 * \x}:1) 
                    node[draw, fill=black, circle, inner sep=0.04cm] {}
                    % ({360/6 * \x}:1.2)
                    % node {$z_\x$}
                    ;
            }
        \end{tikzpicture}
    \end{center}
    We define $D_{2n}$ to be $\Isom(X_n)$, the $n$th dihedral group.
\end{definition}

\begin{theorem}
    $|D_{2n}| = 2n$
\end{theorem}

\begin{lemma}[Kite Lemma]
    Let $x_{1},x_{2},y_{1},y_{2} \in \mathbb{C}$.
    If $|y_{1} - x_{1}| = |y_{2} - x_{1}|$
    and $|y_{1} - x_{2}| = |y_{2} - x_{2}|$,
    then $x_{2} - x_{1}$ is perpendicular to $y_{2} - y_{1}$.
\end{lemma}
\begin{center}
    \begin{tikzpicture}[scale=2]
        \draw
            (-1,0) coordinate (x1)
            node[circle, draw, fill=black, inner sep=0.04cm] {}
            node[left] {$x_{1}$};
        \draw
            (3,0) coordinate (x2)
            node[circle, draw, fill=black, inner sep=0.04cm] {}
            node[right] {$x_{2}$};
        \draw
            (0,1) coordinate (y1)
            node[circle, draw, fill=black, inner sep=0.04cm] {}
            node[above] {$y_{1}$};
        \draw
            (0,-1) coordinate (y2)
            node[circle, draw, fill=black, inner sep=0.04cm] {}
            node[below] {$y_{2}$};
        \draw (x1) -- (y1) -- (x2) -- (y2) -- cycle;
        \draw (x1) -- (x2);
        \draw (y1) -- (y2);
        \draw (0,0) node[below right] {O};
    \end{tikzpicture}
\end{center}

\begin{proof}
    By symmetry, $\angle x_{1}O y_{1} = \angle x_{1} O y_{2}$,
    but they add to $\pi$, so they must each be $\pi/2$.

    Note that in the degenerate case of $y_{1} = y_{2}$,
    $y_{2} - y_{1} = 0$, and $0$ is perpendicular to every complex number.
\end{proof}

\begin{lemma}[3-Point Lemma]
    Let $X \subseteq \mathbb{C}$ and $f \in \Isom(X)$.
    If there are $x_{1},x_{2},x_{3} \in X$
    not colinear, s.t. $f(x_i) = x_i$
    for $i = 1,2,3$, then $f = \id X$.
\end{lemma}
\begin{proof}
    The proof is by contradiction.
    Suppose that $f(y) \ne y$ for some $y \in X$.
    Then,
    \begin{align*}
        |f(y) - x_i| = |f(y) - f(x_i)| = |y - x_i|.
    \end{align*}
    for $i = 1,2,3$.
    Applying the kite lemma with $y_{1}=y$, $y_{2}=f(y)$ yields that
    $x_{2} - x_{1}$ is perpendicular to $f(y) - y$.
    Applying the kite lemma again for $x_{2},x_{3}$ yields that
    $x_{3} - x_{2}$ is also perpendicular to $f(y) - y$.
    Since $f(y) - y$ is non-zero, $x_{3} - x_{2}$ must be parallel to $x_{2} - x_{1}$.
    This contradicts the assumption that $x_{1},x_{2},x_{3}$ are not colinear. \contra
\end{proof}

This lemma can be generalized to the $n+1$-point lemma,
applicable in $\mathbb{R}^{n}$.

\appendix
\newpage

\section{Notation}
Let $X$ a set, then we denote the identity map on $X$,
$\id X : X \to X$, defined by $\id X(x) = x$.

Given functions $f:A\to B$ and $g:B\to C$,
we defined $g\circ f:A\to C$ by $g\circ f(a) = g(f(a))$.

Since it is somewhat cumbersome to write $(G,\cdot,e)$
for a group, we will abuse notation and simply write $G$ for the group.

We will then write $e_{G}$ and $\cdot_{G}$ for the operations of the group
if it is important that they correspond to the group $G$.

\section{Proof of \autoref{thm:right_is_left}}
\label{sec:proof_of_right_is_left}
\begin{proof}
    (i)
    Assume $a,b \in G$ s.t. $a \cdot b = e$.
    \begin{align*}
        b &= b\cdot e &\text{by the identity axiom} \\
          &= b \cdot (a \cdot b) &\text{by assumption}\\
          &= (b \cdot a) \cdot b &\text{by associativity}
    \end{align*}
    By the inverse axiom, we have $c$ such that $b \cdot c = e$.
    Right multiplying the above equation by $c$, we have
    \begin{align*}
        b \cdot c &= ((b \cdot a) \cdot b) \cdot c\\
        b \cdot c &= (b \cdot a) \cdot (b \cdot c) & \text{by the associativity axiom}\\
        e &= (b \cdot a) \cdot e & \text{by assumption (inverse axiom)}\\
          &= b \cdot a & \text{by the identity axiom}.
    \end{align*}
    Thus, $e = b\cdot a$, and we have (i).

    (ii)
    By the inverse axiom and (i) we have $b \in G$ s.t. $a \cdot b = e = b \cdot a$.
    Now, we have
    \begin{align*}
        e \cdot a &= (a \cdot b) \cdot a &\text{as above}\\
                  &= a \cdot (b \cdot a) &\text{by the associativity axiom}\\
                  &= a \cdot e &\text{as above}\\
                  &= a &\text{by the identity axiom}.
    \end{align*}
    Thus, $e \cdot a = a$, and we have (ii).

    (iii)
    Suppose $b,b' \in G$ s.t. $a\cdot b = e = a \cdot b'$. Then we have
    \begin{align*}
        b' &= e \cdot b' &\text{by (ii)}\\
           &= (b \cdot a) \cdot b' &\text{by above and (i)}\\
           &= b \cdot (a \cdot b') &\text{by the associativity axiom}\\
           &= b \cdot e &\text{as above}\\
           &= b &\text{by the identity axiom}.
    \end{align*}
    Thus, $b' = b$, and we have (iii).

    (iv)
    Suppose $e' \in G$ s.t. $a \cdot e' = a$.
    By the inverse axiom and (i) we have $b \in G$ s.t. $b \cdot a = e$.
    Left multiplying by $b$ gives 
    \begin{align*}
        b \cdot (a \cdot e') &= b \cdot a\\
        (b \cdot a) \cdot e'   &= b \cdot a &\text{by the associativity axiom}\\
        e \cdot e' &= e &\text{as above}\\
        e' &= e &\text{by (ii)}
    \end{align*}
    Thus, we have (iv).
\end{proof}

\end{document}
