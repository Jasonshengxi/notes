\documentclass[12pt]{article}
\usepackage[left = 1in, right = 1in, top = 1in, bottom = 1in]{geometry}
\usepackage{paralist}
\usepackage{cancel}
\usepackage{enumitem}
\usepackage{amsmath}
\usepackage{amssymb}
\usepackage{amsthm}
\usepackage{tkz-euclide}
\usepackage{hyperref}
\usepackage{esdiff}
\usepackage{parskip}
\usepackage{accents}
\usepackage{xcolor}

\usetikzlibrary{arrows.meta,positioning}

\newtheoremstyle{customstyle}
  {8pt} % Space above (adjust as needed)
  {0} % Space below (adjust as needed)
  {} % Body font
  {} % Indent amount
  {\bfseries} % Theorem head font
  {. } % Punctuation after theorem head
  {0pt} % Space after theorem head
  {} % Theorem head spec
\theoremstyle{customstyle}
\newtheorem{theorem}{Theorem}[section]
\newtheorem{corollary}{Corollary}[theorem]
\newtheorem{lemma}{Lemma}[section]
\newtheorem{definition}{Definition}[section]

\def\definitionautorefname{Definition}
\def\corollaryautorefname{Corollary}

\newenvironment{nonproof}{\par \textit{Nonproof:}}{\hfill$\cancel\square$}

\def\contra{\tikz[baseline, x=0.22em, y=0.22em, line width=0.032em]\draw (0,2.83)--(2.83,0) (0.71,3.54)--(3.54,0.71) (0,0.71)--(2.83,3.54) (0.71,0)--(3.54,2.83);}

\renewcommand{\Re}{\operatorname{Re}}
\renewcommand{\Im}{\operatorname{Im}}
\renewcommand{\bar}{\overline}


\begin{document}
% Lectured by Christopher Thomas
% Email: c.e.thomas@damtp.cam.ac.uk

\setcounter{section}{-1}
\section{Introduction}

DEs are cool.

For example, Newton's
\[
    m \diff[2]xt = F(x,t)
\]
Where $m$ is mass, $F$ is force, and we call
$x$ the \emph{dependent variable} and $t$ the \emph{independent variable}.

\section{Basic Calculus}

\subsection{Differentiation}

\begin{definition}\label{def:derivative}

    The derivative of a fn $f(x)$ wrt its argument $x$ is
    the function
    \[
        \diff f x = \lim_{h\to 0}\frac{f(x+h) - f(x)}{h}.
    \]
    For the derivative to exist, we require the left-handed
    and right-handed limits to both exist and be equal.
\end{definition}

For example, $|x|$ is not differentiable at $0$ since
the right limit is $1$ and the left limit is $-1$.
(it is however differentiable everywhere else)

\subsubsection*{Order parameters}

Consider the behavior of a function close to a limiting point, $x_{0}$.

\subsubsection*{Big O}

\emph{Big O} roughly means \emph{can be bounded by}.

\begin{definition}
    For finite $x_{0}$, we say
    $f(x)$ is $O(g(x))$ as $x \to x_{0}$ if
    there exists $\delta >0$ and $M > 0$ s.t.
    for all $x$ with $0 < |x - x_{0}| < \delta $,
    we have $|f(x)| \le M|g(x)|$.
\end{definition}

It follows that $\dfrac{f(x)}{g(x)}$ is bounded as $x \to x_{0}$.

Note that this definition is for $x_{0}$ finite,
whereas most usages of $O$, for instance for time complexity,
uses it as $x_{0} = \infty$.

Examples:
\begin{compactitem}
    \item $x \ne O(x^{2})$ as $x \to 0$.
    \item $x^{2} = O(x)$ as $x \to 0$.
    \item $x = O(\sqrt{x})$ as $x \to 0$.
    \item $\sin(2x) = O(x)$ as $x \to 0$, since $|\sin 2x| \le 2|x|$ for all $x$.
\end{compactitem}

\begin{definition}
    For infinite $x_{0}$, we say
    $f(x)$ is $O(g(x))$ as $x \to \infty$
    if there exists $x_{1}$ and $M > 0$ s.t.
    for all $x > x_{1}$, we have
    $|f(x)| \le M|g(x)|$.
\end{definition}

For example, $2x^{3}+4x+12 = O(x^{3})$ as $x \to \infty$.

\subsubsection*{Little o}
\emph{Little o} roughly means \emph{much smaller than}.
It is written $\underline{o}$ in handwritten math, 
to distinguish from $O$.

\begin{definition}
    For finite $x_{0}$, we say
    $f(x)$ is $o(g(x))$ as $x \to x_{0}$ if,
    for any $\varepsilon > 0$, there exists $\delta >0$ s.t.
    for all $0 < |x - x_{0}| < \delta $,
    we have $|f(x)| \le \varepsilon |g(x)|$.
\end{definition}

Note that if $g(x) \ne 0$ in the vicinity of $x_{0}$,
but not necessarily at $x_{0}$, the above definition is equivalent to
\[
\lim\limits_{x\to x_{0}} \frac{f(x)}{g(x)} = 0.
\]
For example, $x^{2} = o(x)$ as $x \to 0$, since $\lim\limits_{x \to 0} \frac{x^{2}}{x} = 0$.

The infinite definition is similar.

\subsubsection*{Notes}

\textbf{$f(x) = o(g(x))$ is a stronger statement than $f(x) = O(g(x))$}.
Intuitively, big $O$ means bounded by some given multiple,
whereas little $o$ means bounded by any multiple.

Thus, $f(x) = o(g(x))$ implies $f(x) = O(g(x))$,
but the converse does not hold.

\textbf{Constants \emph{don't} matter}. 
If $f(x) = O(g(x))$, then $af(x) = O(g(x))$ and $f(x) = O(ag(x))$
for any $a \ne 0$.

\subsubsection*{Usage}

Order parameters are useful to classify remainder terms before taking limits.
Let $\varepsilon (h)$ in
\[
f(x_{0}+h) - f(x_{0}) = hf'(x_0) + \varepsilon (h),
\]
for some finite $h$. Then we have,
by dividing by $h$ on both sides and taking the limit,
\[
\lim\limits_{h \to 0}\left[\frac{f(x_{0}+h)-f(x_{0})}{h}\right]
= f'(x_{0}) + \lim\limits_{h \to 0}\left[\frac{\varepsilon (h)}{h}\right]
\]
The limit on the RHS vanishes by the definition of $f'$.
Therefore, $\varepsilon (h) = o(h)$ as $h \to 0$.

\newpage
\subsection{Rules for Differentiation}

\begin{theorem}[Chain Rule]
    Given $f(x) = F(g(x))$, we have
    \[
    \diff f x = F'(g(x)) \diff gx = \diff Fg \diff gx,
    \]
    where $F'$ denotes the derivative of $F$ wrt to its argument,
    and $F'(g(x))$ denotes the evaluation of that derivative at $g(x)$.
\end{theorem}

\begin{theorem}[Product Rule]
    Given $f(x) = u(x)v(x)$, we have
    \[
    \diff f x = v \diff ux + u \diff vx.
    \]
\end{theorem}

\begin{theorem}[Leibniz's Rule]
    Given $f(x) = u(x)v(x)$, we have
    \[
    f^{(n)} = \sum\limits_{r=0}^{n}\binom nr \left(u^{(r)} \cdot v^{(n-r)}\right).
    \]
\end{theorem}

\subsection{Taylor Series}

For $f(x)$ infinitely differentiable at a point $x = x_{0}$,
the \emph{Taylor series of $f$ about $x_{0}$} is 
\[
T_{f}(x) = \sum\limits_{r=0}^{\infty} \frac{f^{(r)}(x_{0})}{r!}(x - x_{0})^{r}.
\]
We define the \emph{Taylor polynomial of degree n}, $P_{n}(x)$, to be
the truncation of the Taylor series up to the $n$th term, i.e.
\[
P_{n}(x) = \sum\limits_{r=0}^{n} \frac{f^{(r)}(x_{0})}{r!}(x - x_{0})^{r}.
\]
Note that $P_n(x)$ matches the first $n$ derivatives of $f(x)$
at $x_{0}$.

\subsubsection*{How good of an approximation is it?}

As $h \to 0$, we have
\[
    f(x_{0}+h) = f(x_{0}) + hf'(x_{0}) + o(h).
\]
This extends to Taylor's series, in a result called the \emph{Taylor theorem}.

\begin{theorem}[Taylor's Theorem]
    For $f(x)$ $n$-times differentiable at $x_{0}$, we have
    \[
    f(x_{0}+h) = f(x_{0}) + hf'(x_{0}) + \frac{1}{2!}h^{2}f''(x_{0})
    + \cdots + \frac{h^{n}}{n!}f^{(n)}(x_{0}) + E_n,
    \]
    where $E_n$ denotes the error. We have that
    \[
    E_n = o(h^{n})
    \]
    as $h \to 0$.
\end{theorem}

We cna attain a stronger result if $f$ is $(n+1)$-times differentiable
in some interval $(x_{0},x_{0}+h)$ and $f^{(n+1)}$ is continuous in this range, we have
\[
E_n = O(h^{n+1})
\]
as $h \to 0$. so
\[
E_n = f^{(n+1)}(x_{n}) \frac{h^{n+1}}{(n+1)!}
\]
for some $x_{n}$ with $x_{0} \le x_n \le x_{0} + h$. (See Analysis I)

Note that $E_n = O(h^{n+1})$ is a stronger statement than $o(h^{n})$.
For example, $h^{n+\frac{1}{2}}$ is $o(h^{n})$ but \emph{not} $O(h^{n+1})$ as $h \to 0$.

With $x = x_{0} + h$, Taylor's theorem gives
\[
f(x) = P_n(x) + E_n,
\]
so $P_n(x)$ provides a local approximation to $f(x)$ 
in the vicinity of $x_{0}$ with error $o(h^{n})$ or $O(h^{n+1})$.
If $\lim\limits_{n\to\infty}E_{n} = 0$, then the
Taylor series converges to $f(x)$.

For example, take $f(x) = \exp(x)$ about $x_{0} = 0$,
then $f(x)$ is infinitely differentiable and continuous everywhere, so
\[
E_n = \frac{h^{n+1}}{(n+1)!}\exp(x_n),
\]
for some $0 \le x_n \le h$. (note that $h > 0$.)
Consider the fractional error, we have
\[
    \dfrac{E_n}{\exp(h)}= \frac{h^{n+1}}{(n+1)!}\exp(x_n - h).
\]
We have $-h \le x_n - h \le 0$, so $0 < \exp(x_n - h) \le 1$, so
\[
\frac{E_n}{\exp(h)} \le \frac{h^{n+1}}{(n+1)!}
\]
For a given target accuracy at $x=h$, 
this specifies how large $n$ must be.

\subsection{L'h\^{o}pital's Rule}

\begin{theorem}[L'h\^{o}pital's Rule]
    Let $f(x)$ and $g(x)$ be differentiable at $x_{0}$ with
    continuous first derivatives, and
    \begin{align*}
    \lim\limits_{x\to x_{0}}f(x) = f(x_{0}) = 0,\\
    \lim\limits_{x\to x_{0}}g(x) = g(x_{0}) = 0,\\
    \end{align*}
    Then if $g'(x_{0}) \ne 0$, 
    \[
    \lim\limits_{x\to x_{0}} \frac{f(x)}{g(x)} = 
    \lim\limits_{x\to x_{0}} \frac{f'(x)}{g'(x)},
    \]
    Provided the limit on the RHS exists.
\end{theorem}

\begin{proof}
    Using Taylor's theorem, we have
    \[
    f(x) = f(x_{0}) + (x - x_{0})f'(x_{0}) + o(x - x_{0}),
    \]
    as $x \to x_{0}$ and similarly for $g$. Since $f(x_{0}) = g(x_{0}) = 0$, we have
    \begin{align*}
        \lim\limits_{x\to x_{0}}\frac{f(x)}{g(x)} 
        &= \lim\limits_{x \to x_{0}}
        \left[\frac{f'(x_{0}) + o(x-x_{0})/(x - x_{0})}{g'(x_{0}) + o(x-x_{0})/(x - x_{0})}\right]\\
        &= \frac{\lim\limits_{x\to x_{0}} \cdots}{\lim\limits_{x\to x_{0}} \cdots}\\
        &= \frac{f'(x_{0})}{g'(x_{0})}
        = \lim\limits_{x\to x_{0}} \frac{f'(x)}{g(x)},
    \end{align*}
    by the properties of limits and the continuity of first derivatives.
\end{proof}

L'h\^{o}pital's Rule can be generalized,
for instance if $f'(x) = g'(x) = 0$ and 
$f,g$ have continuous second derivatives,
we have
\[
\lim\limits_{x\to x_{0}}\frac{f(x)}{g(x)} = \lim\limits_{x\to x_{0}}\frac{f''(x)}{g''(x)},
\]
and similarly for higher degrees.


\section{Integration}

\subsection{Integrals as Riemann Sums}

\begin{definition}
    The \emph{integral} of a (suitably well defined)
    function $f(x)$ is the limit of a sum. For example,
    \[
        \int_{a}^{b}f(x)dx = \lim_{N\to\infty}\sum_{n=0}^{N-1} f(x_n)\Delta x
    \]
    where $\Delta x = (b-a)/N$, $x_n = a + n\Delta x$.
\end{definition}

\begin{center}
    \begin{tikzpicture}
    \draw[->] (0,0) -- (5, 0) node[right] {$x$};
    \draw[->] (0,0) -- (0, 5) node[above] {$y$};
    \draw[color=red, domain=1:4] plot (\x, \x*\x/4 + 0.5)
        node[above right] {$f(x)$};
    \foreach \x in {1,1.25,...,1.75,3,3.25,...,3.75}
    \draw (\x, 0) rectangle ++(0.25, {\x*\x/4 + 0.5});
    \draw (2.5, 0.75) node {$\cdots$};
\end{tikzpicture}
\end{center}

$f(x)$ is \emph{Riemann integrable} if generalised sum
doesn't dpened on exactly how choose the rectangles in the limit
that all $\Delta x \to 0$, for instance if $\Delta x$ was non-uniform.

Consider one rectangle,
\begin{center}
    \begin{tikzpicture}
    \draw[->] (0,0) -- (5, 0) node[right] {$x$};
    \draw[->] (0,0) -- (0, 5) node[above] {$y$};
    \draw[color=red, domain=1:4] plot (\x, \x*\x/6 + 2)
        node[above right] {$f(x)$};
    \def\x{1.5}
    \draw (\x, 0) node[below] {$x_n$} rectangle ++(2, {\x*\x/6 + 2});
    \draw ({\x + 2}, 0)
        node[below] {$x_{n+1}$};
    \draw[dashed] ({\x + 2}, {\x*\x/6 + 2})
        -- ({\x + 2}, {(\x+2)*(\x+2)/6+2});
\end{tikzpicture}
\end{center}


The Mean value theorem states that
for $f(x)$ continuous, the area under the curve from $x_n$ to $x_{n+1}$ is
\[
A_n = (x_{n+1}-x_n)f(c_n)
\]
for some $x_n \le c_n \le x_{n+1}$.
If $f(x)$ is differentiable,
\begin{align*}
    f(c_n) &= f(x_n) + o(c_n - x_n)\\
           &= f(x_n) + o(\Delta x)
\end{align*}
substituting we get
\[
A_n = \Delta x f(x_n) + o((\Delta x)^{2})
\]
So the total area from $a$ to $b$ is
\begin{align*}
    A &= \lim_{N \to \infty} \sum_{n=0}^{N-1}A_n\\
      &= \lim_{N\to\infty} \sum_{n=0}^{N-1}f(x_n)\Delta x 
      + \lim_{N\to\infty} N O \left[\left(\frac{b-a}{N}\right)^{2}\right]\\
      &= \int_a^b f(x) dx + \lim_{N\to\infty}O(1/N)\\
      &= \int_a^b f(x) dx
\end{align*}
Thus, the integral is the area.

\subsection{Fundamental theorem of Calculus (FTC)}

\begin{theorem}[FTC]
    Let $F(x) = \int_0^x f(t)dt$. then,
    \[
        \diff Fx = \diff{}x \left[\int_a^x f(t)dt\right] = f(x)
    \]
\end{theorem}
\begin{proof}
    \begin{align*}
        \diff Fx &= \lim_{h\to 0}\frac{1}{h}\left[\int_a^{x+h}f(t)dt
        - \int_a^x f(t) dt\right]\\
                 &= \lim_{h\to 0}\frac{1}{h}
                 \int_x^{x+h}f(t)dt\\
                 &= \lim_{h\to 0}\frac{1}{h}\biggl[
                     f(x)h + O(h_{2})\biggr]\\
                 &= \lim_{h\to 0} f(x) + O(h)\\
                 &= f(x)
    \end{align*}
\end{proof}

Note that this $F(x)$ is a solution to $\diff Fx = f(x)$,
specifically the solution with $F(a)=0$.

Some corollaries:
\begin{align*}
    \diff{}x \int_x^b f(t)dt &= -f(x)\\
    \diff{}x \int_x^{g(x)} f(t)dt &= \diff{}x F(g(x)) = \diff Fg \diff gx\\
                                  &= f(g(x)) \diff gx
\end{align*}

\subsection{Integration Techniques}

\subsubsection*{Substitution}

Substitute.

\subsubsection*{Trigonometric Substitutions}

Substitute.

\begin{table}[h]
    \centering
    \begin{tabular}{c|c}
        Integrand & Substitute \\
        \hline
        $\sqrt{1 - x^{2}}$ & $\sin\theta $ \\
        $\sqrt{x^{2} + 1}$ & $\sinh\theta $ \\
        $\sqrt{x^{2} - 1}$ & $\cosh\theta $ \\
        $1 + x^{2}$ & $\tan\theta $ \\
        $1 - x^{2}$ & $\tanh\theta $ \\
    \end{tabular}
\end{table}

\subsubsection*{Integration by parts}

Recalling the product rule, we have $(uv)'=u'v + uv'$, 
rearranging and adding integral signs yields
\[
\int uv'dx = uv - \int u'v dx.
\]
\section{Partial Differentiation}
\subsection{Functions of Several Variables}

Differentiating \emph{multivariate functions}.

% Contour plot:
% \begin{center}
%     \begin{tikzpicture}
%         \draw[->] (-3, -3) -- (3, -3) node [right] {$x$};
%         \draw[->] (-3, -3) -- (-3, 3) node [right] {$y$};
%         \foreach \x in {0.5, 1.0, 1.4, 1.7}
%         \draw (0, 0) ellipse ({1.5*\x cm} and \x cm);
%     \end{tikzpicture}
% \end{center}

\subsection{Partial Derivatives}

\begin{definition}
    Given a function of several variables, say $f(x,y)$,
    the \emph{partial derivative} of $f$ wrt $x$ at fixed $y$ is
    \[
        \left.\diffp{f}{x}\right|_{y} 
            = \lim_{\delta x \to 0} \frac{f(x+\delta x,y) - f(x,y)}{\delta x}.
    \]
\end{definition}
Note that this can be interpreted as
the slope of $f$ when moving in the $+x$ direction.

We also shorten many things, like
\[
    f_x \equiv \diffp f x
\]
or
\[
    f_{xy} \equiv \diffp f{yx}
\]
Note that if $f$ has continuous second derivatives then
\[
    \diffp f{xy} = \diffp f{yx}
\]
This is known as Schwarz's theorem.

\subsection{Multivariate Chain Rule}

Given path $x(t),y(t)$ and $f(x,y)$,
what is $\diff ft$ along the path?

Consider a small change in $f$, under $(x,y) \mapsto (x + \delta x, y+\delta y)$.
\begin{align*}
    \delta f &= f(x + \delta x, y + \delta y) - f(x,y)\\
             &= f(x + \delta x, y + \delta y) - f(x + \delta x,y)\\
             &+ f(x + \delta x, y)  - f(x,y)
\end{align*}
By Taylor's theorem, $f(x+\delta x,y) - f(x,y) = f_x(x,y)\delta x + o(\delta x)$.\\
For the other half, we have, 
\begin{align*}
    f(x+\delta x,y + \delta y) - f(x,y + \delta y) 
    &= f_y(x + \delta x,y)\delta y + o(\delta y)\\
    &= \biggl(f_y(x,y) + f_{yx}(x, y)\delta x + o(\delta x)\biggr)\delta y + o(\delta y)
\end{align*}
Substituting, we have
\[
    \delta f = \biggl(f_y(x,y) + f_{yx}(x,y)\delta x + o(\delta x)\biggr)\delta y
    + f_x(x,y)\delta x + o(\delta y) + o(\delta x)
\]
Taking the limit when $\delta x \to 0$ and $\delta y \to 0$, we have
\[
    \mathrm{d} f = \diffp f x \mathrm{d} x + \diffp f y \mathrm{d} y
\]
For the final answer, we have
\begin{align*}
    \diff{}t f(x(t),y(t)) 
    &= \lim_{\delta x,\delta y,\delta t\to 0} 
    \left(\diffp f x \frac{\delta x}{\delta t} + \diffp f y \frac{\delta y}{\delta t}\right)\\
    &= \diffp f x \diff x t + \diffp f y \diff y t
\end{align*}
If we ask for $\diff f x$ instead, we have
\begin{align*}
    \diff fx &= \diffp f x \diff x x + \diff fy \diff yx\\
             &= \diffp fx + \diff fy \diff yx
\end{align*}

\subsubsection*{Integral form of chain rule}

For the change in $f$ between two end points, denoted $\Delta f$, we have
\[
    \Delta f = \int \di f = \int \diffp fx \di x + \int \diffp fy \di y
\]
which is trivial from substituting $\diff ft$.

\subsection{Applications of the Multivariate Chain Rule}


\appendix

\newpage
\section{Notation}

\subsection{Limits}

For left and right limits, we may write $\lim\limits_{h \to 0^{-}}$ and $\lim\limits_{h \to 0^{+}}.$ 

\subsection{Derivatives}

For derivatives, we may write
\[
    \diff f x = f'(x) = \dot f(x).
\]
Similarly, we have, for a second derivative,
\[
    \diff {}x \diff f x = \diff[2] f x = f''(x) = \ddot f(x).
\]
And for an arbitrary $n$th derivative, we may write
\[
    \diff[n] f x = f^{(n)}(x).
\]

\subsection{Order parameters}

Some may write $f(x) = O(g(x))$ when $f(x)$ is $O(g(x))$.
Some say it is better to write $f(x) \sim O(g(x))$.
It is up to the reader to decide which one is better.

\subsection{Integration}

Indefinite integrals have two notations: $\int f(x)dx$, or $\int^x f(t)dt$.
It has no specified lower limit so the an integration constant appears.

\end{document}
