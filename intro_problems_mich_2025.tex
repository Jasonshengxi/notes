\documentclass{article}
\usepackage[left = 1in, right = 1in]{geometry}
\usepackage{paralist}
\usepackage{amsmath}
\usepackage{amssymb}
\usepackage{amsthm}
\usepackage{tikz}
\usepackage{hyperref}

\newtheorem{question}{Question}

\newenvironment{answer}{\par\noindent\textit{Answer.}}{\par}

\begin{document}

\begin{question}
    Is it possible for three consecutive odd positive integers to be prime?
    If so, how many such prime triplets are there?
\end{question}

\begin{answer}
    Yes, only one. 
    Let $p,p+2,p+4$ be such a triplet.
    We have $p,p+2,p+4 \equiv p,p+2,p+1 \mod 3$,
    which enumerates every possible residual mod $3$.
    Thus, one of them must be divible by $3$.
    If it is not $3$, it must be composite.
    Thus, the only possible triplets are $1,3,5$ and $3,5,7$.
    It is trivially easy to see that $3,5,7$ are the only such triplet.
\end{answer}

\begin{question}
There are four prime numbers between $0$ and $10$, and also between $10$ and $20$.
Does it ever happen again that there are four prime numbers between two consecutive multiples of $10$?
\end{question}

\begin{answer}
    Yes, $101,103,107,109$.
\end{answer}

\begin{question}
Consider the sequence $41,43,47,53,61,\cdots$ 
Is every number in this sequence prime?
\end{question}

\begin{answer}
    Let the sequence $x_{0},x_{1},x_{2},\cdots$ represent the former sequence.
    It an be written $x_n = n^{2} + n + 41$.
    It is clear that $x_{41} = 41^{2} + 41 + 41$ is divisible by $41$ and is thus composite.
\end{answer}

\begin{question}
Show that a positive integer is a multiple of $9$ iff
the sum of its digits is divisible by $9$.
Find a similar test for divisibility by $11$.
\end{question}

\begin{answer}
    Every positive integer $n$ can be written in its base $10$ expansion as
    \[
        \begin{aligned}
            n &= \sum\limits_{i=0}^{k} 10^{i}x_k \\
              &\equiv \sum_{i=0}^{k} 1^{i}x_k \mod 9 \\
              &= \sum_{i=0}^{k} x_k \\
        \end{aligned}
    \]
    Thus, every positive integer is equal to the sum of its digits mod $9$.
    Thus, every positive integer is divisible by $9$ iff its sum of digits is.
\end{answer}

\begin{question}
What is the sum of the digits of the sum of the digits of the sum of the digits of $4444^{4444}$
\end{question}

\begin{answer}
    We write $d(n)$ for the number of digits of a number $n$,
    and $D(n)$ for the sum of the digits of $n$.

    Two trivial lemma are that $a \le b \implies d(a) \le d(b)$,
    and $D(n) \le 9d(n)$.

    We let $N = 4444^{4444}$. Since $N < (10000)^{4444}$, $d(N) \le d(10000^{4444}) = 17776$.
    Thus, $D(N) \le 9 \cdot 17776 = 159984$.

    Similarly, $D(D(N)) < 9d(D(N)) = 9 \cdot 6 = 54$.

    Finally, the largest value of $D(D(D(N)))$ is $13$, corresponding to $D(D(N)) = 49$.

    I can't be bothered to finish this for now.
\end{answer}

\begin{question}
Does there exist a subset $A$ of $\mathbb{N}$ such that
for all $d \in \mathbb{N}$, there exists unique $a,b \in A$ such that $a - b = d$?
\end{question}

\begin{answer}
\end{answer}

Given a subset $A$ of $\mathbb{N}$, notice that its elements divide
the real line into infinite disjoint intervals.
Let the lengths of those intervals be $x_{0},x_{1},\cdots$, ordered by
their position starting from $-\infty$. 
Note that $x_{0}$ must always be infinite, since all $x \in A$ have $x \ge 0$.

Let $A$ satisfy the condition in the question.
Let $d \in \mathbb{N}$. Then, there exist $a - b = d > 0$.
Thus, the length of the segment between $a$ and $b$ is equal to $a - b$.
However, the length of the segment between $a$ and $b$ is also
equal to the sum of the lengths of all the intervals between them.
Thus, there exists $n,m \in \mathbb{N}$ such that $n<m$ and
\[\sum_{i=n}^{m}x_i = a - b = d.\]

Similarly, given a sequence $x_1,x_2,\cdots$ such that
for any $d \in \mathbb{N}$, there exist $n,m \in \mathbb{N}$ with $n < m$ such that
$\sum_{i=n}^{m}x_i  = d.$
can be used to generate a subset $A$ of $\mathbb{N}$, defined by
$A = \{\sum_{i=1}^{n}x_i \mid n \in \mathbb{N} \cup \{0\}\}$.

Now, we define $A_k = \{\sum_{i=0}^{k-1}x_{i+n} \mid n \in \mathbb{N}\}$.
Any $A_i,A_j$ where $i \ne j$ are disjoint.
If an element $a \in A_i \cap A_j$, then it can be written both
as a sum of $i$ values and as a sum of $j$ values, which violates the assumption.
It is also obvious that there is a bijection between $\mathbb{N}$ and $A_{k}$.
By the transitive nature of bijections, there exists a bijection between $A_{i}$ and $A_{j}$.

\end{document}
