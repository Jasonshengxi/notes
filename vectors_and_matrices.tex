\documentclass[12pt]{article}
\usepackage[left = 1in, right = 1in, top = 1in, bottom = 1in]{geometry}
\usepackage{paralist}
\usepackage{cancel}
\usepackage{enumitem}
\usepackage{amsmath}
\usepackage{amssymb}
\usepackage{amsthm}
\usepackage{tkz-euclide}
\usepackage{hyperref}
\usepackage{esdiff}
\usepackage{parskip}
\usepackage{accents}
\usepackage{xcolor}

\usetikzlibrary{arrows.meta,positioning}

\newtheoremstyle{customstyle}
  {8pt} % Space above (adjust as needed)
  {0} % Space below (adjust as needed)
  {} % Body font
  {} % Indent amount
  {\bfseries} % Theorem head font
  {. } % Punctuation after theorem head
  {0pt} % Space after theorem head
  {} % Theorem head spec
\theoremstyle{customstyle}
\newtheorem{theorem}{Theorem}[section]
\newtheorem{corollary}{Corollary}[theorem]
\newtheorem{lemma}{Lemma}[section]
\newtheorem{definition}{Definition}[section]

\def\definitionautorefname{Definition}
\def\corollaryautorefname{Corollary}

\newenvironment{nonproof}{\par \textit{Nonproof:}}{\hfill$\cancel\square$}

\def\contra{\tikz[baseline, x=0.22em, y=0.22em, line width=0.032em]\draw (0,2.83)--(2.83,0) (0.71,3.54)--(3.54,0.71) (0,0.71)--(2.83,3.54) (0.71,0)--(3.54,2.83);}

\renewcommand{\Re}{\operatorname{Re}}
\renewcommand{\Im}{\operatorname{Im}}
\renewcommand{\bar}{\overline}


\begin{document}
% Lectured by Angela Capel Gueras (ac2722)
% Office: BO.14, Department of Applied Mathematics and Theoretical Physics

\section{Complex Numbers}

\subsection{Definition}

We construct $\mathbb{C}$ by adding $i$ to $\mathbb{R}$, where $i^{2} = -1$.
Then, any $z \in \mathbb{C}$ has the form $z = x + iy$, where $x,y \in \mathbb{R}$.
We define the real part, $x = \Re(z)$, and the imaginary part, $y = \Im(z)$.

\subsection{Properties}

\begin{compactenum}[i)]
\item Addition: $z_{1} \pm z_{2} = (x_{1} \pm x_{2}) + i(y_{1} \pm y_{2})$.
\item Multiplication: $z_{1}z_{2} = (x_{1}y_{1} - x_{2}y_{2}) + i(x_{1}y_{2} + x_{2}y_{1})$.
\end{compactenum}
Note that both addition and multiplication are associative and commutative.
\begin{compactenum}[i)]
\setcounter{enumi}{2}
\item Identity: The group $(\mathbb{C},+)$ is an abelian group with identity $0$.
\item Inverse: The \emph{inverse of $z$} is given by
    \[
    z^{-1} = \frac{x - iy}{x^{2} + y^{2}}
    \]
    and it satisfies $z \cdot z^{-1} = 1$.
\end{compactenum}

Thus, $(\mathbb{C}^{*}, \cdot)$ is an abelian group with identity $1$,
where $\mathbb{C}^{*} = \mathbb{C} - \{0\}$.

Note that the distributive property is also satisifed, where
\[
    (z_{1}+z_{2})z_{3} = z_{1}z_{2} + z_{2}z_{3}
\]
\begin{compactenum}[i)]
\setcounter{enumi}{4}
\item Complex Conjugate: For any $z = x + iy$, 
    the \emph{complex conjugate of $z$}, denoted $\bar z$, or $z^{*}$, is equal to $x - iy$.
\end{compactenum}
Thus, we also have $\Re(z) = (z + \bar z) / 2$, and $\Im(z) = (z - \bar z) / 2i$.
We have rules
\begin{compactitem}
\item $\bar {\bar z} = z$
\item $\bar{z_{1} + z_{2}} = \bar z_{1} + \bar z_{2}$
\item $\bar {z_{1}z_{2}} = \bar z_{1} \bar z_{2}$
\end{compactitem}

\begin{compactenum}[i)]
\setcounter{enumi}{5}
\item Modulus: For any $z = x + iy$, we define the \emph{modulus of $z$} by $|z|$, or $r$, by
    a real and non-negative number such that
    \[
    |z|^{2} = x^{2} + y^{2}
    \]
\item Argument: The \emph{argument} of a complex number $z = x + iy \ne 0$
    is a real number, denoted by $\theta = \arg(z)$, such that
    \[
        z = r(\cos \theta + i \sin \theta)
    \]
    Which is the \emph{polar form} of $z$. We can verify
    that $\tan\theta = y/x$.
\end{compactenum}

Note that if $\theta$ is an argument of $z$, so is $\theta + 2\pi n$ for $n \in \mathbb{Z}$.
To make it unique, we restrict $-\pi < \theta \le \pi$.
This value of $\theta$ is the \emph{principal value}.

Remarks:
\begin{compactenum}[(1)]
\item $\mathbb{R} \subset \mathbb{C}$, since for $a \in \mathbb{R}$ we have
    $a = a + i 0 \in \mathbb{C}$.
\item A complex number $0 + ib$ is said to be a \emph{pure imaginary number}.
\item The representation of a complex number in terms
    of its real and imaginary parts is unique.
\end{compactenum}

\subsubsection*{More Properties / Consequences}
\begin{compactenum}[(i)]
\setlength{\parskip}{4pt}
\item $(\mathbb{C},+,\cdot)$ is a field.
\item \emph{Fundamental Theorem of Algebra}

    A polynomial of degree $n$ with coefficients in $\mathbb{C}$
    can be written as the product of $n$ linear factors:

    \[
        \begin{aligned}
            p(z) &= c_n z^{n} + \cdots + c_{0} \\
                &= c_n (z - \alpha _1)\cdots(z - \alpha_2)
        \end{aligned}
    \]
    
    where $c_i \in \mathbb{C}$, and $c_n \ne 0$, with roots $\alpha_i \in \mathbb{C}$.
    Thus, $p(z) = 0$ has at least one root,
    and $n$ roots counted with multiplicity.
\item The modulus satisifes the following properties:
    \begin{itemize}
        \item $|z_{1}z_{2}| = |z_{1}||z_{2}|$
        \item $|z_{1}| + |z_{2}| \le |z_{1}| + |z_{2}|$
        \item $|z_{1} - z_{2}| \ge \left||z_{1}| - |z_{2}|\right|$
    \end{itemize}
\end{compactenum}

\subsection{Argand Diagram}

\begin{tikzpicture}
    \tkzInit[xmin=-1,ymin=-1,xmax=5,ymax=4]
    \tkzDrawX\tkzDrawY
    \tkzDefPoint(0,0){O}
    \tkzDefPoint(4,3){A}

    \tkzDrawSegment[arrows={-Latex}](O,A)
\end{tikzpicture}

\subsection{Exponential and Trigonometric Functions}

\subsection{Logarithm and Complex Powers}

\subsection{Lines and Circles}

\end{document}
