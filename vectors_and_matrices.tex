\documentclass[12pt]{article}
\usepackage[left = 1in, right = 1in, top = 1in, bottom = 1in]{geometry}
\usepackage{textcomp}
\usepackage{gensymb}
\usepackage{paralist}
\usepackage{cancel}
\usepackage{enumitem}
\usepackage{amsmath}
\usepackage{amssymb}
\usepackage{amsthm}
\usepackage{tkz-euclide}
\usepackage{hyperref}
\usepackage{esdiff}
\usepackage{parskip}
\usepackage{accents}
\usepackage{xcolor}

\usetikzlibrary{arrows.meta,positioning}

\newtheoremstyle{customstyle}
  {8pt} % Space above (adjust as needed)
  {0pt} % Space below (adjust as needed)
  {} % Body font
  {} % Indent amount
  {\bfseries} % Theorem head font
  {. } % Punctuation after theorem head
  {0pt} % Space after theorem head
  {} % Theorem head spec
\theoremstyle{customstyle}
\newtheorem{theorem}{Theorem}[section]
\newtheorem{exercise}{Exercise}[section]
\newtheorem{claim}[theorem]{Claim}
\newtheorem{prop}[theorem]{Proposition}
\newtheorem{corollary}[theorem]{Corollary}
\newtheorem{lemma}[theorem]{Lemma}
\newtheorem{definition}[theorem]{Definition}
\newtheorem{question}{Question}
\newtheorem{subquestion}{Part}[question]

\def\definitionautorefname{Definition}
\def\corollaryautorefname{Corollary}

\newenvironment{nonproof}{\par $\cancel {\text{\textit{Proof}}}.$}{\hfill$\cancel\square$}

\def\contra{\tikz[baseline, x=0.22em, y=0.22em, line width=0.032em]\draw (0,2.83)--(2.83,0) (0.71,3.54)--(3.54,0.71) (0,0.71)--(2.83,3.54) (0.71,0)--(3.54,2.83);}
\newenvironment{answer}{\par\noindent\textit{Answer.}}{\par}

\renewcommand{\CancelColor}{\color{red}}
\renewcommand{\Re}{\operatorname{Re}}
\renewcommand{\Im}{\operatorname{Im}}
\renewcommand{\bar}{\overline}
% \renewcommand{\vec}[1]{\undertilde{\mathrm{#1}}}
% \renewcommand{\vec}[1]{\undertilde{#1}}
% \renewcommand{\vec}[1]{\underline{#1}}
\renewcommand{\vec}[1]{\mathbf{#1}}
\newcommand{\unitvec}[1]{\hat{\vec{#1}}}
\newcommand{\sol}{$\operatorname{sol}^{\simeq}$}
\newcommand{\Arg}{\operatorname{Arg}}
\newcommand{\Log}{\operatorname{Log}}
\newcommand{\vecspan}{\operatorname{span}}
\newcommand{\id}[1]{\operatorname{id}_{#1}}
\newcommand{\indic}[1]{i_{#1}}
\newcommand{\Sym}{\operatorname{Sym}}
\newcommand{\Isom}{\operatorname{Isom}}
\newcommand{\di}{\mathrm{d}}
\newcommand{\gen}[1]{\langle{#1}\rangle}

\definecolor{applegreen}{rgb}{0.55, 0.71, 0.0}
\definecolor{ufogreen}{rgb}{0.24, 0.82, 0.44}


\begin{document}
% Lectured by Angela Capel Gueras (ac2722)
% Office: BO.14, Department of Applied Mathematics and Theoretical Physics

\section{Complex Numbers}

\subsection{Definition}

We construct $\mathbb{C}$ by adding $i$ to $\mathbb{R}$, where $i^{2} = -1$.
Then, any $z \in \mathbb{C}$ has the form $z = x + iy$, where $x,y \in \mathbb{R}$.
We define the real part, $x = \Re(z)$, and the imaginary part, $y = \Im(z)$.

\subsection{Properties}

\begin{compactenum}[i)]
\item Addition: $z_{1} \pm z_{2} = (x_{1} \pm x_{2}) + i(y_{1} \pm y_{2})$.
\item Multiplication: $z_{1}z_{2} = (x_{1}y_{1} - x_{2}y_{2}) + i(x_{1}y_{2} + x_{2}y_{1})$.
\end{compactenum}
Note that both addition and multiplication are associative and commutative.
\begin{compactenum}[i)]
\setcounter{enumi}{2}
\item Identity: The group $(\mathbb{C},+)$ is an abelian group with identity $0$.
\item Inverse: The \emph{inverse of $z$} is given by
    \[
    z^{-1} = \frac{x - iy}{x^{2} + y^{2}}
    \]
    and it satisfies $z \cdot z^{-1} = 1$.
\end{compactenum}

Thus, $(\mathbb{C}^{*}, \cdot)$ is an abelian group with identity $1$,
where $\mathbb{C}^{*} = \mathbb{C} - \{0\}$.

Note that the distributive property is also satisifed, where
\[
    (z_{1}+z_{2})z_{3} = z_{1}z_{2} + z_{2}z_{3}
\]
\begin{compactenum}[i)]
\setcounter{enumi}{4}
\item Complex Conjugate: For any $z = x + iy$, 
    the \emph{complex conjugate of $z$}, denoted $\bar z$, or $z^{*}$, is equal to $x - iy$.
\end{compactenum}
Thus, we also have $\Re(z) = (z + \bar z) / 2$, and $\Im(z) = (z - \bar z) / 2i$.
We have rules
\begin{compactitem}
\item $\bar {\bar z} = z$
\item $\bar{z_{1} + z_{2}} = \bar z_{1} + \bar z_{2}$
\item $\bar {z_{1}z_{2}} = \bar z_{1} \bar z_{2}$
\end{compactitem}

\begin{compactenum}[i)]
\setcounter{enumi}{5}
\item Modulus: For any $z = x + iy$, we define the \emph{modulus of $z$} by $|z|$, or $r$, by
    a real and non-negative number such that
    \[
    |z|^{2} = x^{2} + y^{2}
    \]
\item Argument: The \emph{argument} of a complex number $z = x + iy \ne 0$
    is a real number, denoted by $\theta = \arg(z)$, such that
    \[
        z = r(\cos \theta + i \sin \theta)
    \]
    Which is the \emph{polar form} of $z$. We can verify
    that $\tan\theta = y/x$.
\end{compactenum}

Note that if $\theta$ is an argument of $z$, so is $\theta + 2\pi n$ for $n \in \mathbb{Z}$.
To make it unique, we restrict $-\pi < \theta \le \pi$.
This value of $\theta$ is the \emph{principal value}.

Remarks:
\begin{compactenum}[(1)]
\item $\mathbb{R} \subset \mathbb{C}$, since for $a \in \mathbb{R}$ we have
    $a = a + i 0 \in \mathbb{C}$.
\item A complex number $0 + ib$ is said to be a \emph{pure imaginary number}.
\item The representation of a complex number in terms
    of its real and imaginary parts is unique.
\end{compactenum}

\subsubsection*{More Properties / Consequences}
\begin{compactenum}[(i)]
\setlength{\parskip}{4pt}
\item $(\mathbb{C},+,\cdot)$ is a field.
\item \emph{Fundamental Theorem of Algebra}:

    A polynomial of degree $n$ with coefficients in $\mathbb{C}$
    can be written as the product of $n$ linear factors:

    \[
        \begin{aligned}
            p(z) &= c_n z^{n} + \cdots + c_{0} \\
                &= c_n (z - \alpha _1)\cdots(z - \alpha_2)
        \end{aligned}
    \]
    
    where $c_i \in \mathbb{C}$, and $c_n \ne 0$, with roots $\alpha_i \in \mathbb{C}$.
    Thus, $p(z) = 0$ has at least one root,
    and $n$ roots counted with multiplicity.
\item The modulus satisifes the following properties:
    \begin{itemize}
        \item $|z_{1}z_{2}| = |z_{1}||z_{2}|$
        \item $|z_{1}| + |z_{2}| \le |z_{1}| + |z_{2}|$
        \item $|z_{1} - z_{2}| \ge \left||z_{1}| - |z_{2}|\right|$
    \end{itemize}
\end{compactenum}

\subsubsection*{De Moivre's Theorem}

\begin{lemma}
    If $z_{1} = r_{1}(\cos \theta_{1} + i \sin \theta_{1})$,
    and $z_{2} = r_{2}(\cos \theta_{2} + i \sin \theta_{2})$,
    then $z_{1}z_{2} = r_{1}r_{2}(\cos (\theta_{1} + \theta_{2}) + i \sin (\theta_{1} + \theta_{2})$.
\end{lemma}

\begin{proof}
    Multiplying naively gives
    \[
    z_{1}z_{2} = r_{1}r_{2}(
        (\cos\theta_{1} \cos\theta_{2} - \sin\theta_{1}\sin\theta_{2})
        +i(\cos\theta_{1} \sin\theta_{2} + \cos\theta_{2} \sin\theta_{1})
    )
    \]
    Applying the addition rule of $\sin$ and $\cos$, the result is obtained.
\end{proof}

\begin{theorem}[De Moivre's]
    \label{thm:de_moivre}
    For any $\theta \in \mathbb{R}$, $n \in \mathbb{Z}$, we have
    \[
        (\cos\theta + i\sin\theta )^{n} = \cos(n\theta ) + i\sin(n\theta ).
    \]
\end{theorem}
\begin{proof}
    The proof is trivial and left as an exercise to the reader.
\end{proof}

\setcounter{subsection}{3}
\subsection{Exponential and Trigonometric Functions}

\subsubsection{Exponential Functions}

We define the exponential function on $z \in \mathbb{C}$ by
\[
    \exp(z) = e^{z} = \sum\limits_{n=0}^{\infty}\frac{1}{n!}z^{n}.
\]
The function only exists when the series converges,
and it does for all $z \in \mathbb{C}$.
A proof will not be provided for that result.

Properties:
\begin{compactitem}
\item $e^{z}e^{w} = e^{z+w}$, for all $z,w \in \mathbb{C}$.
\item If $z \in \mathbb{R}$, then $\exp(z)$ reduces to usual exponential.
\item $e^{0} = 1$.
\item $(e^{z})^{n} = e^{nz}$.
\end{compactitem}

\subsubsection{Trigonometric Functions}

We have
\begin{align*}
    \cos(z) &= \frac{1}{2}(e^{iz} + e^{-iz}) \\
            &= \frac{1}{2}\left(
            \sum\limits_{n=0}^{\infty}\frac{1}{n!}(iz)^{n}
            +\sum\limits_{n=0}^{\infty}\frac{1}{n!}(-iz)^{n} \right) \\
            &= \sum\limits_{n=0}^{\infty}(-1)^{n}\frac{z^{2n}}{(2n)!}
\end{align*}

Similarly, 
\begin{align*}
    \sin(z) &= \frac{1}{2i}(e^{iz} - e^{-iz}) \\
            &= \frac{1}{2i}\left(
            \sum\limits_{n=0}^{\infty}\frac{1}{n!}(iz)^{n}
            -\sum\limits_{n=0}^{\infty}\frac{1}{n!}(-iz)^{n} \right) \\
            &= \sum\limits_{n=0}^{\infty}(-1)^{n}\frac{z^{2n+1}}{(2n+1)!}
\end{align*}

If $z \in \mathbb{R}$, this clearly reduces to the usual trigonometric functions.

It is also possible to notice that
\[
e^{iz} = \cos z + i\sin z,
\]
in particular, if $z = \pi$, then
\begin{equation*}
    e^{i\pi} = -1
\end{equation*}
which is \emph{Euler's Identity}.

\begin{corollary}
    $e^{z} = 1 \iff z = 2\pi ni$,
    where $n \in \mathbb{Z}$.
\end{corollary}
\begin{proof}
    Let $z = x+iy$, then 
    $e^{z} = e^{x}e^{iy} = e^{x}(\cos y+i\sin y) 
    = e^{x}\cos y + ie^{x} \sin y = 1$.
    Then, $e^{x}\sin y = 0$, so $\sin y = 0$, so $y = n\pi$. 
    Similarly, $e^{x}\cos y = 1$,
    substituting gives $e^{x}(\pm 1) = 1$, 
    and since $e^{x}$ is positive, $\cos y = 1$, 
    so $y = 2n\pi$.
    Thus, $e^{x} = 1$, so $x = 0$, and $z = 2\pi n i$.
\end{proof}

Finally, we can write $z = r(\cos\theta + i\sin\theta) = re^{i\theta }$,

\subsubsection{Roots of Unity}

Let $z = re^{i\theta }$, and let for some $N \in \mathbb{N}$ that $z^{N} = 1$.
Then we have
\[
z^{N} = r^{N}e^{i\theta N}.
\]
Trivially then we have $r = 1$, $\theta N= 2\pi n$, so
\[
    z = e^{\frac{2\pi i}{N}n}
\]

\subsection{Logarithm and Complex Powers}

We say, if $z \in \mathbb{C}$ and $z \ne 0$, that $w = \log z$ iff $e^{w} = z$.
By definition, $\log$ is the inverse of $\exp$. 
Note that $\log$ here means the natural logarithm.

Importantly, this is a many-to-one relationship.
\[
r = re^{i\theta } = e^{\log r}e^{i\theta } = e^{\log r + i\theta },
\]
so
\[
\log z = \log r + i\theta  = \log r + i \arg(z)
\]
for any $\theta  \in \Arg(z)$.
Thus, if $r + i\theta $ is a log of $z$, then so is $r + i(\theta + 2\pi)$.
To make it unique, we simply use the principal value of $\arg$,
as in $-\pi < \theta  \le \pi$.

\subsubsection*{Complex Powers}

We define $z$ to the power of $\alpha $ to be
\[
z^{\alpha } = e^{\alpha \log z}
\]
This is multivalued due to the multi-valued nature of $\log$.

Examples:
\begin{compactitem}
\item $\log(i) = \log(e^{i\pi / 2}) = i \left(\frac{\pi}{2} + 2\pi n\right)$.
\item $\log(1 + i) = \log\sqrt{2} + i\left(\frac{\pi}{4} + 2\pi n\right)$.
\item $(1 + i)^{1/2} = \sqrt[4]{2} + i\left(\frac{\pi}{8} + \pi n\right)$.
\end{compactitem}

\subsection{Lines and Circles}

\subsubsection*{Lines}

Lines are defined by a point $z_{0} \in \mathbb{C}$ and a direction $\omega \in \mathbb{C}$,
and given by
\[
z = z_{0} + \lambda \omega
\]
where $\lambda \in \mathbb{R}$.
Taking conjugates, we have $\bar z = \bar z_{0} + \lambda \bar \omega$,
so $\bar \omega z - \omega \bar z = \bar \omega z_{0} - \omega \bar z_{0}$.

\subsubsection*{Circles}

Circles are defined by a center $c \in \mathbb{C}$ and a radius $\rho > 0$,
and given by
\[
z = c + \rho e^{i\theta}
\]
where $\theta \in \mathbb{R}$. Note that $|z - c| = \rho$.

\end{document}
