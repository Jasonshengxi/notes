\documentclass[12pt]{article}
\usepackage[left = 1in, right = 1in, top = 1in, bottom = 1in]{geometry}
\usepackage{paralist}
\usepackage{cancel}
\usepackage{enumitem}
\usepackage{amsmath}
\usepackage{amssymb}
\usepackage{amsthm}
\usepackage{tkz-euclide}
\usepackage{hyperref}
\usepackage{esdiff}
\usepackage{parskip}
\usepackage{accents}
\usepackage{xcolor}

\usetikzlibrary{arrows.meta,positioning}

\newtheoremstyle{customstyle}
  {8pt} % Space above (adjust as needed)
  {0} % Space below (adjust as needed)
  {} % Body font
  {} % Indent amount
  {\bfseries} % Theorem head font
  {. } % Punctuation after theorem head
  {0pt} % Space after theorem head
  {} % Theorem head spec
\theoremstyle{customstyle}
\newtheorem{theorem}{Theorem}[section]
\newtheorem{corollary}{Corollary}[theorem]
\newtheorem{lemma}{Lemma}[section]
\newtheorem{definition}{Definition}[section]

\def\definitionautorefname{Definition}
\def\corollaryautorefname{Corollary}

\newenvironment{nonproof}{\par \textit{Nonproof:}}{\hfill$\cancel\square$}

\def\contra{\tikz[baseline, x=0.22em, y=0.22em, line width=0.032em]\draw (0,2.83)--(2.83,0) (0.71,3.54)--(3.54,0.71) (0,0.71)--(2.83,3.54) (0.71,0)--(3.54,2.83);}

\renewcommand{\Re}{\operatorname{Re}}
\renewcommand{\Im}{\operatorname{Im}}
\renewcommand{\bar}{\overline}


\begin{document}
% Lectured by Dr Zoe Wyatt (zw253)

\section{Introduction}

Expect:
\begin{compactitem}
    \item precise definitions
    \item rigorous proofs
    \item foundational questions
\end{compactitem}

We start with assumptions called \emph{axioms}.

A \emph{statement} is a sentence that can have a true or false value.
A \emph{proof} is a sequence of true statements without logical gaps
establishing some conclusion.

\begin{compactitem}
\item show they are true
\item gain insight into why they are true
\item the proof might be cool
\end{compactitem}

\subsection{Number Systems}

Define $\mathbb{N}$ as the set of natural numbers, $\{1,2,3,\cdots\}$ (note the lack of $0$).
Define $\mathbb{Z},\mathbb{Q},\mathbb{R}$ to be 
integer, rational, and real sets respectively.

A real number is \emph{algebraic} if it is the
root of some polynomial with integer coefficients.
Non-algebraic numbers are called \emph{transcendental}.

The existence of a transcendental number was shown in 1844.

\subsection{Some Proofs and Nonproofs}

\begin{claim}
    For all positive integers $n$, $n^{3} - n$ is always a multiple of $3$.
\end{claim}
\begin{proof}
    Let $n \in \mathbb{N}$, we have $n^{3} - n = n(n^{2} - 1) = (n - 1) \times n \times (n+1)$.
    One of the three consecutive integers $n-1,n,n+1$ must be a multiple of $3$.
    Thus, the product $n^{3} - n$ must also be a multiple of $3$.
\end{proof}

Note the little box given for free by \LaTeX, on the right side of the page.
That denotes the end of the proof.

\begin{claim}
    For any positive integer $n$, if $n^{2}$ is even then so is $n$.
\end{claim}
\begin{nonproof}
    Given $n \in \mathbb{N}$, we can write $n = 2k$ where $k \in \mathbb{N}$.
    We have $n^{2} = 4k^{2} = 2(2k^{2})$, which is even.
\end{nonproof}

\begin{proof}
    Suppose on the contrary that $n^{2}$ is even so $n = 2k+1$ for some $k \in \mathbb{N}$.
    Then we have,
    \[
    \begin{aligned}
        n^{2} = (2k+1)^{2} &= 4k^{2} + 4k + 1 \\
                     &= 4(k^{2} + k) + 1.
    \end{aligned}
    \]
    which is odd, contradicting the assumption that $n^{2}$ is even.
    \contra
\end{proof}

\begin{claim}
    The \sol{} to $x^{2} - 5x + 6 = 0$ is $x = 2$ or $x = 3$.
\end{claim}
\begin{proof}
    If $x = 2$ or $x = 3$,
    then $x - 2 = 0$ or $x - 3 = 0$.
    So $(x-2)(x-3) = x^{2} - 5x + 6 = 0$.

    If $x^{2}-5x+6=0$, then
    $(x-2)(x-3)=0$,
    so either $x-2=0$ or $x-3=0$,
    which yields $x=2$ or $x=3$.
\end{proof}

Alternatively, we can write
\begin{proof}
    \begin{align*}
        &x=2 \quad \text{or} \quad x=3\\
        \iff &x-2=0 \quad\text{or}\quad x-3=0 \\
        \iff &(x-2)(x-3)=0\\
        \iff &x^{2}-5x+6=0
    \end{align*}
\end{proof}

\begin{claim}
    Every positive real number is greater than or equal to $1$.
\end{claim}
\begin{nonproof}
    Let $r$ be the least positive real.
    Either $r = 1$ or $r < 1$ or $r > 1$.

    If $r < 1$, then $0 < r^{2} < r$, 
    so $r^{2}$ is a smaller positive real. \contra

    If $r > 1$, then $0 < \sqrt{r} < r$,
    so $\sqrt{r}$ is a small positive real. \contra
    
    Thus, $r = 1$.
\end{nonproof}

The problem lies in the nonexistence of a least positive real.
The moral is that
\begin{center}
    \underline{Every claim must be justified}.
\end{center}

\subsection{Combining Claims}

The truth of assertions like $A \land B$ and $A \lor B$ depend on
the truth of $A$ and $B$, as summarised in the \emph{truth table}:

\begin{table}[h]
    \centering
    \begin{tabular}{ |c|c|c|c|c|c|c| }
        \hline
        A & B & $A \land B$ & $A \lor B$ & $\neg A$ & $A \Rightarrow B$ \\
        \hline
        F & F & F & F & T & T\\
        F & T & F & T & T & T\\
        T & F & F & T & F & F\\
        T & T & T & T & F & T\\
        \hline
    \end{tabular}
    \caption{The truth table of various logical constructs.}
\end{table}

Note, for example, that $\neg (A \land B)$ is equivalent to
$(\neg A) \lor (\neg B)$, by comparing truth tables.
Similarly, $A \Rightarrow B$ is equivalent to $(\neg A) \lor B$,
and hence $B \lor (\neg A)$,
and hence to $(\neg B) \Rightarrow (\neg A)$.
This is called the \emph{contrapositive}.

\subsubsection*{Negating Quantifiers}

A claim may involve "quantifiers" like $\forall$ or $\exists$.
$\neg(\forall x, A(x))$ is equivalent to $\exists x, \neg A(x)$.
Similarly, $\neg(\exists x, A(x)) \iff \forall x, \neg A(x)$.

\section{Sets, Functions, and Relations}

A \emph{set} is a collection of mathematical objects.
For example, $\mathbb{R}$, $\mathbb{N}$, $\{1,5,9\}$, $(-2,3]$.

Two important facts required for a set are that 
\begin{compactenum}
\item the order of elements in the set is immaterial, and
\item each element in the set occurs only once.
\end{compactenum}

For instance, $\{1,3,7\} = \{1,7,3\}$, and $\{3,4,4,8\} = \{3,4,8\}$.

Two sets are equal if they have the same elements. 
That is, $A = B$ iff $\forall x, x \in A \iff x \in B$.
There is only one \emph{empty set} $\emptyset$ i.e. the set with no elements.

A set $B$ is a \emph{subset} of $A$, written $B \subseteq A$, or $B \subset A$,
if every element of $B$ is an element of $A$, or equivalently $\forall x \in B, x \in A$.
$B$ is said to be a \emph{proper subset of A} if $B \subseteq A$ and $B \ne A$,
sometimes written $B \subsetneq A$.

Note that $A = B$ iff $A \subseteq B$ and $B \subseteq A$.

If $A$ is a set and $P$ is a property of (some) elements of $A$,
we can write $\{x \in A : P(x)\}$ for the usbset of $A$ comprising
of the elements $x$ for which $P(x)$ holds true.

For example, $\{u \in \mathbb{N} : n\text{ is prime}\} = \{2,3,5,7,11,\cdots\}$.

If $A,B$ are sets, then their \emph{union} $A \cup B$ is
given by $\{x : x \in A \lor x \in B\}$.
Their \emph{intersection} $A \cap B$ is defined to be
$\{x : x \in A \land x \in B\}$.
We say $A$ and $B$ are \emph{disjoint} if $A \cap B = \emptyset$.

Note that we can view intersection as a special case of subset selection,
in that $A \cap B = \{x \in A : x \in B\}$.

Similarly, have \emph{set difference} $A \setminus B = \{x \in A : x \notin B\}$.
Note that set difference is non-commutative, whereas $\cap$ and $\cup$ 
are commutative and associative.

Laos $\cup$ and $\cap$ are distributive over each other, in that
\begin{align*}
    A \cup (B \cap C) = (A \cup B) \cap (A \cup C),\\
    A \cap (B \cup C) = (A \cap B) \cup (A \cap C).\\
\end{align*}

\section{Integers and Counting}
\section{Elementary Number Theory}
\section{Real Numbers}
\section{(Un)Countability}

\appendix

\newpage
\section{Notation}

We write $A \implies B$ for "if A then B".
We write \sol{} for a solution, as in the \sol{} of a polynomial, or $3$ is a \sol{}.

If $A$ and $B$ are assertions,
we can (but usually don't) write $A \land B$ for "A and B",
and $A \lor B$ for "A or B".
We can similarly write $\neg A$ for "not A".

We write $x \in A$ if $x$ is an element of the set $A$, and $x \notin A$ if not.

\end{document}
