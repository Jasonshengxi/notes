\documentclass[12pt]{article}
\usepackage[left = 1in, right = 1in, top = 1in, bottom = 1in]{geometry}
\usepackage{textcomp}
\usepackage{gensymb}
\usepackage{paralist}
\usepackage{cancel}
\usepackage{enumitem}
\usepackage{amsmath}
\usepackage{amssymb}
\usepackage{amsthm}
\usepackage{tkz-euclide}
\usepackage{hyperref}
\usepackage{esdiff}
\usepackage{parskip}
\usepackage{accents}
\usepackage{xcolor}

\usetikzlibrary{arrows.meta,positioning}

\newtheoremstyle{customstyle}
  {8pt} % Space above (adjust as needed)
  {0pt} % Space below (adjust as needed)
  {} % Body font
  {} % Indent amount
  {\bfseries} % Theorem head font
  {. } % Punctuation after theorem head
  {0pt} % Space after theorem head
  {} % Theorem head spec
\theoremstyle{customstyle}
\newtheorem{theorem}{Theorem}[section]
\newtheorem{exercise}{Exercise}[section]
\newtheorem{claim}[theorem]{Claim}
\newtheorem{prop}[theorem]{Proposition}
\newtheorem{corollary}[theorem]{Corollary}
\newtheorem{lemma}[theorem]{Lemma}
\newtheorem{definition}[theorem]{Definition}
\newtheorem{question}{Question}
\newtheorem{subquestion}{Part}[question]

\def\definitionautorefname{Definition}
\def\corollaryautorefname{Corollary}

\newenvironment{nonproof}{\par $\cancel {\text{\textit{Proof}}}.$}{\hfill$\cancel\square$}

\def\contra{\tikz[baseline, x=0.22em, y=0.22em, line width=0.032em]\draw (0,2.83)--(2.83,0) (0.71,3.54)--(3.54,0.71) (0,0.71)--(2.83,3.54) (0.71,0)--(3.54,2.83);}
\newenvironment{answer}{\par\noindent\textit{Answer.}}{\par}

\renewcommand{\CancelColor}{\color{red}}
\renewcommand{\Re}{\operatorname{Re}}
\renewcommand{\Im}{\operatorname{Im}}
\renewcommand{\bar}{\overline}
% \renewcommand{\vec}[1]{\undertilde{\mathrm{#1}}}
% \renewcommand{\vec}[1]{\undertilde{#1}}
% \renewcommand{\vec}[1]{\underline{#1}}
\renewcommand{\vec}[1]{\mathbf{#1}}
\newcommand{\unitvec}[1]{\hat{\vec{#1}}}
\newcommand{\sol}{$\operatorname{sol}^{\simeq}$}
\newcommand{\Arg}{\operatorname{Arg}}
\newcommand{\Log}{\operatorname{Log}}
\newcommand{\vecspan}{\operatorname{span}}
\newcommand{\id}[1]{\operatorname{id}_{#1}}
\newcommand{\indic}[1]{i_{#1}}
\newcommand{\Sym}{\operatorname{Sym}}
\newcommand{\Isom}{\operatorname{Isom}}
\newcommand{\di}{\mathrm{d}}
\newcommand{\gen}[1]{\langle{#1}\rangle}

\definecolor{applegreen}{rgb}{0.55, 0.71, 0.0}
\definecolor{ufogreen}{rgb}{0.24, 0.82, 0.44}


\begin{document}
% Lectured by Dr Zoe Wyatt (zw253)

\section{Introduction}

Expect:
\begin{compactitem}
    \item precise definitions
    \item rigorous proofs
    \item foundational questions
\end{compactitem}

We start with assumptions called \emph{axioms}.

A \emph{statement} is a sentence that can have a true or false value.
A \emph{proof} is a sequence of true statements without logical gaps
establishing some conclusion.

\begin{compactitem}
\item show they are true
\item gain insight into why they are true
\item the proof might be cool
\end{compactitem}

\subsection{Number Systems}

Define $\mathbb{N}$ as the set of natural numbers, $\{1,2,3,\cdots\}$ (note the lack of $0$).
Define $\mathbb{Z},\mathbb{Q},\mathbb{R}$ to be 
integer, rational, and real sets respectively.

A real number is \emph{algebraic} if it is the
root of some polynomial with integer coefficients.
Non-algebraic numbers are called \emph{transcendental}.

The existence of a transcendental number was shown in 1844.

\subsection{Some Proofs and Nonproofs}

\begin{theorem}
    For all positive integers $n$, $n^{3} - n$ is always a multiple of $3$.
\end{theorem}
\begin{proof}
    Let $n \in \mathbb{N}$, we have $n^{3} - n = n(n^{2} - 1) = (n - 1) \times n \times (n+1)$.
    One of the three consecutive integers $n-1,n,n+1$ must be a multiple of $3$.
    Thus, the product $n^{3} - n$ must also be a multiple of $3$.
\end{proof}

Note the little box given for free by \LaTeX, on the right side of the page.
That denotes the end of the proof.

\begin{theorem}
    For any positive integer $n$, if $n^{2}$ is even then so is $n$.
\end{theorem}
\begin{nonproof}
    Given $n \in \mathbb{N}$, we can write $n = 2k$ where $k \in \mathbb{N}$.
    We have $n^{2} = 4k^{2} = 2(2k^{2})$, which is even.
\end{nonproof}

\begin{proof}
    Suppose on the contrary that $n^{2}$ is even so $n = 2k+1$ for some $k \in \mathbb{N}$.
    Then we have,
    \[
    \begin{aligned}
        n^{2} = (2k+1)^{2} &= 4k^{2} + 4k + 1 \\
                     &= 4(k^{2} + k) + 1.
    \end{aligned}
    \]
    which is odd, contradicting the assumption that $n^{2}$ is even.
    \contra
\end{proof}

\section{Sets, Functions, and Relations}
\section{Integers and Counting}
\section{Elementary Number Theory}
\section{Real Numbers}
\section{(Un)Countability}

\end{document}
