\documentclass[12pt]{article}
\usepackage[left = 1in, right = 1in, top = 1in, bottom = 1in]{geometry}
\usepackage{paralist}
\usepackage{cancel}
\usepackage{enumitem}
\usepackage{amsmath}
\usepackage{amssymb}
\usepackage{amsthm}
\usepackage{tkz-euclide}
\usepackage{hyperref}
\usepackage{esdiff}
\usepackage{parskip}
\usepackage{accents}
\usepackage{xcolor}

\usetikzlibrary{arrows.meta,positioning}

\newtheoremstyle{customstyle}
  {8pt} % Space above (adjust as needed)
  {0} % Space below (adjust as needed)
  {} % Body font
  {} % Indent amount
  {\bfseries} % Theorem head font
  {. } % Punctuation after theorem head
  {0pt} % Space after theorem head
  {} % Theorem head spec
\theoremstyle{customstyle}
\newtheorem{theorem}{Theorem}[section]
\newtheorem{corollary}{Corollary}[theorem]
\newtheorem{lemma}{Lemma}[section]
\newtheorem{definition}{Definition}[section]

\def\definitionautorefname{Definition}
\def\corollaryautorefname{Corollary}

\newenvironment{nonproof}{\par \textit{Nonproof:}}{\hfill$\cancel\square$}

\def\contra{\tikz[baseline, x=0.22em, y=0.22em, line width=0.032em]\draw (0,2.83)--(2.83,0) (0.71,3.54)--(3.54,0.71) (0,0.71)--(2.83,3.54) (0.71,0)--(3.54,2.83);}

\renewcommand{\Re}{\operatorname{Re}}
\renewcommand{\Im}{\operatorname{Im}}
\renewcommand{\bar}{\overline}


\begin{document}
% Lectured by Henry Wilton (hjrw2)

\section{Examples and Definitions}

You can think of groups in two ways:
It is either something to do with algebra,
or something to do with symmetry.

We will begin with the symmetry point of view.

\subsection{Symmetry}

An equilateral triangle has at least two different kinds of symmetry.
It has \emph{rotational} and \emph{reflective} symmetry,
and most importantly, the identity symmetry.

\begin{figure}[h]
    \centering
    \begin{tikzpicture}[scale=3]
        \draw (90:1) coordinate (A) 
            -- (210:1) coordinate (B) -- (330:1) coordinate (C) -- cycle;
    \end{tikzpicture}
    \caption{The "identity" symmetry}
\end{figure}

Combining all symmetries, there are 6 different symmetries 
on an equilateral triangle:
\begin{compactenum}
    \item identity,
    \item rotate by $120\degree$ anticlockwise,
    \item rotate by $240\degree$ anticlockwise
    \item 5. 6. reflections along the three axes.
\end{compactenum}

\begin{exercise}
    How many symmetries does a regular $n$-gon have?
\end{exercise}
\begin{answer}
    the number of symmetries of a regular $n$-gon is $2n$.
\end{answer}

Now we begin to consider the \emph{composition} of symmetries,
or, from an algebraic standpoint, the \emph{product} of symmetries.

Say we compose a rotation by $120\degree$ anticlockwise and
a flip around the vertical axis. This is a reflection around 
the axis of positive slope.

Some important features to notice from this are:
\begin{compactitem}
\item Symmetries can be composed, 
    i.e. symmetries are closed under composition.
\item There is an identity symmetry that leaves everything the same.
\item Every symmetry has an inverse. Specifically, in this case:
    \begin{compactitem}
    \item the inverse of the identity is itself.
    \item the inverse of a rotation is the opposite rotation.
    \item the inverse of a reflection is itself.
    \end{compactitem}
\item Composition of symmetries is \emph{associative},
    as in $a(bc) = (ab)c$, but for symmetries.
\end{compactitem}
Additionally, there is one important non-feature: 
symmetries are not always commutative. 
It is trivial to check by multiplying a rotation by a reflection,
then the other way around.

But isn't this whole drawing thing exhausting (especially for me)?
Let's go to the algebra side.

\subsection{Algebra}

\begin{definition}
    A binary operation on a set $X$ is
    a function $f : X \times X \to X$.
    It is also denoted $\cdot$ for multiplication.
    We write $a \cdot b$ for $f(a,b)$.
\end{definition}

Note that this definition includes closure,
in that for any $x,y \in X$, we have $f(x,y) \in X$.

\begin{definition}
    A group is a triple $(G,\cdot,e)$ where
    $G$ is a set, $\cdot$ is a binary operation on $G$,
    and $e \in G$, which satisfies the following four axioms:
    \begin{compactenum}[(i)]
    \item Closure: for all $a,b \in G$, we have $a \cdot b \in G$.
    \item Associativity: for all $a,b,c \in G$, we have $(a \cdot b) \cdot c = a \cdot (b \cdot c)$.
    \item (Right) Identity: for all $a \in G$, we have $a \cdot e = a$.
    \item (Right) Inverse: for all $a \in G$, there exists $a^{-1} \in G$ s.t. $a \cdot a^{-1} = e$.
    \end{compactenum}
\end{definition}

Note that the closure axiom is indeed redundant,
by the definition of a \emph{binary operation on G}.

Example: Notice that the set of symmetries of 
the equilateral triangle forms a group.

We can also think of this definition as
encompassing \emph{algebra with one operation}.

\begin{exercise}
    $(\mathbb{Z},+,0)$ forms a group.
\end{exercise}
\begin{proof}
    Closure and identity and associativity are trivial.
    The inverse of $x \in \mathbb{Z}$ is $-x$, in that $x + (-x) = 0$.
\end{proof}

\begin{theorem}
    Let $(G,\cdot,e)$ be a group, and $a,b,b'e' \in G$. We have:
    \begin{compactenum}[(i)]
    \item If $a \cdot b = e$ then $b \cdot a = e$. \\
        (i.e. right inverses are left inverses)
    \item If $e \cdot a = a$. \\
        (i.e. right identities are left identities)
    \item If $a,b,b'$ are s.t. $a \cdot b = e = a \cdot b'$,
        then $b = b'$. \\
        (i.e. inverses are unique)
    \item If $a \cdot e' = a$ then $e' = e$.
        (i.e. the identity is unique).
    \end{compactenum}
\end{theorem}

\end{document}
