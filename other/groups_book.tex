\documentclass{article}
\usepackage[left = 1in, right = 1in]{geometry}
\usepackage{paralist}
\usepackage{enumitem}
\usepackage{amsmath}
\usepackage{amssymb}
\usepackage{amsthm}
\usepackage{tikz}
\usepackage{hyperref}
\newtheorem{theorem}{Theorem}[section]
\newtheorem{corollary}{Corollary}[theorem]
\newtheorem{lemma}{Lemma}[section]
\newtheorem{definition}{Definition}[section]
\def\definitionautorefname{Definition}
\def\corollaryautorefname{Corollary}

\title{Replica of Groups and Symmetry}
\author{JsonJ\_\_}

\begin{document}
\maketitle

\section{Symmetries of the Tetrahedron}
\section{Axioms}

\begin{definition}
    A group is a tuple $(X,f)$ where $X$ is a set and $f : X \times X \to X$ is a multiplication on that set.
    It must also satisfy the following rules:
    \begin{itemize}[noitemsep]
        \item Identity: $\exists e \in X, \forall x \in X, ex = xe = x$
        \item Associativity $\forall x,y,z \in X, x(yz) = (xy)z$
        \item Inverse: $\forall x \in X, \exists x^{-1} \ in X, xx^{-1} = x^{-1}x = e$
        \item Closure: $\forall x,y \in X, xy \in X$
    \end{itemize}
    Note that closure is already encoded in the definition of multiplication, but is stated again for clarity.
\end{definition}

\begin{theorem}
    The identity of a group is unique.
\end{theorem}
\begin{proof}
    Assume $e_{1}$ and $e_{2}$ are both identities.
    $e_{1} = e_{1}e_{2} = e_{2}$.
\end{proof}

\begin{theorem}
    The inverse of an element is unique.
\end{theorem}
\begin{proof}
    Let $x \in G$.
    If $y$ and $z$ are both inverses of $x$,
    $y = ey = (zx)y = z(xy) = ze = z$.
\end{proof}

\begin{theorem}
    $(xy)^{-1} = y^{-1}x^{-1}$
\end{theorem}
\begin{proof}
    \[
        \begin{aligned}
            (xy)^{-1}xy &= e \\
            (xy)^{-1}xyy^{-1}x^{-1} &= y^{-1}x^{-1} \\
            (xy)^{-1}xx^{-1} &= y^{-1}x^{-1} \\
            (xy)^{-1} &= y^{-1}x^{-1}
        \end{aligned}
    \]
\end{proof}

\section{Numbers}

\begin{definition}
    An \emph{abelian} group is such that $\forall x,y \in G, xy = yx$.
\end{definition}

\section{Dihedral Groups}

\begin{definition}
    The dihedral group $D_n$ is 
    $\langle r,s \mid r^{n} = s^{2} = e, sr^{-1} = rs\rangle$
\end{definition}

It contains the elements $r^{m}$ and $r^{m}s$ for all $0 \le m < n$.

\begin{definition}
    The \emph{order} of a group is the number of elements in the group.
\end{definition}

\begin{definition}
    The \emph{order} of an element $x$ in a group is
    the least $m$ that satisfies $x^{m} = e$.
    If no such $m$ exists, we define the order to be infinity.
\end{definition}

\begin{theorem}
    Given $x,y \in G$, if $x,y,xy$ all have order 2, $xy = yx$.
\end{theorem}
\begin{proof}
    $xy = (xy)^{-1} = y^{-1}x^{-1} = yx$
\end{proof}

\begin{theorem}
    Given $x,g \in G$, $gxg^{-1}$ has the same order as $x$.
\end{theorem}
\begin{proof}
    Clearly, $(gxg^{-1})^{n} = g(x^{n})g^{-1}$.
    If $n$ is the order of $x$, $x^{n} = e$.
    Since $(gxg^{-1})^{n} = g(x^{n})g^{-1} = gg^{-1} = e$, $n$ satisfies the condition for order.
    Assume $m < n$ exists such that $(gxg^{-1})^{m} = e$.
    We have $x^{m} = g^{-1}eg = g^{-1}g = e$, so $m$ satisfies $x^{m} = e$, which cannot be true.
    Thus, $n$ is the smallest number satisfying $(gxg^{-1})^{n} = e$, and is the order of $gxg^{-1}$.
\end{proof}

\section{Subgroups and Generators}

\begin{definition}
    A \emph{subgroup} is a subset of a group that is also a group itself, with the same multiplication.
    It is denoted $H < G$.
\end{definition}
\begin{definition}
    The \emph{subgroup generated by $x$}, written $\langle x\rangle$,
    is the group composed of all $x^{m}, m \in \mathbb{Z}$.
\end{definition}

\begin{theorem}
    \label{thm:order_elem_order_subgroup}
    The order of an element $x$ is equal to the order of $\langle x\rangle$.
\end{theorem}
\begin{proof}
    Assume $x$ has infinite order,
    No $n,m \in \mathbb{Z}$ exist such that $x^{n} = x^{m}$, 
    since that would imply $x^{n - m} = e = x^{m - n}$, which violates an assumption.
    Thus, $\langle x\rangle$ has infinite order.

    Assume $x$ has finite order equal to $n$.
    We claim that $\langle x\rangle$ is composed of $x^{m}$ for $0 \le m < n$.
    First, no $x^{a} = x^{b}$ where $0 < a < b < n$,
    since that would imply $x^{b - a} = e$, where $0 < b - a < n$, which violates an assumption.
    Thus, all $x^{m}$ are distinct.
    For any other integer $a$, there exists $q \ne 0$ and $0 \le r < n$ such that $a = qn + r$.
    Trivially, $x^{a} = e^{q}x^{r} = x^{r}$.
    Thus, $\langle x\rangle$ is composed of $x^{m}$ where $0 \le m < n$, and has order $n$.
\end{proof}

\begin{definition}
    Given a subset $X$ of a group $G$, we consider the following expression:
    \[
        x_{1}^{m_{1}}x_{2}^{m_{2}} \cdots x_{k}^{m_{k}}
    \]
    where $x_i \in X, m_i \in \mathbb{Z}$. This is called a \emph{word} in the elements of $X$.
    The set of all words in the elements of $X$ forms a group,
    and is called the \emph{subgroup generated by $X$}, denoted $\langle x_{1},x_{2}, \cdots x_n\rangle$,
    where $x_i$ are the distinct elements of $X$ and $n = |X|$.
    If this group fills $G$, we call $X$ a \emph{set of generators for $G$},
    or \emph{the elements of $X$ generate $G$}.
\end{definition}
\begin{proof}
    Let $H$ be the subgroup generated by $X$.
    Clearly, all words are elements of $G$, so $H \subseteq G$.
    Inverse is given by $x_k^{-m_k}x_{k-1}^{-m_{k-1}} \cdots x_{1}^{-m_1} \in H$, and can be checked trivially.
    Closure is trivially checked.
    Combining closure and inverse is sufficient for \autoref{thm:xyiv_subgroup}.
\end{proof}

\begin{theorem}
    If $X$ is a set of generators for $G$,
    and $X$ is a subset of the subgroup generated by $Y$,
    $Y$ is a set of generators for $G$.
\end{theorem}
\begin{proof}
    let $H$ and $K$ be the subgroups generated by $X$ and $Y$ respectively.
    We have that $G \subseteq H$ and $X \subseteq K$.
    Consider a word from $H$, $h = x_{1}^{m_{1}}x_{2}^{m_{2}}\cdots$
    Since $X \subseteq K$, and $x_i \in X$, we have $x_i \in K$.
    Thus, there exists $y_{i1}^{m_{i1}}y_{i2}^{m_{i2}}\cdots = x_i$.
    Substituting, it is clear that $h \in K$, so $H \subseteq K$.
    Thus, $G \subseteq H \subseteq K$, so $K$ fills $G$,
    and $Y$ is a set of generators for $G$.
\end{proof}

\begin{definition}
    A group is \emph{cyclic} if there exists $x \in G$ such that $\langle x\rangle = G$.
\end{definition}

Note that the trivial group is trivially cyclic, with $\langle e\rangle = \{e\}$.

\begin{theorem}
    \label{thm:xyiv_subgroup}
    If $H \subseteq G$ and $H \ne \varnothing$,
    $\forall x,y \in H, xy^{-1} \in H$
    iff $H$ is a subgroup of $G$.
\end{theorem}
\begin{proof}
    If $H$ is a subgroup of $G$, $H$ is a group,
    so $xy^{-1} \in H$.
    If $xy^{-1} \in H$, we show the properties of a group:
    Since $H$ is nonempty, there must exist $x \in H$,
    and $xx^{-1} = e \in H$;
    For any element $y \in H$, $ey^{-1} = y^{-1} \in H$,
    so the inverse always exists.
    Thus, $x(y^{-1})^{-1} = xy \in H$, showing closure.
    Since $H$ is a subset of group $G$, associativity is always given.
    Thus, H is a subgroup of $G$.
\end{proof}

\begin{theorem}
    The intersection of two subgroups is a subgroup.
\end{theorem}
\begin{proof}
    Let $H$ and $K$ be subgroups of $G$.
    Consider $x,y \in H \cap K$.
    \autoref{thm:xyiv_subgroup} shows that
    $xy^{-1} \in H$ and $xy^{-1} \in K$,
    so $xy^{-1} \in H \cap K$.
    Since the identity is in every subgroup,
    $H \cap K$ is nonempty and thus a subgroup of $G$.
\end{proof}

\begin{theorem}
    \label{thm:finite_xy_subgroup}
    A finite nonempty subset $H$ of a group $G$ is a subgroup
    iff $\forall x,y \in H, xy \in H$.
\end{theorem}
\begin{proof}
    If $H$ is a subgroup, $xy \in H$ by closure.
    Assume $\forall x,y \in H, xy \in H$. 
    Clearly, $\forall x \in H, \forall m \ge 0, x^{m} \in H$,
    Consider any element $x \in H$ and its order in $G$.
    If it was infinite, all $x^{n}$ for $n \ge 0$ are distinct and in $H$, so $H$ cannot be finite.
    Thus, the order of $x$ must be finite. Let it be $n$.
    Inverse is given by $x^{-1} = x^{n - 1} \in H$, since $x^{n-1}x = e = xx^{n - 1}$.
    Thus, $\forall x,y \in H, xy^{-1} \in H$, and by \autoref{thm:xyiv_subgroup}, $H$ is a subgroup.
\end{proof}

\begin{theorem}
    The set of elements of finite order of an abelian group forms a subgroup.
\end{theorem}
\begin{proof}
    Let $G$ be an abelian group and $H$ the set of elements of finite order.
    For any $x \in H$, let its order be $n$.
    Since $x^{-1} = x^{n - 1}$, we have $(x^{-1})^{n} = x^{n(n - 1)} = e^{n - 1} = e$.
    Thus, $x^{-1} \in H$, showing inverse.
    Now consider $y \in H$, and let $m$ be the order of $y^{-1}$.
    We have $(xy^{-1})^{nm} = x^{nm}(y^{-1})^{nm} = e^{m}e^{n} = e$,
    so $xy^{-1}$ has finite order and $xy^{-1} \in H$.
    Thus, by \autoref{thm:xyiv_subgroup}, $H$ is a subgroup of $G$.
\end{proof}

\section{Permutations}

\begin{definition}
    A \emph{permutation} of a set $X$ is an bijection from $X$ to $X$ (an automorphism of $X$).
    $S_X$ denotes the set of permutations of $X$.
\end{definition}

\begin{theorem}
    $S_X$ forms a group under function composition.
\end{theorem}
\begin{proof}
    Identity is given by $e(x) = x$.
    Closure is given by $fg(x) = f(g(x)) \in X$.
    Inverse is given by $f^{-1}(x)$, since each $f$ is a bijection.
    Associativity is given by the nature of function composition.
\end{proof}

\begin{definition}
    $S_X$ when $X$ is composed of the first $n$ positive integers is written $S_n$, and denoted $S_n$.
    This is called the \emph{symmetric group} of degree $n$.
\end{definition}
\begin{definition}
    $(a_{1}a_{2}a_{3} \cdots a_n)$ denotes a \emph{cyclic} permutation, where
    $f(a_{1}) = a_{2}$, $f(a_{2}) = a_{3}$, and so on, 
    with $f(a_n) = a_{1}$, and all other elements left unchanged.
    The length of the cyclic permutation is $n$, 
    and a cyclic permutation of length $k$ can be referred to as a $k$-cycle.
    A $2$-cycle is also called a \emph{transposition}.
\end{definition}
\begin{definition}
    Two cyclic permutations $(a_{1}a_{2} \cdots a_n)$ and $(b_{1}b_{2} \cdots b_n)$
    are \emph{disjoint} if there doesn't exist $i,j$ such that $a_i = b_j$.
\end{definition}

Note that disjoint cyclic permutations commute with eachother.

\begin{theorem}
    \label{thm:disjoint_cycles}
    Every element of $S_n$ can be written as a product of disjoint cyclic permutations.
\end{theorem}
\begin{proof}
    For some $\alpha \in S_n$ that is not the identity, consider the following procedure:
    Find an element $x_{1}$ such that $\alpha(x) \ne x_{1}$.
    Repeatedly apply $\alpha$ to $x_{1}$, producing $x_{2},x_{3},\cdots$ until an element that has already been seen appears.
    We write $\sigma_{1} = (x_{1}x_{2}\cdots)$.
    We have $\alpha = \sigma_{1}(\sigma_{1}^{-1}\alpha)$.
    Repeat this procedure on $\sigma_{1}^{-1}\alpha$, producing $\sigma_{2},\sigma_{3},\cdots$.
    The procedure terminates when $\cdots\sigma_{2}^{-1}\sigma_{1}^{-1}\alpha = e$, so that no $x_{1}$ can be found.
    In this case, $\alpha = \sigma_{1}\sigma_{2}\cdots$, which is a product of disjoint cyclic permutations.
    For every $x_i$, $\sigma_{1}^{-1}\alpha(x_i) = \sigma_{1}^{-1}(\alpha(x_i)) = \sigma_{1}^{-1}(x_{i+1}) = x_i$.
    Thus, the number of $x$ such that $\alpha(x) = x$ has decreased, and the procedure must terminate.
\end{proof}

\begin{theorem}
    The transpositions in $S_n$ generate $S_n$.
\end{theorem}
\begin{proof}
    Any cyclic permutation can be written as a product of transpositions, 
    namely that of $(a_{1}a_{2} \cdots a_k) = (a_{1}a_{k})\cdots(a_{1}a_{3})(a_{1}a_{2})$.
    Applying \autoref{thm:disjoint_cycles} proves the statement.
\end{proof}

\begin{theorem}
    $S_n$ can also be generated by any of the following set:
    \begin{compactenum}
        \item $(12),(13) \cdots (1n)$
        \item $(12),(23) \cdots ((n-1)n)$
        \item $(12 \cdots n), (12)$
    \end{compactenum}
\end{theorem}
\begin{proof}
    This is trivially done constructively, so I will be ommiting it here.
\end{proof}

\begin{definition}
    An element of $S_n$ that can be written as a product of an even number of transpositions is a \emph{even permutation}.
    Otherwise, it is an \emph{odd permutation}.
\end{definition}

We consider a quantity for permutations. Take $\alpha \in S_n$.
For every $1 \le i < j \le n$, examine the result of applying $S_n$ to the ascending list of integers.
If $j$ occurs before $i$ does in the resulting order, add $1$ to the total.
This quantity counts the number of out-of-order pairs.

Consider the difference between the quantity measured for some $\alpha \in S_n$ and $(ij)\alpha$, where $i < j$.
An extra transposition has been applied after $\alpha$.
It is clear that the pair $i,j$ has been flipped, but which others?
Every element between the positions of $x_i$ and $x_j$, say $x_m$, has had the pairs $m,i$ and $m,j$ flipped.
All other pairs have not been affected by this operation.
Every flip contributes to a $+1$ or $-1$ on the quantity, which \emph{flips the parity}.
Since every element between the positions contributes two partiy flips, no change to parity happened.
Thus, \emph{The parity of the quantity is flipped by the introduction of a transposition on any permutation}.

The identity has this quantity even. Thus, every even permutation has a even quantity.

\begin{theorem}
    If a permutation is even, it cannot be written as a product of an odd number of transpositions.
    Similarly, odd permutations cannot be written as product of even number of transpositions.
\end{theorem}

\begin{definition}
    The even permutations form a subgroup of $S_n$ with order $n!/2$.
    This is denoted $A_n$, the alternating group of degree $n$.
\end{definition}
\begin{proof}
    For any $x,y \in A_n$, write both as a product of a even number of transpositions.
    Clearly, $xy$ has number of transpositions equal to the sum of the number of transpositions in $x$ and $y$.
    Thus, $xy \in A_n$. Since $A_n$ is a subset of $S_n$ and $S_n$ is finite, $A_n$ is finite.
    By \autoref{thm:finite_xy_subgroup}, $A_n$ is a subgroup.

    Consider $f(\alpha) = (12)\alpha$, as $f:A_n \to S_n - A_n$.
    It is an injection, since $f(\alpha) = f(\beta)$ implies $(12)(12)\alpha = (12)(12)\beta$, so $\alpha = \beta$,
    It is a surjection, since for any odd permutation $\alpha$, $(12)\alpha$ is even and
    $f((12)\alpha) = (12)(12)\alpha = \alpha$.
    Thus, $f$ is a bijection. Thus, $|A_n| = |S_n - A_n|$, so $|A_n| = |S_n|/2 = n!/2$.
\end{proof}

\begin{theorem}
    The 3-cycles generate $A_n$ if $n > 3$.
\end{theorem}
\begin{proof}
    It is possible to write any permutation as a product of transpositions of the form $(1i)$.
    Since $(1i)$ is a transposition, any even permutation must be written in an even number of them.
    Since $(1a)(1b) = (1ab)$, by pairing consecutive pairs of transpositions, 
    any even permutation can be written as a product of 3-cycles.
\end{proof}

\begin{theorem}
    If $H$ is a subgroup of $S_n$ and not contained in $A_n$,
    exactly half of $H$ must be even permutations.
\end{theorem}
\begin{proof}
    Since $H$ is not contained in $A_n$, there exists odd $h \in H$.
    Consider $f : H \cap A_n \to H \cap (S_n - A_n)$, defined as $f(\alpha) = h\alpha$.
    It is a bijection since $f^{-1}(\alpha) = h^{-1}\alpha$ exists.
    Thus, $H$ must be partitioned equally into two parts, one even, one odd.
\end{proof}

\section{Isomorphisms}

\begin{definition}
    Two groups $G$ and $G'$ are \emph{isomorphic} iff there exists an bijection $\phi : G \to G'$
    such that $\phi(xy) = \phi(x)\phi(y)$ for all $x,y \in G$. This is denoted $G \cong G'$.
    $\phi$ is an \emph{isomorphism} between $G$ and $G'$.
\end{definition}

Note that the definition works equally well when $G$ and $G'$ are swapped with $\phi^{-1} : G' \to G$.
Thus, certain arguments will only be made one-way, and the symmetric property of this definition
will show the opposite direction. 
For the next section, $G$, $G'$, and $\phi$ assume the roles in the above definition.

\begin{theorem}
    An isomorphism sends the identity of one to that of the other.
\end{theorem}
\begin{proof}
    $\phi(e) = \phi(e)\phi(e)\phi(e)^{-1} = \phi(ee)\phi(e)^{-1} = \phi(e)\phi(e)^{-1} = e'$
\end{proof}

\begin{theorem}
    An isomorphism sends inverses to inverses in that $\phi(x^{-1}) = \phi(x)^{-1}$.
\end{theorem}
\begin{proof}
    $\phi(x^{-1}) = \phi(x^{-1})\phi(x)\phi(x)^{-1} = \phi(x^{-1}x)\phi(x)^{-1} = \phi(e)\phi(x)^{-1} = \phi(x)^{-1}$
\end{proof}

\begin{theorem}
    $G$ is abelian iff $G'$ is abelian.
\end{theorem}
\begin{proof}
    $\phi(x)\phi(y) = \phi(xy) = \phi(yx) = \phi(y)\phi(x)$
\end{proof}

\begin{theorem}
    $H < G \iff \phi(H) < G'$
\end{theorem}
\begin{proof}
    $\phi(H)$ is nonempty as it contains $\phi(e)$.
    For any two elements $\phi(x),\phi(y) \in \phi(H)$,
    $\phi(x)\phi(y)^{-1} = \phi(x)\phi(y^{-1}) = \phi(xy^{-1}) \in \phi(H)$,
    by \autoref{thm:xyiv_subgroup} $\phi(H) < G'$.
\end{proof}

\begin{theorem}
    An isomorphism preserves the order of each element.
\end{theorem}
\begin{proof}
    The order of an element is the order of the subgroup it generates.
    For any $x \in G$, define $H = \langle x\rangle$, and consider $H' = \langle\phi(x)\rangle$.
    An element of $H'$ will be of the form $h' = \phi(x)^{m} = \phi(x^{m}) \in \phi(H)$.
    This shows a bijection between $H'$ and $\phi(H)$, so $|H'| = |\phi(H)| = |H|$,
    order is preserved.
\end{proof}

\begin{theorem}
    The bijection $\phi(x) = x^{-1}$ forms an isomorphism from $G$ to $G$ (an automorphism of $G$) iff $G$ is abelian.
\end{theorem}
\begin{proof}
    If $G$ is abelian, $\phi(xy) = (xy)^{-1} = y^{-1}x^{-1} = x^{-1}y^{-1} = \phi(x)\phi(y)$, $\phi$ is an automorphism.
    If $\phi$ is an isomorphism, $\phi(xy) = y^{-1}x^{-1} = x^{-1}y^{-1} = (yx)^{-1} = \phi(yx)$,
    since $\phi$ is an isomorphism, $xy = yx$, $G$ is abelian.
\end{proof}

\begin{theorem}
    If $G$ is cyclic, $x$ generates $G$, and $\phi$ is an automorphism of $G$,
    then $\phi$ is completely determined by $\phi(x)$ and $\phi(x)$ generates $G$.
\end{theorem}
\begin{proof}
    If $x$ generates $G$, $G = \langle x\rangle \implies G \subseteq \langle x\rangle$.
    Thus, every $g \in G$ can be written as $g = x^{m}$ where $m \in \mathbb{Z}$.
    The corresponding element in $G$ is $\phi(x^{m}) = \phi(x)^{m} \in \langle\phi(x)\rangle$.
    Thus, $\phi(x)$ generates $G$.
    If we know the value of $\phi(x)$, we know $\phi(x^{-1}) = \phi(x)^{-1}$, and $\phi(e) = e$.
    Then, by $\phi(x^{m}) = \phi(x)\phi(x^{m - 1}) = \phi(x^{-1})\phi(x^{m + 1})$,
    it is possible to deduce the value of all $\phi(x^{m})$ where $m \in \mathbb{Z}$, which is all of $G$.
    Thus, $\phi$ is completely determined by $\phi(x)$.
\end{proof}

\section{Plato's Solids and Cayley's Theorem}

\begin{theorem}[Cayley's Theorem]
    \label{thm:cayley}
    Every group $G$ is isomorphic to a subgroup of $S_G$.
\end{theorem}
\begin{proof}
    For every element $g \in G$, consider a permutation $L_g : G \to G$ defined as $L_g(x) = gx$.
    It is clearly a bijection with $L_{g}^{-1}(x) = L_{g^{-1}}(x) = g^{-1}x$.
    We claim that this set of permutations is a subgroup of $S_G$.
    It is nonempty as it includes $f(x) = ex = x$ with $g = e$.
    For any two $g,h$, we have $L_g(L_h(x)) = ghx = L_{gh}(x)$, so it is closed under function composition.
    By \autoref{thm:finite_xy_subgroup}, it is a subgroup of $S_G$, and it is clearly isomorphic to $G$, with $\phi(g) = L_g$.
\end{proof}

\begin{theorem}
    Every finite group of order $n$ is isomorphic to a subgroup of $S_n$.
\end{theorem}
\begin{proof}
    $S_n$ is trivially isomorphic to $S_G$. Let $\psi$ be that isomorphisms. Isomorphisms send subgroups to subgroups.
    Let $G'$ be the group produced by the procedure in \autoref{thm:cayley},
    and let $G'' = \psi(G')$. Thus, we have $G \cong G' \cong G''$.
\end{proof}

\section{Matrix Groups}

The original book takes the convention of row vectors. 
This means that the linear transformation $f_A$ corresponding to $A$
is defined to be $f_A(\mathbf{x}) = \mathbf{x}A^{t}$.
This replica will do the same for now.

\begin{definition}
    The set of $n \times n$ invertible matrices of real numbers forms a group under matrix multiplication,
    and it written $GL_n$ (optionally $GL_n(\mathbb{R})$), the \emph{General Linear Group}.
    $GL_n(\mathbb{C})$ similarly denotes the group of $n \times n$ invertible complex matrices.
\end{definition}
\begin{proof}
    Matrix multiplication is trivially associative, the identity is $I_n$, 
    and closure is given by the fact that 
    products of invertible matrices are invertible, by $(AB)^{-1} = B^{-1}A^{-1} \in GL_n$.
    By definition, an inverse exists, so $GL_n$ forms a group, similarly so does $GL_n(\mathbb{C})$.
\end{proof}

\begin{theorem}
    $GL_n$ is isomorphic to a subgroup of $GL_{n+1}$.
\end{theorem}
\begin{proof}
    For every linear transformation $A \in GL_n$,
    consider the $n+1$ dimensional linear transformation $B \in GL_{n+1}$ that leaves the last coordinate fixed,
    and applies $A$ to all other dimensions. In matrix form, this is written
    \[B = \begin{bmatrix}
        A & 0 \\
        0 & 1 \\
    \end{bmatrix}\]
    It is obvious that this is an isomorphism between $GL_n$ and a subgroup of $GL_{n+1}$.
\end{proof}

\begin{definition}
    \label{def:ortho_group}
    The subset of $GL_n$ of orthogonal matrices forms a subgroup denoted $O_n$, the \emph{Orthogonal Group}.
    The subset of $O_n$ with determinant $+1$ forms a subgroup denoted $SO_n$, the \emph{Special Orthogonal Group}.
\end{definition}
\begin{proof}
    An orthogonal matrix is such that $A^{t}A = I.$
    $O_n$ is nonempty since $I$ is orthogonal.
    Notice that $(AB^{-1})^{t}(AB^{-1}) = (B^{-1})^{t}A^{t}AB^{-1} = (B^{t})^{t}B^{-1} = BB^{-1} = I$.
    Thus, by \autoref{thm:xyiv_subgroup}, $O_n$ is a subgroup of $GL_n$.
    $SO_n$ is nonempty since $I$ is special orthogonal.
    For any $A,B \in SO_n$, $\det(AB^{-1}) = 1\times1^{-1} = 1$, so $AB^{-1} \in SO_n$.
    Thus, $SO_n$ is a subgroup of $O_n$ and $GL_n$.
\end{proof}

\begin{theorem}
    For any $A \in O_n$, the linear transformation $f_A$ preserves distance and orthogonality.
\end{theorem}
\begin{proof}
    Take two vectors $\mathbf{x},\mathbf{y}$.
    $f_A(\mathbf{x}) \cdot f_A(\mathbf{y}) 
    = (\mathbf{x}A^{t})(\mathbf{y}A^{t})^{t}
    = \mathbf{x}A^{t}A\mathbf{y}^{t}
    = \mathbf{x}\mathbf{y}^{t}
    = \mathbf{x}\cdot\mathbf{y} $.
    Thus, orthogonality is preserved as a special case of $\mathbf{x}\cdot\mathbf{y} = 0$,
    and distance is preserved as a special case of $|\mathbf{x}-\mathbf{y}| = \sqrt{(\mathbf{x}-\mathbf{y})\cdot(\mathbf{x}-\mathbf{y})}$.
\end{proof}

\begin{theorem}
    Any element of $O_2$ is either a rotation or a reflection,
    and any element of $SO_2$ is a rotation.
\end{theorem}
\begin{proof}
    The columns of any matrix in $O_2$ are two orthonormal vectors.
    Let $(\cos\theta, \sin\theta)$ be one of the vectors.
    Then, the other vector is either $(-\sin\theta, \cos\theta)$,  or $(\sin\theta, -\cos\theta)$.
    In the former case, we have
    \[\begin{bmatrix}
        \cos\theta & -\sin\theta \\
        \sin\theta & \cos\theta \\
    \end{bmatrix},\]
    which represents an anticlockwise rotation by $\theta$ around the origin, and is also an element of $SO_2$.
    In the latter case, we have
    \[\begin{bmatrix}
        \cos\theta & \sin\theta \\
        \sin\theta & -\cos\theta \\
    \end{bmatrix},\]
    which is a reflection around the line with angle $\theta/2$.
    Thus, we have exhausted all possibles members of $O_2$ and $SO_2$.
\end{proof}

\begin{theorem}
    Any element of $SO_3$ is a rotation of an angle around an axis which passes through the origin.
\end{theorem}
\begin{proof}
    Unfortunately, I do not appear to have the linear algebra prowess for this proof.
    Thus, this will simply be accepted as truth for now.
\end{proof}

\begin{definition}
    The subset of $GL_n(\mathbb{C})$ of unitary matrices forms a subgroup denoted $U_n$, the \emph{Unitary Group}.
    The subset of $O_n$ with determinant $+1$ forms a subgroup denoted $SU_n$, the \emph{Special Unitary Group}.
\end{definition}
\begin{proof}
    It's largely identical to \autoref{def:ortho_group}, and will be left as an exercise to the reader.
\end{proof}

\begin{definition}
    The set of real matrices formed by orthogonal matrices with integer entries
    is a subgroup of $GL_n(\mathbb{R})$, and is denoted $GL_n(\mathbb{Z})$.
\end{definition}
\begin{proof}
    The identity of $GL_n(\mathbb{Z})$ is $I$, which has determinant 1 and only integer entries.
    It is obvious that the product of two integer matrices is another integer matrix.
    The inverse of a integer matrix with determinant $\pm 1$ is an integer matrix,
    as shown constructively by the method of inverting arbitrary matrices using cofactors.
    Thus, $GL_n(\mathbb{Z})$ is a subgroup of $GL_n(\mathbb{R})$ and also $O_n$.
\end{proof}

\section{Products}

\begin{definition}
    A \emph{direct product} of $G$ and $H$, written $G \times H$, is defined to be 
    the group formed by $\{(g,h) \mid g \in G, h \in H\}$ under multiplication per element.
\end{definition}
\begin{proof}
    The identity is $(e_g,e_h)$, and the inverse of $(g,h)$ is $(g^{-1},h^{-1})$..
    The product is $(gg',hh')$ and since $gg' \in G$ and $hh' \in H$, $(gg',hh') \in G \times H$.
    It is associative by the associativity of the product of $G$ and $H$.
\end{proof}

\begin{theorem}
    \label{thm:GxH=HxG}
    For any two groups $G,H$, $G \times H$ is isomorphic to $H \times G$.
\end{theorem}
\begin{proof}
    The bijection $\phi : G \times H \to H \times G$ defined by $\phi((g,h)) = (h,g)$ is an isomorphism between $G \times H$ and $H \times G$,
    $\phi((g,h))\phi((g',h')) = (hh',gg') = \phi((gg',hh')) = \phi((g,h)(g',h'))$.
\end{proof}

\begin{theorem}
    The order of $G \times H$ is infinite if either $G$ or $H$ have infinite order,
    and equal to the product of their orders otherwise.
\end{theorem}
\begin{proof}
    The map defined by $g \mapsto (g,e)$ is clearly an injection, 
    so $G \times H$ must have at least as many elements as $G$.
    Thus, if $G$ is infinite, so is $G \times H$.
    By \autoref{thm:GxH=HxG}, this also applies to $H$.
    For finite cases, it should be clear that for every $g$, 
    there are $|H|$ distinct pairs $(g,h)$, which makes $|G| \times |H|$ distinct values in total.
\end{proof}

\begin{theorem}
    $G$ and $H$ are both abelian iff $G \times H$ is abelian.
\end{theorem}
\begin{proof}
    If $G$ and $H$ are both abelian, then clearly $G \times H$ is abelian by
    $(g,h)(g',h') = (gg',hh') = (g'g,h'h) = (g',h')(g,h)$.
    If $G \times H$ is abelian, consider the subgroup $(g,e)$ and its isomorphism to $G$. 
    Since the subgroup of an abelian group is abelian, so is $G$.
    By \autoref{thm:GxH=HxG}, this also applies to H.
\end{proof}

\begin{definition}
    \emph{Klein's Group} is defined to be $\mathbb{Z}^{2} \times \mathbb{Z}^{2}$, 
    is also called the \emph{four group}, and is denoted $V$.
\end{definition}

\begin{theorem}
    $\mathbb{Z}_{m} \times \mathbb{Z}_{n}$ is a cyclic group iff $(1,1)$ generates it.
\end{theorem}
\begin{proof}
    If $(1,1)$ generates $\mathbb{Z}_m \times \mathbb{Z}_n$, then by definition it is cyclic.
    If $Z_{m} \times Z_{n}$ is a cyclic group,
    there exists an element $(a,b)$ which generates $\mathbb{Z}_m \times \mathbb{Z}_n$.
    $a$ must generate $\mathbb{Z}_m$ since if it isn't, 
    then there exists $x \in \mathbb{Z}_m$ such that no multiple of $a$ equals $x$.
    In that case $(x,0)$ cannot be generated by $(a,b)$.
    Similarly, $b$ generates $\mathbb{Z}_n$.
    Thus, the map $f((a^{x},b^{y})) = (x,y)$ is an automorphism of $\mathbb{Z}_m \times \mathbb{Z}_n$
    that sends $(a,b)$ to $(1,1)$, so $(1,1)$ generates $\mathbb{Z}_m \times \mathbb{Z}_n$.
\end{proof}

Examining the above proof, it is possible to generalize half of the argument to all groups.

\begin{corollary}
    If $G \times H$ is a cyclic group, then $G$ and $H$ are both cyclic.
\end{corollary}

\begin{theorem}
    $\mathbb{Z}_{m} \times \mathbb{Z}_{n}$ is a cyclic group iff $n$ is coprime to $m$.
\end{theorem}
\begin{proof}
    If $\mathbb{Z}_m \times \mathbb{Z}_n$ is a cyclic group, $(1,1)$ is a generator.
    Thus, for any element $(a,b)$ there exists $k$ such that
    $k = a \mod m$ and $k = b \mod n$.
    If there exists $q > 1$ such that $q \mid n$ and $q \mid m$,
    let $(a,b) = (0,1)$, then we have $k = 0 \mod q$ and $k = 1 \mod q$,
    which is a contradiction. Thus, no such $q$ exists and $n$ and $m$ are coprime.

    If $n$ is coprime to $m$, consider the elements $(k,k)$ generated by $(1,1)$.
    $(k_{1},k_{1})$ is equal to $(k_{2},k_{2})$ iff $k_{1} \equiv k_{2} \mod m$ and $k_{1} \equiv k_{2} \mod n$.
    This occurs iff $k_{1} \equiv k_{2} \mod nm$, since $n$ is coprime to $m$.
    Thus, there are $nm$ different values of $k$ that create distinct multiples of $(1,1)$,
    so the subgroup generated by $(1,1)$ fills $\mathbb{Z}_m \times \mathbb{Z}_n$, and thus it is cyclic.
\end{proof}

\begin{theorem}
    If $n$ is odd, $SO_n \times \{I,-I\}$ is isomorphic to $O_n$ by $\phi((A,U)) = AU$.
\end{theorem}
\begin{proof}
    $\phi$ preserves the algebraic structure, by
    $\phi((A,U))\phi((B,V)) = AUBV = ABUV = \phi((AB,UV)) = \phi((A,U)(B,V))$, 
    since $I,-I$ are both commutative with any element of $SO_n$.
    For any element $A$ of $O_n$, if $\det A = 1$, $A = AI$, and $A \in SO_{n}$;
    if $\det A = -1$, $A = (-A)(-I)$, and $\det(-A) = -1$, so $-A \in SO_{n}$.
    Thus, $\phi$ is a surjection.
    If $\phi((A,U)) = \phi((B,V))$, we have $AU = BV$.
    Since $\det A = \det B = 1$, we have $\det(AU) = \det U = \det V = \det(BV)$, 
    and since $I,-I$ are the only values and have distinct determinants, $U = V$.
    Since U = V, dividing both sides yields $A = B$, so $\phi$ is an injection.
    Thus, $\phi$ is a bijection and thus an isomorphism.
\end{proof}

\begin{corollary}
    If $n$ is odd, $SO_n \times \mathbb{Z}_2 \cong O_n$.
\end{corollary}

The reason that theorem doesn't work for even $n$ is that $\det(-A) = (-1)^{n}\det A = \det A$ for even $n$.

\begin{theorem}
    \label{thm:isomorphic_product_of_subgroups}
    If $H,K$ are subgroups of $G$, and the following conditions hold:
    \begin{compactitem}
    \item $HK = G$;
    \item $e$ is the only element in $H \cap K$;
    \item $xy = yx$ for all $x \in H, y \in K$;
    \end{compactitem}
    then $H \times K \cong G$.
\end{theorem}
\begin{proof}
    Consider the isomorphism $\phi : H \times K \to G$ defined by $\phi((x,y)) = xy$.
    Since $HK = G$, for any $g \in G$ there exists $xy = G$, so $\phi$ is a surjection.
    If $\phi((x,y))=\phi((x',y'))$, we have $xy = x'y'$, so $x = x'(y'y^{-1})$.
    Since $y'y^{-1} \in K$, and $x'(y'y^{-1}) = x \in H$, 
    $y'y^{-1} = e$ since $e$ is the only common element, so $y' = y$.
    Similarly, $x' = x$, so $\phi$ is a bijection.
    $\phi((x,y))\phi((x',y')) = xyx'y' = xx'yy' = (xx')(yy') = \phi((xx',yy')) = \phi((x,y)(x',y'))$,
    so $\phi$ preserves algebraic properties, and is an isomorphism.
\end{proof}

\begin{lemma}
    If a group $G$ has every element other the identity at order 2,
    it is abelian.
\end{lemma}
\begin{proof}
    For every element, if it is the identity, $e = e^{-1}$. 
    Otherwise, order 2 implies $g^{2} = e$, so $g = g^{-1}$.
    For any $a,b \in G$, since $ab \in G$, $ab = (ab)^{-1} = b^{-1}a^{-1} = ba$.
\end{proof}

\begin{theorem}
    If a finite group $G$ has every element other the identity at order 2,
    it is isomorphic to a power of $\mathbb{Z}_2$.
\end{theorem}
\begin{proof}
    For every element $g \in G$, since $g^{2} = e$, $\{e,g\}$ forms a subgroup.
    Every one of these subgroups is isomorphic to $\mathbb{Z}_{2}$, by $e \mapsto 0, g \mapsto 1$.
    Start by picking two distinct elements $g_{1},g_{2}$ of $G$ that aren't $e$.
    Then, repeated select an element that doesn't belong to the subgroup generated by the currently selected elements.
    Since $G$ is finite, the procedure terminates.
    These elements $g_{1},g_{2},\cdots,g_n$ form a set of generators for the group. 

    This means that for any $g \in G$,
    there exists a word in the generators $g_i$ equal to $g$.
    However, since $G$ is abelian, it is possible to reorder the $x_i$ to merge equal elements.
    More precisely, there exists $m_{1},m_{2},\cdots,m_k \in \mathbb{Z}$,
    such that $g_{1}^{m_{1}}g_{2}^{m_{2}} \cdots g_{n}^{m_{n}} = g$.
    However, since the order of all $g_i$ is $2$, we have $m_i \in \mathbb{Z}_2$ instead.
    Thus, any $g \in G$ is expressible as the product of a subset of the generators $g_i$.

    Thus, $G_{2} = \langle g_{1},g_{2}\rangle$ contains the elements $\{e,g_{1},g_{2},g_{1}g_{2}\}$.
    Consider $H = \langle g_{1}\rangle$ and $K = \langle g_{2}\rangle$.
    It is clear that $HK = \langle g_{1},g_{2}\rangle$,
    $e$ is the only shared element since $H = \{e,g_{1}\}$ and $K = \{e,g_{2}\}$,
    and since $G$ is abelian, so are its subgroups.
    Thus, \autoref{thm:isomorphic_product_of_subgroups} applies to this situation, and $G_{2} \cong H \times K \cong \mathbb{Z}_2 \times \mathbb{Z}_2$.
    
    Next, consider $G_{3} = \langle g_{1},g_{2},g_{3}\rangle$.
    By separating cases where the power of $g_{3}$ is $0$ and the cases which it is $1$,
    we can see that $G_{3}$ consists of the distinct subsets $G_{2}$ and $G_{2}g_{3}$.
    Let $K_{3} = \langle g_{3}\rangle = \{e,g_{3}\}$, and we see that $G_{2}K_{3} = G_{3}$.
    $e$ is the only element shared between $G_{2}$ and $K_{3}$, since $g_{3} \notin G_{2}$ by the selection procedure.
    Since the two sets trivially commute by the abelian property,
    \autoref{thm:isomorphic_product_of_subgroups} applies and $G_{3} \cong G_{2} \times K_{3} \cong \mathbb{Z}_2^{2} \times \mathbb{Z}_2$

    By repeating this procedure $n$ times, we have $G = G_n \cong \mathbb{Z}_2^{n}$.
\end{proof}

\section {Lagrange's Theorem}

\begin{theorem}[Lagrange's Theorem]
    If $H$ is a subgroup of a finite group $G$, the order of $H$ divides that of $G$.
\end{theorem}
\begin{proof}
    If $H = G$, then the theorem holds trivially.
    Otherwise, select an element $g_{1} \in G - H$, and consider $g_{1}H = \{g_{1}h \mid h \in H\}$.
    We claim $g_{1}H$ is disjoint from $H$.
    If an element $x \in H$ is also present in $g_{1}H$, there exists $y \in H$ such that $x = g_{1}y$.
    But that implies $g_{1} = xy^{-1} \in H$, which contradicts our assumption. Thus, $g_{1}H$ and $H$ are disjoint.

    The order of $g_{1}H$ is exactly equal to the order of $H$ by the bijection $h \mapsto g_{1}h$.
    Thus, if $g_{1}H \cup H = G$, the theorem holds with $2|H| = |G|$.
    Otherwise, it is possible to select $g_{2} \in G - (H + g_{1}H)$, and define $g_{2}H$.
    We claim $g_{2}H$ is disjoint from $g_{1}H$.
    If an element $g_{1}x$ is also present in $g_{2}H$, there exists $y \in H$ such that $g_{1}x = g_{2}y$.
    In that case, $g_{2} = g_{1}xy^{-1} = g_{1}(xy^{-1}) \in g_{1}H$, which contradicts the selection of $g_{2}$.
    Thus, $g_{2}H$ is wholly disjoint from $H$ or $g_{1}H$, and also of the same order as $H$.

    The procedure must terminate since the number of elements in $H + \sum g_i H$ continues to increase,
    and $G$ is a finite group.
    Thus, upon procedure termination, $k|H| = |G|$ and the theorem is proven.
\end{proof}

\begin{corollary}
    \label{thm:order_elem_divide_order_group}
    The order of any element divides the order of the group.
\end{corollary}
\begin{proof}
    Recall \autoref{thm:order_elem_order_subgroup}.
\end{proof}

\begin{corollary}
    Any prime order group must be cyclic.
\end{corollary}
\begin{proof}
    Using the previous theorem, the order of every element must be equal to the order of the group if it is prime, so it is cyclic.
\end{proof}

\begin{corollary}
    \label{thm:x^|G|=e}
    For any $x \in G$, $x^{|G|} = e$.
\end{corollary}
\begin{proof}
    Since the order of $x$ divides $G$, we have $k|x| = |G|$,
    and by the definition of order $x^{|x|} = e$.
    Thus, $x^{|G|} = x^{k|x|} = (x^{|x|})^{k} = e^{k} = e$.
\end{proof}

\begin{theorem}
    Let $R_n$ denote the set of all integers $1 \le m \le n - 1$ which are coprime to $n$.
    $R_n$ forms a group under multiplication modulo $n$.
\end{theorem}
\begin{proof}
    Since $1$ always satisfies the constraints, $1 \in R_n$ and it is the identity.
    Multiplication is of course naturally associative and commutative.
    For any two elements $x,y$, $xy \in R_n$ iff $xy$ mod $n$ is coprime to $n$.
    Since neither $x$ nor $y$ share any prime factors with $n$, $xy$ is coprime to $n$.
    If $xy$ mod $n$ isn't coprime to $n$, there exists $d$ such that $xy - kn \equiv 0 \mod d$ and $n \equiv 0 \mod d$.
    substituting $n \equiv 0$ yields $xy \equiv 0 \mod d$, and thus $xy$ shares a factor $d$ with $n$. This is a contradiction.
    Thus, $xy$ mod $n$ is coprime to $n$, and $R_n$ is closed under multiplication mod $n$.

    For any $m$, we know that there exists $x,y$ such that $xm + yn = 1$.
    Then, we have $xm = 1 - yn \equiv 1 \mod n$, so $x$ is an inverse for $m$.
    Since $xm \equiv 1 \mod n$, $xm$ is coprime to $n$. Thus, $x$ must be coprime to $n$.
    Thus, $R_n$ is a group. It is also clearly abelian.
    Its order is $\phi(n)$, which is Euler's totient function.
\end{proof}

\begin{corollary}[Euler's Theorem]
    If $x$ is coprime to $n$, $x^{\phi(n)} \equiv 1 \mod n$.
\end{corollary}
\begin{proof}
    Take $x$ modulo $n$ which is an element $m \in R_n$. By $m^{\phi(n)} \equiv 1 \mod n$.
    Thus, $x^{\phi(n)} \equiv m^{\phi(n)} \equiv 1 \mod n$.
\end{proof}

\begin{corollary}[Fermat's Little Theorem]
    If $p$ is prime and $x$ is not a multiple of $p$, then
    $x^{p-1} \equiv 1 \mod p$
\end{corollary}
\begin{proof}
    Note the special case of Euler's Theorem when $n$ is prime, $\phi(p) = p - 1$.
\end{proof}

\begin{theorem}
    Given $H < G$, we have $g_{1}H = g_{2}H \iff g_{1}^{-1}g_{2} \in H$.
\end{theorem}
\begin{proof}
    If $g_{1}^{-1}g_{2} \in H$, we have $(g_{1}^{-1}g_{2})^{-1} = g_{2}^{-1}g_{1} \in H$.
    For any element $g_{2}h$ of $g_{2}H$, note that
    $g_{2}h = g_{1}g_{1}^{-1}(g_{2}h) = g_{1}(g_{1}^{-1}g_{2}h) \in g_{1}H$,
    so $g_{2}H \subseteq g_{1}H$.
    By $g_{2}^{-1}g_{1} \in H$, we similarly have $g_{1}H \subseteq g_{2}H$, so $g_{1}H = g_{2}H$ as required.
    Conversely, if $g_{1}H = g_{2}H$, then
    for any $g_{2}h \in g_{2}H$, we have $h' \in H$ such that $g_{2}h = g_{1}h'$.
    Thus, $g_{1}^{-1}g_{2} = h'h^{-1} \in H$.
\end{proof}

\begin{theorem}
    Given subgroups $H,K$ of $G$, if $|H|$ is coprime to $|K|$,
    the identity element must be the only element they have in common.
\end{theorem}
\begin{proof}
    If there exists $g \in H \cap K$, 
    then its order must divide both $|H|$ and $|K|$ by \autoref{thm:order_elem_divide_order_group}.
    Thus, $g$ must be the identity, the only element with order $1$.
    Thus, the identity is the only element in common.
\end{proof}

\begin{theorem}
    Given finite subsets $X,Y$ of group $G$, if 
    $Y$ is a subgroup of $G$ and $XY \subseteq X$,
    then $|X|$ is a multiple of $|Y|$.
\end{theorem}
\begin{proof}
    TODO
\end{proof}

\end{document}


