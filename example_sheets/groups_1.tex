\documentclass[12pt]{article}
\usepackage[left = 1in, right = 1in, top = 1in, bottom = 1in]{geometry}
\usepackage{textcomp}
\usepackage{gensymb}
\usepackage{paralist}
\usepackage{cancel}
\usepackage{enumitem}
\usepackage{amsmath}
\usepackage{amssymb}
\usepackage{amsthm}
\usepackage{tkz-euclide}
\usepackage{hyperref}
\usepackage{esdiff}
\usepackage{parskip}
\usepackage{accents}
\usepackage{xcolor}

\usetikzlibrary{arrows.meta,positioning}

\newtheoremstyle{customstyle}
  {8pt} % Space above (adjust as needed)
  {0pt} % Space below (adjust as needed)
  {} % Body font
  {} % Indent amount
  {\bfseries} % Theorem head font
  {. } % Punctuation after theorem head
  {0pt} % Space after theorem head
  {} % Theorem head spec
\theoremstyle{customstyle}
\newtheorem{theorem}{Theorem}[section]
\newtheorem{exercise}{Exercise}[section]
\newtheorem{claim}[theorem]{Claim}
\newtheorem{prop}[theorem]{Proposition}
\newtheorem{corollary}[theorem]{Corollary}
\newtheorem{lemma}[theorem]{Lemma}
\newtheorem{definition}[theorem]{Definition}
\newtheorem{question}{Question}
\newtheorem{subquestion}{Part}[question]

\def\definitionautorefname{Definition}
\def\corollaryautorefname{Corollary}

\newenvironment{nonproof}{\par $\cancel {\text{\textit{Proof}}}.$}{\hfill$\cancel\square$}

\def\contra{\tikz[baseline, x=0.22em, y=0.22em, line width=0.032em]\draw (0,2.83)--(2.83,0) (0.71,3.54)--(3.54,0.71) (0,0.71)--(2.83,3.54) (0.71,0)--(3.54,2.83);}
\newenvironment{answer}{\par\noindent\textit{Answer.}}{\par}

\renewcommand{\CancelColor}{\color{red}}
\renewcommand{\Re}{\operatorname{Re}}
\renewcommand{\Im}{\operatorname{Im}}
\renewcommand{\bar}{\overline}
% \renewcommand{\vec}[1]{\undertilde{\mathrm{#1}}}
% \renewcommand{\vec}[1]{\undertilde{#1}}
% \renewcommand{\vec}[1]{\underline{#1}}
\renewcommand{\vec}[1]{\mathbf{#1}}
\newcommand{\unitvec}[1]{\hat{\vec{#1}}}
\newcommand{\sol}{$\operatorname{sol}^{\simeq}$}
\newcommand{\Arg}{\operatorname{Arg}}
\newcommand{\Log}{\operatorname{Log}}
\newcommand{\vecspan}{\operatorname{span}}
\newcommand{\id}[1]{\operatorname{id}_{#1}}
\newcommand{\indic}[1]{i_{#1}}
\newcommand{\Sym}{\operatorname{Sym}}
\newcommand{\Isom}{\operatorname{Isom}}
\newcommand{\di}{\mathrm{d}}
\newcommand{\gen}[1]{\langle{#1}\rangle}

\definecolor{applegreen}{rgb}{0.55, 0.71, 0.0}
\definecolor{ufogreen}{rgb}{0.24, 0.82, 0.44}


\begin{document}

\begin{question}
    Let group $G$, show that the identity $e$ is the only element $g \in G$
    such that $g^{2} = g$.
\end{question}
\begin{proof}
    Let $g \in G$ such that $g^{2} = g$. We have
    $g = (g^{-1}g)g = g^{-1}(gg) = g^{-1}g^{2} = g^{-1}g = e$.
\end{proof}

\begin{question}
    Let $H$ and $K$ be two subgroups of a group $G$.
    Show that $H \cap K$ is a subgroup of $G$.
\end{question}
\begin{proof} 
    Since $H,K$ are both subgroups, we have $e \in H$ and $e \in K$,
    so $e \in H \cap K$.
    Similarly, let $a \in H \cap K$, we have $a \in H$ and $a \in K$,
    and since $H,K$ are groups, $a^{-1} \in H$ and $a^{-1} \in K$,
    so $a^{-1} \in H \cap K$.
    Finally, let $a,b \in H \cap K$, then
    $ab \in H$ and $ab \in K$, so $ab \in H \cap K$.
\end{proof}

\begin{subquestion}
    Show that $H \cup K$ is a subgroup of $G$
    iff $H \subseteq K$ or $K \subseteq H$.
\end{subquestion}
\begin{proof}
    If $H \subseteq K$ or $K \subseteq H$,
    let $A$ be the larger subgroup, then clearly $H \cup K = A$ is a subgroup.

    If $H \cup K$ is a subgroup, 
    and $H \nsubseteq K$ and $K \nsubseteq H$,
    there exists both $a \in H \setminus K$ and $b \in K \setminus H$.
    Clearly we have $a,b \in H \cup K$.
    Then $ab \in H \cup K$, so $ab \in H$ or $ab \in K$.
    \item If $ab \in H$, then $b = (a^{-1}a)b = a^{-1}(ab) \in H$.
    \item Otherwise, $ab \in K$, then $a = a(bb^{-1}) = (ab)b^{-1} \in K$.
\end{proof}

\begin{question}
    Let $G = \mathbb{R} \setminus \{-1\}$, 
    and let $x * y = x + y + xy$, where $xy$ is usual product.
    Show that $(G,*,0)$ is a group.
\end{question}
\begin{proof}
    For identity, $0$ trivially satisfies $x * 0 = x + 0 + 0x = x$.
    For inverse, let $x^{-1} = -\dfrac{x}{x+1}$. Then we have
    \begin{align*}
        x * x^{-1} &= x - \frac{x}{x + 1} - \frac{x^{2}}{x + 1} \\
                       &= \frac{x(x+1) - x - x^{2}}{x+1} \\
                       &= \frac{x^{2} + x - x - x^{2}}{x + 1}\\
                       &= 0.
    \end{align*}
    Thus, inverse is satisfied. Let $a,b,c \in \mathbb{R} \setminus \{-1\}$, then we have
    \begin{align*}
        a * b * c &= (a + b + ab) * c\\
                  &= (a + b + ab) + c + (a + b + ab)c\\
                  &= a + b + c + ab + ac + bc + abc\\
                  &= a + (b + c + bc) + a(b + c + bc)\\
                  &= a * (b + c + bc)\\
                  &= a * (b * c).
    \end{align*}
    Thus, associativity is satisfied. Last, closure is needed.
    Assume there exists $a,b \in \mathbb{R} \setminus \{-1\}$ such that $a*b = -1$, and we have
    \begin{align*}
        a*b &= a + b + ab\\
            &= (a + 1)(b + 1) - 1
    \end{align*}
    substituting we have $(a + 1)(b + 1) = 0$,
    so we must have $a = -1$ or $b = -1$, which is a contradiction.
    Thus, no such $a,b$ exists, and closure is satisfied.
\end{proof}

\begin{question}
    Let $G$ be a finite group.
    Let $g \in G$, show that there exists $n \in \mathbb{N}$ such that $g^{n} = e$.
\end{question}
\begin{proof}
    Let $a > b$ both naturals, and assume $g^{a} = g^{b}$.
    Then, we have by right multiplying by $g^{-b}$ that $g^{a - b} = e$.
    Since $a > b$, we have $n = a - b$.
    Otherwise, if no such $a,b$ exists, then all $g^{a}$ where $a \in \mathbb{N}$
    are distinct. 
    That makes $\{g^{a} \mid a \in \mathbb{N}\}$ an infinite set,
    but $g^{a} \in G$ for all $a \in \mathbb{N}$, 
    so an infinite set is a subset of a finite one, which is a contradiction.
\end{proof}

\begin{subquestion}
    Show that there exists $N \in \mathbb{N}$ such that $g^{N} = e$
    for all $g \in G$.
\end{subquestion}
\begin{proof}
    Take $N = \prod\limits_{g \in G} \text{order of }g$.
    Thus, for any $g \in G$, let $n$ be the order of $g$,
    there exists $k$ such that $N = nk$.
    Then $g^{N} = g^{nk} = (g^{n})^{k} = e^{k} = e$.
\end{proof}

\begin{question}
    Let $S$ be a finite non-empty set of non-zero complex number
    which are closed under multiplication.
    Show that $S$ is a subset of the set $\{z \in \mathbb{C} : |z| = 1\}$.
\end{question}
\begin{proof}
    Let arbitrary $z \in S$. Assume $|z| \ne 1$.
    Since $z \ne 0$, we have $|z| > 0$.
    If $0 < |z| < 1$, 
    we have $|z|^{n} \ne |z|^{m}$ if $n \ne m$,
    so $z^{n} \ne z^{m}$ if $n \ne m$,
    so the set $\{z^{n} \mid n \in \mathbb{N}\}$
    is infinite, but also a subset of finite set $S$ 
    since $S$ is closed under multiplication,
    which is a contradiction.
    Similarly, if $|z| > 1$, the same argument holds.
    Thus, $|z| = 1$ for all $z \in S$.
\end{proof}

\begin{subquestion}
    Show that $S$ is a group with respect to multiplication.
\end{subquestion}
\begin{proof}
    Closure and associativity are trivially given.

\end{proof}

\end{document}
