\documentclass[12pt]{article}
\usepackage[left = 1in, right = 1in, top = 1in, bottom = 1in]{geometry}
\usepackage{paralist}
\usepackage{cancel}
\usepackage{enumitem}
\usepackage{amsmath}
\usepackage{amssymb}
\usepackage{amsthm}
\usepackage{tkz-euclide}
\usepackage{hyperref}
\usepackage{esdiff}
\usepackage{parskip}
\usepackage{accents}
\usepackage{xcolor}

\usetikzlibrary{arrows.meta,positioning}

\newtheoremstyle{customstyle}
  {8pt} % Space above (adjust as needed)
  {0} % Space below (adjust as needed)
  {} % Body font
  {} % Indent amount
  {\bfseries} % Theorem head font
  {. } % Punctuation after theorem head
  {0pt} % Space after theorem head
  {} % Theorem head spec
\theoremstyle{customstyle}
\newtheorem{theorem}{Theorem}[section]
\newtheorem{corollary}{Corollary}[theorem]
\newtheorem{lemma}{Lemma}[section]
\newtheorem{definition}{Definition}[section]

\def\definitionautorefname{Definition}
\def\corollaryautorefname{Corollary}

\newenvironment{nonproof}{\par \textit{Nonproof:}}{\hfill$\cancel\square$}

\def\contra{\tikz[baseline, x=0.22em, y=0.22em, line width=0.032em]\draw (0,2.83)--(2.83,0) (0.71,3.54)--(3.54,0.71) (0,0.71)--(2.83,3.54) (0.71,0)--(3.54,2.83);}

\renewcommand{\Re}{\operatorname{Re}}
\renewcommand{\Im}{\operatorname{Im}}
\renewcommand{\bar}{\overline}


\begin{document}
\begin{question}
    If $n^{2}$ is a multiple of $3$, must $n$ be a multiple of $3$?
\end{question}
\begin{answer}
    Yes.
    The statement can be rephrased as $3 \mid n^{2} \implies 3 \mid n$.
    We will prove the contrapositive, $3 \nmid n \implies 3 \nmid n^{2}$.
    Let $n$ such that $3 \nmid n$, we have $n = 3k + r$
    where $k \in \mathbb{N}_0$ and $r \in \{1,2\}$.
    Then we have
    \begin{align*}
        n^{2} &= (3k + r)^{2}\\
              &= 9k^{2} + 6kr + r^{2}\\
              &= 3(3k^{2} + 2kr) + r^{2}
    \end{align*}
    Since $r = 1$ or $r = 2$, we have $r^{2} = 1$ or $r^{2} = 4$,
    neither of which are a multiple of $3$.
    Thus, $n^{2}$ is not a multiple of $3$.
\end{answer}


\begin{question}
    Consider the sequence $41,43,47,53,61,\cdots$.
    Are all these numbers prime?
\end{question}
\begin{answer}
    No.
    If the starting term is $x_{0}$, The sequence is $x_n = n^{2} + n + 41$.
    Thus, $x_{41} = 41^{2} + 41 + 41$ is divisible by $41$.
\end{answer}

\begin{question}
    
\end{question}

\begin{question}
    Suppose we have some positive integers (not necessarily distinct)
    whos sum is $100$. How large can their product be?
\end{question}

\begin{answer}
    If we consider the function
    \[
        f(x) = x^{100/x} = \exp(100x^{-1}\ln x),
    \]
    then we have
    \begin{align*}
        f'(x) &= 100(-x^{-2}\ln x + x^{-2})\exp(100x^{-1}\ln x)\\
                &= 100x^{-2}(1-\ln x)\exp(100x^{-1}\ln x)
    \end{align*}
    Solving for zeros, we have $\ln x = 1$, so $x = e$.

    $(k+a)(k-a) = k^{2} - a^{2}$. 
    For two numbers,
    the lower their difference $2a$, the higher their product.

    Consider $(k+a)(k+b)(k+c)$ where $a+b+c=0$,
    we have
    \begin{align*}
        (k+a)(k+b)(k+c) &= (k^{2} + (a+b)k + ab)(k + c)\\
                        &= k^{3} + (a+b+c)k^{2} + (ab + bc + ac)k + abc\\
                        &= k^{3} + 1/2((a+b+c)^{2} - a^{2} - b^{2} - c^{2})k + abc\\
                        &= k^{3} - 1/2(a^{2} + b^{2} + c^{2})k + abc
    \end{align*}

\end{answer}

\end{document}
