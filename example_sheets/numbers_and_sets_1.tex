\documentclass[12pt]{article}
\usepackage[left = 1in, right = 1in, top = 1in, bottom = 1in]{geometry}
\usepackage{paralist}
\usepackage{cancel}
\usepackage{enumitem}
\usepackage{amsmath}
\usepackage{amssymb}
\usepackage{amsthm}
\usepackage{tkz-euclide}
\usepackage{hyperref}
\usepackage{esdiff}
\usepackage{parskip}
\usepackage{accents}
\usepackage{xcolor}

\usetikzlibrary{arrows.meta,positioning}

\newtheoremstyle{customstyle}
  {8pt} % Space above (adjust as needed)
  {0} % Space below (adjust as needed)
  {} % Body font
  {} % Indent amount
  {\bfseries} % Theorem head font
  {. } % Punctuation after theorem head
  {0pt} % Space after theorem head
  {} % Theorem head spec
\theoremstyle{customstyle}
\newtheorem{theorem}{Theorem}[section]
\newtheorem{corollary}{Corollary}[theorem]
\newtheorem{lemma}{Lemma}[section]
\newtheorem{definition}{Definition}[section]

\def\definitionautorefname{Definition}
\def\corollaryautorefname{Corollary}

\newenvironment{nonproof}{\par \textit{Nonproof:}}{\hfill$\cancel\square$}

\def\contra{\tikz[baseline, x=0.22em, y=0.22em, line width=0.032em]\draw (0,2.83)--(2.83,0) (0.71,3.54)--(3.54,0.71) (0,0.71)--(2.83,3.54) (0.71,0)--(3.54,2.83);}

\renewcommand{\Re}{\operatorname{Re}}
\renewcommand{\Im}{\operatorname{Im}}
\renewcommand{\bar}{\overline}


\begin{document}
\begin{question}
    If $n^{2}$ is a multiple of $3$, must $n$ be a multiple of $3$?
\end{question}
Yes.
\begin{proof}
    The statement can be rephrased as $3 \mid n^{2} \implies 3 \mid n$.
    We will prove the contrapositive, $3 \nmid n \implies 3 \nmid n^{2}$.
    Let $n$ such that $3 \nmid n$, we have $n = 3k + r$
    where $k \in \mathbb{N}_0$ and $r \in \{1,2\}$.
    Then we have
    \begin{align*}
        n^{2} &= (3k + r)^{2}\\
              &= 9k^{2} + 6kr + r^{2}\\
              &= 3(3k^{2} + 2kr) + r^{2}
    \end{align*}
    Since $r = 1$ or $r = 2$, we have $r^{2} = 1$ or $r^{2} = 4$,
    neither of which are a multiple of $3$.
    Thus, $n^{2}$ is not a multiple of $3$.
\end{proof}


\begin{question}
    Consider the sequence $41,43,47,53,61,\cdots$.
    Are all these numbers prime?
\end{question}
No.
\begin{proof}
    If the starting term is $x_{0}$, The sequence is $x_n = n^{2} + n + 41$.
    Thus, $x_{41} = 41^{2} + 41 + 41$ is divisible by $41$.
\end{proof}

\begin{question}
    There are four primes between 0 and 10,
    and between 10 and 20.
    Does it ever happen again that there are 
    four primes between two consecutive multiples of 10?
\end{question}
No
\begin{proof}
    $101,103,107,109$.
\end{proof}

\begin{question}
    Suppose we have some positive integers (not necessarily distinct)
    whos sum is $100$. How large can their product be?
\end{question}

\begin{proof}
    We have that
    \begin{compactenum}[(i)]
    \item 5 can be replaced with 2,3 since $2 \times 3 = 6 > 5$;
    \item Anything larger than 5 can be broken up into sums of 5 and a number less than 5;
    \item 4,4 can be replaced with 2,3,3 since $2 \times 3 \times 3 = 18 > 16 = 4 \times 4$;
    \item 1 should never exist;
    \item 2,2,2 can be replaced by 3,3 for $3 \times 3 = 9 > 8 = 2 \times 2 \times 2$;
    \end{compactenum}
    Thus, the sequence should contain:
    \begin{compactenum}[(i)]
    \item no 1s,
    \item no more than two 2s,
    \item no more than one 4,
    \item no 5s,
    \item no numbers larger than 5.
    \end{compactenum}
    Thus, the optimal sequence is 32 threes and 2 twos,
    with the product of $4 \times 3^{32}$.
\end{proof}

\begin{question}
    Write down the negation of the following assertions:
    \begin{compactenum}[(i)]
        \item $\forall m\ \exists n\ \forall a\ \forall b\ (n \ge m) \land [(a = 1) \lor (b = 1) \lor (ab \ne n)]$
        \item if Bumrah is not a faster bowler than Tait, 
            then Australia is worse than England in cricket.
    \end{compactenum}
\end{question}

\begin{answer}\ \\
    (i)
    $\exists m\ \forall n\ \exists a\ \exists b\ (n < m) \lor [(a \ne 1) \land (b \ne 1) \land (ab = n)]$

    (ii) 
    Bumrah is not a faster bowler than Tait and Australia is better or equal to England in cricket.
\end{answer}

\begin{question}
    Prove that $A \setminus (B \cup C) = (A \setminus B) \cap (A \setminus C)$.
\end{question}
\begin{proof}
    Let's use indicator functions. We have
    \[
        \indic{A \setminus B} = \indic A (1 - \indic B).
    \]
    Thus,
    \begin{align*}
        \indic{A \setminus (B \cup C)}
            &= \indic{A} (1 - \indic{B \cup C})\\
            &= \indic{A} (1 - \indic{B} - \indic{C} + \indic{B \cap C})\\
            &= \indic{A \cap A} (1 - \indic{B})(1 - \indic{C})\\
            &= (\indic{A}(1 - \indic{B}))(\indic A(1 - \indic C))\\
            &= \indic{A \setminus B} \indic{A \setminus C}\\
            &= \indic{(A \setminus B) \cap (A \setminus C)}
    \end{align*}
\end{proof}

\begin{question}
    Let $A_{1},A_{2},\cdots$ be sets such that
    for each $n \in \mathbb{N}$, we have $A_{1} \cap \cdots \cap A_n \ne \emptyset$.
    Can we have $A_{1} \cap A_{2} \cap \cdots = \emptyset$?
\end{question}
Yes, we can, by counterexample.
\begin{proof}
    We write $S_n$ for $A_{1} \cap \cdots \cap A_n$,
    and $S_\infty$ for $A_{1} \cap A_{2} \cap \cdots$.
    We have by definition that 
    \[
    S_n = \{x : \forall k \in \{1,\cdots,n\},x \in A_k\}
    \]
    and
    \[
    S_\infty = \{x : \forall k \in \mathbb{N},x \in A_k\}
    \]
    Let $A_n = \{n,n+1,\cdots\}$, or in other words
    \[
    A_n = \{x \in \mathbb{N} : x \ge n\}.
    \]
    Clearly, $n + 1 \in S_n$.
    However, no $x \in S_\infty$ exists, since that would require
    $x \in \mathbb{N}$ such that for all $n \in \mathbb{N}$,
    we have $x > n$, which is plainly not true by $n = x + 1$.
\end{proof}

\begin{question}
    Does $f \circ g$ injective imply $f$ injective?
    Does it imply $g$ injective?
    What happens if we replace \lq{}injective\rq{} with \lq{}surjective\rq{}?
\end{question}

It does not imply that $f$ is injective,
but it does imply that $g$ is injective.

Assume that $g$ is not injective. Thus, there exists
$a,b$ such that $g(a) = g(b)$ but $a \ne b$.
Then, we have $f(g(a)) = f(g(b))$ and $a \ne b$, so $f \circ g$
isn't injective, which is a contradiction.
Thus, $g$ is injective.

Alternatively, remember that injectivity
is equivalent to left-invertibility, so 
there exists $h$ such that $h \circ f \circ g \equiv \id{}$.
Then, $(h \circ f) \circ g \equiv \id{}$, so $g$ is left-invertible,
and thus injective.

For the injectivity of $f$,
let $g : \{1\} \to \{1,2\}$ be defined by $g(1) = 1$,
and $f : \{1,2\} \to \{1\}$ be defined by $f(1) = 1$, $f(2) = 1$.
It is clear that $f$ is not injective, since $f(1) = f(2)$ but $1 \ne 2$.
However, $f \circ g$ is injective, since its only input is $1$
with $f \circ g(1) = f(g(1)) = f(1) = 1$.

With surjection, it does imply that $f$ is surjective,
but not that $g$ is surjective.

$f \circ g$ is surjective means that for any $y$,
there exists $x$ such that $f \circ g(x) = y$.
Thus, $f(g(x)) = y$, and $f$ is surjective.

Alternatively, remember that surjectivity
is equivalent to right-invertibility
so there exists $h$ such that $f \circ g \circ h \equiv \id{}$.
Then, $f \circ (g \circ h) \equiv \id{}$, so $g \circ h$ is the
right identity of $f$, and $f$ is surjective.

For $g$ surjectivity, use the previous example,
we have $f \circ g$ surjective but $g$ not surjective.

\begin{question}
    Let $f : X \to Y$. Let $A,B \subset X$ and $C,D \subset Y$.
    For each of the claims below, give a proof or counterexample.
    Show that each false claim can be turned true
    by replacing $=$ with either $\subset$ or $\supset$.
\end{question}

Note that $\subset$ is equivalent to $\subseteq$
in this question, and $\subsetneq$ is (un)used for proper subsets.

\begin{claim*}[i]
\[
f(A \cup B) = f(A) \cup f(B)
\]
\end{claim*}
\begin{proof}
    \begin{align*}
        &y \in f(A \cup B)\\
        &\iff \exists x \in A \cup B : f(x) = y\\
        &\iff (\exists x \in A : f(x) = y) \lor (\exists x \in B : f(x) = y)\\
        &\iff y \in f(A) \lor y \in f(B)\\
        &\iff y \in f(A) \cup f(B)
    \end{align*}
\end{proof}

\begin{claim*}[ii]
    \[
    f^{-1}(C \cup D) = f^{-1}(C) \cup f^{-1}(D)
    \]
\end{claim*}
\begin{proof}
    If $f^{-1}$ is a function itself,
    then this statement is trivial from the last claim.
    Otherwise, let $f^{-1}(C)$ be the preimage of $C$, i.e.
    $f^{-1}(C) = \{x \in X : f(x) \in C\}$. Then
    \begin{align*}
        &x \in f^{-1}(C \cup D)\\
        &\iff f(x) \in C \cup D\\
        &\iff f(x) \in C \lor f(x) \in D\\
        &\iff x \in f^{-1}(C) \cup f^{-1}(D)
    \end{align*}
\end{proof}

\begin{claim*}[iii]
    \[
    f(A \cap B) = f(A) \cap f(B)
    \]
\end{claim*}

This claim is false, but can be made true
by writing $\subseteq$.

\begin{proof}
    By counterexample,
    let $f(x) = 1$, and $A = \{1\}$, $B = \{2\}$.
    Then, $f(A \cap B) = f(\emptyset) = \emptyset$.
    On the RHS, $f(A) = \{1\} = f(B)$, so $f(A) \cap f(B) = \{1\} \ne \emptyset$.
    After replacement by $\subseteq$, we have
    \begin{align*}
        &y \in f(A \cap B)\\
        &\iff \exists x \in A \cap B : f(x) = y\\
        &\implies (\exists x \in A : f(x) = y) \land (\exists x \in B : f(x) = y)\\
        &\iff y \in f(A) \land y \in f(B)\\
        &\iff y \in f(A) \cap f(B)
    \end{align*}
    Thus we have $f(A \cap B) \subseteq f(A) \cap f(B)$.
\end{proof}
The core reason of why this claim is false is that
$f(A) \cap f(B)$ requires $a,b$ such that $f(a) = f(b) = y$,
but $f(A \cap B)$ requires $a = b$ on top of that.

\begin{claim*}[iv]
    \[
    f^{-1}(C \cap D) = f^{-1}(C) \cap f^{-1}(D)
    \]
\end{claim*}
\begin{proof}
    \begin{align*}
        &x \in f^{-1}(C \cap D)\\
        &\iff f(x) \in C \cap D\\
        &\iff f(x) \in C \land f(x) \in D\\
        &\iff x \in f(C) \cap f(D)
    \end{align*}
\end{proof}
This claim is interesting, because it holds
even under the inverse function definition,
despite the fact that $f^{-1}$ is itself a function.
The core lies in that $f^{-1}$ is always injective.

\begin{claim*}[v]
    \[
    f^{-1}(f(A)) = A
    \]
\end{claim*}

This claim is true if $f^{-1} : B \to A$ is the inverse function,
and false if $f^{-1} : \mathcal{P}(B) \to \mathcal{P}(A)$ is the preimage function.
But it is true in all cases if replaced with $\supseteq$.

\begin{proof}
    If $f^{-1}$ is the inverse function, by the definition of the inverse,
    we have $f^{-1} \circ f \equiv \id X$, and $\id X(A) = A$.

    If $f^{-1}$ is the preimage, then by counterexample,
    let $f : \mathbb{N} \to \mathbb{N}$. with $f(x) = 1$, and $A = \{1\}$.
    then $f^{-1}(f(A)) = f^{-1}(\{1\}) = \mathbb{N} \ne A$.

    However, recall that $x \in f^{-1}(f(A))$ iff
    $f(x) \in f(A)$. If $x \in A$, then
    $f(x) \in f(A)$. Thus $f^{-1}(f(A)) \supseteq A$.
\end{proof}

\begin{claim*}[vi]
    \[
    f(f^{-1}(C)) = C
    \]
\end{claim*}
\begin{proof}
    \begin{align*}
        &y \in f(f^{-1}(C))\\
        &\iff \exists x \in X : f(x) \in C \land f(x) = y\\
        &\iff y \in C
    \end{align*}
\end{proof}

\begin{question}
    Define a relation $R$ on $\mathbb{N}$ by
    setting $aRb$ if $a$ divides $b$ or $b$ divides $a$.
    Is $R$ an equivalence relation?
\end{question}

No, $R$ isn't.
\begin{proof}
    $R$ is not transitive by counterexample. 
    We have $2 R 6$, since $2 \times 3 = 6$,
    and $3 R 6$, since $3 \times 2 = 6$,
    but we do not have $2 R 3$, since both are prime.
\end{proof}

\begin{question}
    The relation $S$ contains the relation $R$
    if $aSb$ whenever $aRb$.
    Let $R$ be the relation on $\mathbb{Z}$
    given by $aRb$ if $b = a+3$.
    How many equivalence relations on $\mathbb{Z}$ contain $R$?
\end{question}

% TODO reverse the proof a little.

\begin{proof}
    If an equivalence relation $\sim$ contains $R$,
    thus it must have $a \sim a + 3$.
    By symmetry, it must have $a + 3 \sim a$ or equivalently $a \sim a - 3$.
    By reflexivity. It must have $a \sim a$.

    Noticing that $a \sim a + 3 \sim a + 6 \sim \cdots$ and similarly
    $a \sim a - 3 \sim a - 6 \sim \cdots$, we have that
    $a \sim b$ if $a - b = 3n$, for some $n \in \mathbb{Z}$.

    Besides that, no other constraints are required.
    Consider the sets $A_0,A_1,A_2$, given by $A_n = \{x \in \mathbb{Z} : x \equiv n \mod 3\}$.
    It is clear that for any $a,b \in A_n$, we have $a \sim b$.
    However, relations between the sets are not given,
    but if any $a \in A_i$ and $b \in A_j$ has $a \sim b$,
    then by transitivity all $x \in A_i$ and $y \in A_j$ has $x \sim y$.
    In this case, we write $A_i \sim A_j$.

    Thus, there are five cases:
    \begin{compactenum}
    \item $A_0 \sim A_1 \sim A_2$.
    \item $A_0 \sim A_1$ and $A_0,A_1 \nsim A_2$.
    \item $A_0 \sim A_2$ and $A_0,A_2 \nsim A_1$.
    \item $A_1 \sim A_2$ and $A_1,A_2 \nsim A_0$.
    \item $A_i \nsim A_j$ if $i \ne j$.
    \end{compactenum}
\end{proof}

\begin{question}
    We are given an operation $*$ on the positive integers, satisfying
    \begin{compactenum}[(i)]
    \item $1 * n = n + 1$ for all $n$;
    \item $m * 1 = (m - 1) * 2$ for all $m > 1$;
    \item $m * n = (m - 1) * (m * (n - 1))$ for all $m,n > 1$.
    \end{compactenum}
    Find the value of $5 * 5$.
\end{question}

Let us define $*$ to be right associative,
as in $a * b * c = a * (b * c)$.
Note this defines the lack of brackets
to be bracketing from the right, and is
not a statement of associativity.

We can then write
\begin{align*}
    m * n &= (m - 1) * m * (n - 1)\\
          &= (m - 1) * (m - 1) * m * (n - 2)\\
          &= (m - 1) * \cdots * m * 1\\
          &= (m - 1) * \cdots * (m - 1) * 2
\end{align*}
where the number of $(m - 1)$ terms in the
final expression is equal to $n$.

We need new notation. 
Let the above expression be written $(m - 1) *^{n} 2$.
Note that this continues to be right-associative,
and the power on the $*$ applies to the argument on the left.
Thus, we have the identity $m * n = (m - 1) *^{n} 2$.

The smallest nontrivial equation is $2 * 2 = 1 *^{2} 2 = 4$.
We can consider $2 * 3 = 2 *^{3} 2 = 2 *^{2} 4 =$

We have then:
\begin{align*}
    1 *^{m} n &= n + m\\
    2 * n &= 1 *^{n} 2\\
          &= n + 2\\
    2 *^{m} n &= 2 *^{m - 1} (2 * n) = 2 *^{m - 1} n + 2\\
        &= 2 *^{m - 2} n + 4\\
        &= n + 2m\\
    3 * n &= 2 *^{n} 2\\
          &= 2(n + 1)\\
    3 *^{m} n &= 3 *^{m - 1} (3 * n)\\
              &= 3 *^{m-1} 2n + 2\\
              &= 3 *^{m-2} 4n + 4 + 2\\
              &= 3 *^{m-3} 8n + 8 + 4 + 2\\
              % &= 3 *^{m-4} 16n + 16 + 8 + 4 + 2\\
              &= 2^{m}n + 2(2^{m} - 1)\\
    4 * n &= 3 *^{n} 2\\
          &= 2^{n+1} + 2(2^{n} - 1)\\
          &= 2^{n+2} - 2\\
    4 *^{m} n &= 4 *^{m - 1} 2^{n+2} - 2\\
              &= 4 *^{m - 2} 2^{2^{n+2} - 2 + 2} - 2 = 4 *^{m - 2} 2^{2^{n+2}} - 2\\
              &= 4 *^{m-3} 2^{2^{2^{n+2}}} - 2\\
              &= 2^{2^{\cdots^{n+2}}} - 2
\end{align*}
where the number of $2$s in the main power tower is equal to $m$.

We have then that
\begin{align*}
    5 * 5 &= 4 *^{4} * 2\\
          &= 2^{2^{2^{2^{2+2}}}} - 2\\
          &= 2^{2^{2^{16}}} - 2\\
          &= 2^{2^{65536}} - 2
\end{align*}
The exact value of which is too large to calculate.
The number $2^{65536}$ has 19729 digits.
If the entire number was to be stored,
we would need roughly $\log_2(2^{2^{65536}}) = 2^{65536}$ bits,
or roughly $2 \times 10^{19728}$ bits,
though that would be useless since the bit pattern
is trivial due to its derivation from powers of two.

\begin{question}
    The \emph{symmetric difference} $A \symmdif B$ of two sets $A$ and $B$
    is the set of elements that belong to exactly one of $A$ and $B$.
    Express this in terms of $\cup$, $\cap$, and $\setminus$.
    Prove that $\symmdif$ is associative.
\end{question}

We have $A \symmdif B = (A \cup B) \setminus (A \cap B)$.

\begin{proof}
    We know $x \in (A \symmdif B) \symmdif C$ has
    exactly one of $x \in A \symmdif B$ and $x \in C$,
    and if $x \in A \symmdif B$ then we have
    exactly one of $x \in A$ or $x \in B$.
    Thus, exactly one of $x \in A$, $x \in B$, and $x \in C$ is true.
    Thus, the expression is permutation invariant.
    Combining this with commutativity gives
    \begin{align*}
        (A \symmdif B) \symmdif C &= C \symmdif (A \symmdif B)\\
                                  &= A \symmdif (B \symmdif C)
    \end{align*}
    Thus, associativity is shown.
\end{proof}

\begin{question}
    hah kill me.
\end{question}

\end{document}
