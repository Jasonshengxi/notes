\documentclass[12pt]{article}
\usepackage[left = 1in, right = 1in, top = 1in, bottom = 1in]{geometry}
\usepackage{paralist}
\usepackage{cancel}
\usepackage{enumitem}
\usepackage{amsmath}
\usepackage{amssymb}
\usepackage{amsthm}
\usepackage{tkz-euclide}
\usepackage{hyperref}
\usepackage{esdiff}
\usepackage{parskip}
\usepackage{accents}
\usepackage{xcolor}

\usetikzlibrary{arrows.meta,positioning}

\newtheoremstyle{customstyle}
  {8pt} % Space above (adjust as needed)
  {0} % Space below (adjust as needed)
  {} % Body font
  {} % Indent amount
  {\bfseries} % Theorem head font
  {. } % Punctuation after theorem head
  {0pt} % Space after theorem head
  {} % Theorem head spec
\theoremstyle{customstyle}
\newtheorem{theorem}{Theorem}[section]
\newtheorem{corollary}{Corollary}[theorem]
\newtheorem{lemma}{Lemma}[section]
\newtheorem{definition}{Definition}[section]

\def\definitionautorefname{Definition}
\def\corollaryautorefname{Corollary}

\newenvironment{nonproof}{\par \textit{Nonproof:}}{\hfill$\cancel\square$}

\def\contra{\tikz[baseline, x=0.22em, y=0.22em, line width=0.032em]\draw (0,2.83)--(2.83,0) (0.71,3.54)--(3.54,0.71) (0,0.71)--(2.83,3.54) (0.71,0)--(3.54,2.83);}

\renewcommand{\Re}{\operatorname{Re}}
\renewcommand{\Im}{\operatorname{Im}}
\renewcommand{\bar}{\overline}


\begin{document}
% Mechanics
% www.damtp.cam.ac.uk/user/po242/mechanics.html

\section{Equilibrium of a Single Particle}

\emph{Statics} is a branch of mechanics that studies forces in equilibrium.
We investigate all forces acting on a particle such that it remains in equilibrium,
which is a part of statics.

There are four fundamental forces: 
gravity, electromagnetic, weak nuclear, and strong nuclear.
All other forces are derived from these four forces, i.e. 
friction, tension, etc.

\def\iangle{35} % Angle of the inclined plane
\def\down{-90}
\def\arcr{0.5cm} % Radius of the arc used to indicate angles

\begin{tikzpicture}
    \draw (0,0) coordinate (O) -- (4,0);
    \draw (O) -- coordinate[pos=0.5] (mid) ++(\iangle:5) coordinate (A);
    \draw (mid) node[rotate=\iangle,rectangle,draw,minimum size=0.5cm,yshift=0.25cm] (M) {};
    \draw[-latex] (M.west) -- ++(\iangle:-1.5) node[left] () {F};
    \draw[-latex] (M.east) -- ++(\iangle:1.5) node[above right] () {T};
    \draw[-latex] (M.north) -- ++(\iangle+90:1) node[above left] () {R};
    \draw[-latex] (M) -- ++(0,-1) node[below] () {W};
    \draw (O)++(\arcr,0) arc [start angle=0, end angle=\iangle, radius=\arcr];
    \draw (O)++(\iangle*0.5:\arcr+0.25cm) node {$\alpha$};
\end{tikzpicture}

Resolving forces along parallel and perpendicular directions yields:
\begin{align}
    T &= F + W\sin\alpha \\
    R &= W\cos\alpha 
\end{align}
We can reduce the number of forces in our calculaiton
by employing an empirical result relating friction and normal reaction:
\begin{equation}
F \le \mu R,
\end{equation}
where $\mu$ is the coefficient of friction.

Combining (1), (2) and (3), we have
\begin{equation}
T \le W(\sin\alpha  + \mu\cos\alpha ),
\end{equation}
so $T$ is determined when equality holds.

Now, if the object is sliding \emph{up} the plane as opposed to down,
we can simply flip $F$ to $-F$, yielding
\begin{equation}
T \ge W(\sin\alpha  - \mu\cos\alpha ).
\end{equation}

Combining (4) and (5) yields a range of possible values of $T$,
\[
W(\sin\alpha  - \mu\cos\alpha ) \le T \le W(\sin\alpha  + \mu\cos\alpha ).
\]
The equilibrium condition for a particle is that
the sum of the vector forces are zero:
\[
    \sum_{i}^{} \vec{F}_{i} = \vec{0}
\]

\begin{tikzpicture}[scale=3]
    \draw (-5,0) coordinate (O) 
        ++(45:1) coordinate (F1)
        (O)++(135:1) coordinate (F2)
        (O)++(250:1) coordinate (F3);
    \draw[-latex] (O) -- (F1) node[above right] {F1};
    \draw[-latex] (O) -- (F2) node[above left] {F2};
    \draw[-latex] (O) -- (F3) node[below] {F3};
    
\end{tikzpicture}

\end{document}
