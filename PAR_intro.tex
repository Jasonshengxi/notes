\documentclass[12pt]{article}
\usepackage[left = 1in, right = 1in, top = 1in, bottom = 1in]{geometry}
\usepackage{textcomp}
\usepackage{gensymb}
\usepackage{paralist}
\usepackage{cancel}
\usepackage{enumitem}
\usepackage{amsmath}
\usepackage{amssymb}
\usepackage{amsthm}
\usepackage{tkz-euclide}
\usepackage{hyperref}
\usepackage{esdiff}
\usepackage{parskip}
\usepackage{accents}
\usepackage{xcolor}

\usetikzlibrary{arrows.meta,positioning}

\newtheoremstyle{customstyle}
  {8pt} % Space above (adjust as needed)
  {0pt} % Space below (adjust as needed)
  {} % Body font
  {} % Indent amount
  {\bfseries} % Theorem head font
  {. } % Punctuation after theorem head
  {0pt} % Space after theorem head
  {} % Theorem head spec
\theoremstyle{customstyle}
\newtheorem{theorem}{Theorem}[section]
\newtheorem{exercise}{Exercise}[section]
\newtheorem{claim}[theorem]{Claim}
\newtheorem{prop}[theorem]{Proposition}
\newtheorem{corollary}[theorem]{Corollary}
\newtheorem{lemma}[theorem]{Lemma}
\newtheorem{definition}[theorem]{Definition}
\newtheorem{question}{Question}
\newtheorem{subquestion}{Part}[question]

\def\definitionautorefname{Definition}
\def\corollaryautorefname{Corollary}

\newenvironment{nonproof}{\par $\cancel {\text{\textit{Proof}}}.$}{\hfill$\cancel\square$}

\def\contra{\tikz[baseline, x=0.22em, y=0.22em, line width=0.032em]\draw (0,2.83)--(2.83,0) (0.71,3.54)--(3.54,0.71) (0,0.71)--(2.83,3.54) (0.71,0)--(3.54,2.83);}
\newenvironment{answer}{\par\noindent\textit{Answer.}}{\par}

\renewcommand{\CancelColor}{\color{red}}
\renewcommand{\Re}{\operatorname{Re}}
\renewcommand{\Im}{\operatorname{Im}}
\renewcommand{\bar}{\overline}
% \renewcommand{\vec}[1]{\undertilde{\mathrm{#1}}}
% \renewcommand{\vec}[1]{\undertilde{#1}}
% \renewcommand{\vec}[1]{\underline{#1}}
\renewcommand{\vec}[1]{\mathbf{#1}}
\newcommand{\unitvec}[1]{\hat{\vec{#1}}}
\newcommand{\sol}{$\operatorname{sol}^{\simeq}$}
\newcommand{\Arg}{\operatorname{Arg}}
\newcommand{\Log}{\operatorname{Log}}
\newcommand{\vecspan}{\operatorname{span}}
\newcommand{\id}[1]{\operatorname{id}_{#1}}
\newcommand{\indic}[1]{i_{#1}}
\newcommand{\Sym}{\operatorname{Sym}}
\newcommand{\Isom}{\operatorname{Isom}}
\newcommand{\di}{\mathrm{d}}
\newcommand{\gen}[1]{\langle{#1}\rangle}

\definecolor{applegreen}{rgb}{0.55, 0.71, 0.0}
\definecolor{ufogreen}{rgb}{0.24, 0.82, 0.44}


\begin{document}

\begin{question}
    Complete exercise 3.2 part (i) from p14.
\end{question}


\begin{proof}
    {}\ 

    \begin{tikzpicture}
        \def\width{3}
        \def\xoff{1}
        \def\yoff{4}
        \def\angr{0.7}
        \def\ra{0.2}

        \tkzInit[xmin=-3,xmax=3,ymin=-3,ymax=3]
        \tkzDefPoints{-\width/0/A,\width/0/C,\xoff/\yoff/B}
        \tkzDefLine[altitude](A,B,C) \tkzGetPoint{M}
        \tkzDefLine[altitude](B,A,C) \tkzGetPoint{N}

        \tkzDrawSegments[dashed](B,M A,N)
        \tkzLabelSegment[pos=0.7, right](B,M){$h_{1}$}
        \tkzLabelSegment(A,N){$h_{2}$}
        \tkzMarkRightAngles[size=\ra](B,M,A A,N,B)
        \tkzLabelPoints[below left](A)
        \tkzLabelPoints[below right](C)
        \tkzLabelPoints[below](M)
        \tkzLabelPoints[above right](N)
        \tkzLabelPoints[above](B)

        \tkzDrawSegments(A,B B,C C,A)
        \tkzLabelSegment(A,C){$b$}
        \tkzLabelSegment[above left](A,B){$c$}
        \tkzLabelSegment[above right, pos=0.4](C,B){$a$}

        \tkzLabelAngle[pos={0.25+\angr}](B,C,A){$\gamma$}
        \tkzLabelAngle[pos={0.25+\angr}, left](A,B,C){$\beta$}
        \tkzLabelAngle[pos={0.3+\angr}, fill=white, inner sep=1.7pt](C,A,B){$\alpha$}
        \tkzMarkAngles[size=\angr](B,C,A A,B,C C,A,B)
    \end{tikzpicture}

    Let triangle $\triangle ABC$ be an arbitrary triangle.
    Define $c$ by the length $|AB|$,
    and similarly $a$ by $|BC|$, and $b$ by $|AC|$.
    Drop an altitude of the $\triangle ABC$ through $A$,
    intersecting line $BC$ at $N$. Define $h_{2}$ by $|AN|$.
    Similarly, drop altitude through $B$,
    intersecting $AC$ at $M$. Define $h_{1}$ by $|BM|$.

    Considering $\triangle BCM$, we have $\sin\gamma = h_{1}/a$.
    Repeating this in $\triangle ABM$, we have $\sin\alpha = h_{1}/c$.
    Rearranging for $h_{1}$ in each equation and equating,
    we have $c\sin\alpha = a\sin\gamma $. 
    Dividing by $ac$ on boths sides yields
    \begin{equation}
    \frac{\sin\alpha }{a} = \frac{\sin\gamma }{c}.
    \end{equation}
    Now, similarly,
    in $\triangle ANC$, we have $\sin\gamma = h_{2}/b$.
    In $\triangle ANB$, we have $\sin\beta = h_{2}/c$.
    Similar algebraic manipulation yields
    \begin{equation}
    \frac{\sin\gamma }{c} = \frac{\sin\beta }{b}.
    \end{equation}
    Equating the RHS of $(1)$ with the LHS of $(2)$ yields
    \[
    \frac{\sin\alpha }{a} = \frac{\sin\beta }{b} = \frac{\sin\gamma }{c}.
    \]
\end{proof}

\begin{question}
    If $f: A \to B$ is injective and $g: B \to C$ is injective,
    then the composition $g \circ f : A \to C$ is also injective.
\end{question}
\begin{proof}
    % Aim: if gf(a1) = gf(a2), then a1 = a2.
    Let $a_{1},a_{2} \in A$ such that $g \circ f(a_{1}) = g \circ f(a_{2})$,
    which by definition means $g(f(a_{1})) = g(f(a_{2}))$.
    Since $g$ is injective, $f(a_{1}) = f(a_{2})$.
    Since $f$ is injective, $a_{1} = a_{2}$.
\end{proof}

\begin{question}
    A sequence $(x_n)$ is \emph{convergent} if 
    there is some $l \in \mathbb{R}$ such that for all $\varepsilon > 0$
    there is a positive integer $N$ such that
    for all $n \ge N$ we have $|x_n - l| < \varepsilon $.
\end{question}

\begin{subquestion}
    Use quantifier notation to write this definition.
\end{subquestion}

Writing down the quantifier corresponding to each part in the definition yields
\[
\exists l\in \mathbb{R},\forall \varepsilon > 0,\exists N \in \mathbb{N},\forall n \ge N, |x_n - l| < \varepsilon.
\]
\begin{subquestion}
    Describe the 'demon game' associated with the previous statement.
\end{subquestion}

\noindent In each round of the game, we do the following in order:
\begin{enumerate}[noitemsep]
    \item I pick a $l \in \mathbb{R}$, and then
    \item the demon picks a $\varepsilon > 0$, and then
    \item I pick a $N \in \mathbb{N}$, and finally
    \item the demon picks some $n \ge N$.
\end{enumerate}
After these moves, if the statement $|x_n - l| < \varepsilon$ is true, 
then I win. Otherwise, the demon wins.

\begin{subquestion}
    Let $(x_n)$ be the sequence defined by $x_n = 0$ if $n$ is composite,
    and $x_n = 1/n$ if $n$ is prime, and $x_1 = 17$.
    Prove that $(x_n)$ converges.
\end{subquestion}
\begin{proof}
    We define $l = 0$. Let arbitrary $\varepsilon > 0$.

    If $\epsilon > 1/2$, then let $N = 2$.
    Let arbitrary $n \ge N$.
    If $n$ is composite, then $|x_n - l| = |x_n| = |0| = 0 < \varepsilon$.
    If $n$ is prime, then $|x_n - l| = |x_n| = |n^{-1}| = n^{-1} \le N^{-1} = 1/2 < \varepsilon$.
    Since $n \ge N = 2$, all $n$ is either composite or prime.
    
    If $\epsilon \le 1/2$, then define $N=(\varepsilon/2)^{-1}$.
    Let arbitrary $n \ge N$.
    if $n$ is composite, then $|x_n - l| = |x_n| = |0| = 0 < \varepsilon$.
    If $n$ is prime, then $|x_n - l| = |x_n| = |n^{-1}| = n^{-1} \le N^{-1} = \varepsilon/2 < \varepsilon$.
    Since $N = (\varepsilon/2)^{-1} \ge (1/4)^{-1} = 4$, all
    $n \ge N \ge 4$ must be composite or prime.
\end{proof}

\begin{subquestion}
    Write in quantifier notation a definition of
    "the sequence $y_n$ does not converge."
\end{subquestion}

Negating each quantifier and then the final statement yields
\[
\forall l \in \mathbb{R}, \exists \varepsilon > 0, \forall N \in \mathbb{N}, \exists n \ge N, |y_n - l| \ge \varepsilon.
\]

\begin{subquestion}
    Let $(y_n)$ be the sequence defined by $y_n = 1$ if $n$ is odd and
    $y_n = 0$ if $n$ is even. Prove that $(y_n)$ does not converge.
\end{subquestion}

\begin{proof}
    Letting arbitrary $l \in \mathbb{R}$, we define $\varepsilon = 1/4$.
    Let arbitrary $N \in \mathbb{N}$, 
    we consider two values for $n$, that being $N$ and $N+1$.

    Since $N$ and $N+1$ are consecutive, one of them must be even, let that be $a$.
    By definition, we have $y_a = 0$.
    If $|y_n - l| \ge \varepsilon$, then the proof is finished.
    Otherwise, we have $|y_n - l| < \varepsilon$, so $|l| < 1/4$, 
    which means $-1/4 < l < 1/4$.

    Since $N$ and $N+1$ are consecutive, one of them must be odd. Let that be $b$.
    By definition, we have $y_b = 1$.
    Assuming $|y_b - l| < \varepsilon$, we have $|1 - l| < 1/4$.
    Thus, $-1/4 < 1 - l < 1/4$, so $3/4 < l < 5/4$.
    This contradicts with our previous result that $-1/4 < l < 1/4$,
    so $|y_b - l| \ge \varepsilon$.

    Thus, we have shown that one of $N$ and $N+1$ as a value of $n$
    shows that $|y_n -l| \ge \varepsilon$.
\end{proof}

\begin{question}
    Solve as much of problem 10 of the introductory problems sheet as possible:
    $n$ people each hold one piece of unique information.
    Calls are to be made between pairs of people, where
    the two people are both informed of any pieces of information
    the other knows.
    What is the least number of calls required to inform
    all people of all information?
\end{question}

First, we say that a person $A$ \emph{knows} $B$ iff
$A$ knows the piece of information that only $B$ knew initially.

A \emph{sequence of calls}, or just a \emph{sequence},
is defined as a tuple $(\Omega, S)$,
where $\Omega$ is a set, and an element of $\Omega$ is a person,
and $S$ is a sequence of pairs of people.
We denote $s_n$ for the $n$th element of $S$, where $s_{1}$ is the first element.
The \emph{degree} of a sequence is the size of $\Omega$.
The \emph{length} of a sequence is the length of $S$.

A \emph{state} is a function $f : \Omega \to \Omega^{*}$,
where $\Omega^{*}$ denotes the set of subsets, or power set, of $\Omega$,
and $f(a)$ for a person $a$ is equal to the set of people $a$ knows.
The intial state $f_{0}$ is defined by $f_{0}(a) = \{a\}$ for all $a \in \Omega$.

In this formulation, the application of a call to a state
is a function $g$ that maps a state and a call to a state,
defined by $g(f,(a,b))(x) = f(x)$ if $x \notin \{a,b\}$,
and $g(f,(a,b))(x) = f(a) \cup f(b)$ otherwise.

A sequence a \emph{solution of degree $n$} iff
after applying all calls in $S$ in order,
the final state $f$ has $f(a) = \Omega$ for all $a \in \Omega$.
A solution is \emph{optimal} iff there does not exist
a solution of the same degree with a lower length.

All optimal solutions have the same length.
Let $f(n)$ denote the length of an optimal solution of degree $n$.

We have $f(4) \le 4$, since for four people,
say labeled $A,B,C,D$, the following solution exists: $AB, CD, AC, BD$.

Consider $f(k+1)$.
Notice that $AB$, followed by the optimal solution
for $n=k$, which is $f(k)$ calls long, then followed by $BA$,
is a $n+2$ calls long sequence that informs everyone of everything.
Thus, $f(k+1) \le f(k) + 2$.
Applying trivial induction from $f(4) \le 4$ yields $f(n) \le 2n - 4$.

Consider information $a$ known initially only by $A$.
Let $N_a$ be the number of people knowing $a$.
In each call, one of three things must be true:
\begin{compactitem}
\item Both parties know $a$. 
    Then, no exchange of $a$ occurs and $N_a$ has not increased.
\item No parties know $a$. 
    Then, no exchange of a occurs and $N_a$ has not increased.
\item Exactly one party knows $a$.
    Then, the other party is informed of $a$ and $N_a$ has increased by $1$.
\end{compactitem}
Thus, $N_a$ can increase by at most $1$ in each call. 
After all calls, $N_a = n$, and it must take at least $n-1$ moves
for that to happen. Thus, $f(n) \ge n - 1$.

Let the \emph{degree} of a person be the number of calls they make.
Every call increases the total degree of everybody by $2$,
since both parties gain $1$ to their degree.
Thus, the minimum total degree is $2f(n) \ge 2n - 2$.
Thus, for any possible solution,
we have total degree greater than or equal to $2n - 2$.
Since there are only $n$ people, there must exist
someone with degree greater than or equal to $2$.

Consider the following operation: 
Given a solution $(\Omega, S)$ and $a,b \in \Omega$,
let $\Omega' = \Omega - \{a\}$, and
let $S'$ be $S$ with all occurences of $a$ replaced by $b$.
It is obvious that $(\Omega', S')$ forms a sequence.
We claim that it is a solution.

Define $f_n$ to be $f_{0}$ after application
of the first $n$ calls of $S$.
Similarly, define $f_n'$ to be $f_{0}$ after application
of the first $n$ calls of $S'$.
Let $P(n)$ be the proposition that 
$f_n(a) \subseteq f_n'(b)$ and
for any $u \in \Omega'$,
we have $f_n(u) \subseteq f_n'(u)$.
We claim that $P(n)$ is true for any $n \ge 0$.
As a base case, $P(0)$ is clearly true since $f_{0}=f_{0}'$.

If $P(n)$ is true, for any $u \in \Omega$, we have $f_n(u) \subseteq f_n'(u)$ and $f_n(a) \subseteq f_n'(b)$.
Consider $s _{n+1}$, say equal to $(x,y)$.
If neither $x$ nor $y$ is equal to $a$,
then by definition $s'_{n+1} = s_{n+1}$.
Thus, $f_{n+1}(a) = f_n(a)$.
If $b=x$ or $b=y$, then wlog let $b=y$ and we have
$f_{n+1}'(b) = f_{n}'(b) \cup f_n'(x) \supseteq f_n'(b) \supseteq f_n(a) = f_{n+1}(a)$, 
showing half of $P(n+1)$.

For any $u \in \Omega'$, if $u \ne x$ and $u \ne y$,
then $f_{n+1}(u) = f_n(u) \subseteq f_n'(u) = f_{n+1}'(u)$.
Otherwise, $f_{n+1}(u) = f_n(x) \cup f_n(y)$,
and since $f_n(a) \subseteq f_n'(a)$ and $f_n(b) \subseteq f_n'(b)$,
we have $f_n(a) \cup f_n(b) \subseteq f_n'(a) \cup f_n'(b) = f_{n+1}'(u)$.
Thus, $P(n+1)$ is true if $a \ne x$ and $a \ne y$.

Otherwise, since $x \ne y$, we can wlog let $x = a$, $y \ne a$.
In this case, $s_{n+1} = (a,y)$.
By definition, $s _{n+1}' = (b,y)$.
For any $u \in \Omega' = \Omega - \{a\}$, if $u \ne a$ (always true) and $u \ne y$ and $u \ne b$,
then $f_{n+1}(u) = f_n(u) \subseteq f_n'(u) = f_{n+1}'(u)$.
If $u = y$, then
$f_{n+1}(u) = f_n($

\begin{question}
    Comment on two induction proofs:
\end{question}

\begin{subquestion}
    All horses in a given field have the same color.
\end{subquestion}

The induction doesn't hold as the inductive argument fails to hold
when $P(1)$ is true. $P(2)$ isn't true since a central part of the
argument is that $h_{1},h_{2},\cdots h_n$ have the same color
and so must $h_{2},h_{3},\cdots h_{n+1}$. For $n=1$, this says
that $h_{1}$ have the same color and $h_{2}$ have the same color.
There is no overlap, so $P(2)$ cannot be proven and
the induction cannot continue.

\begin{subquestion}
    Given $n$ fence posts and fences connecting those fence posts,
    if all regions bounded by fences are triangles, there must be $3n-6$ fences.
\end{subquestion}

The claim fails to mention that the proposition is
only true when $n \ge 3$, since the base case of the
induction is $n = 3$, and in each step $P(n+1)$ is proven from $P(n)$.

\begin{question}
    Let $(G,\circ)$ be a group. Prove the following facts.
\end{question}
\begin{subquestion}
    Define a new binary operation $\star$ on $G$ by $a \star b = b \circ a$.
    Is $(G,\star)$ a group? If so, is it isomorphic to $(G,\circ)$?
\end{subquestion}

\begin{proof}
    Yes. The identity in $(G,\circ)$ remains the identity in $(G,\star)$,
    since $e \star x = x \circ e = x$ and vice versa.
    The operation is clearly closed since $a \star b = b \circ a \in G$.
    The operation is associative, since
    $(a \star b) \star c = c \circ (b \circ a) = (c \circ b) \circ a = a \star (b \star c)$.
    The operation is invertible, with the same inverse as $(G,\circ)$,
    since $a \star a^{-1} = a^{-1} \circ a = e$, and vice versa.

    $(G,\star)$ is isomorphic to $(G,\circ)$. Let $\phi(a) = a^{-1}$. 
    It is clearly a bijection. we have
    $\phi(a \circ b) = (a \circ b)^{-1} = b^{-1} \circ a^{-1} = \phi(a) \star \phi(b)$.
\end{proof}

It is possible to notice the isomorphism $\phi(a) = a^{-1}$
by thinking about the identity $(ab)^{-1} = b^{-1}a^{-1}$.

\begin{subquestion}
    Fix an element $u \in G$ and define binary operation $*$ on $G$,
    by $a * b = a \circ u \circ b$. Is $(G,*)$ a group? If so, is it isomorphic to $(G,\circ)$?
\end{subquestion}

\begin{proof}
    For the sake of clarity, $a^{-1}$ always denotes the inverse of $a$ in $(G,\circ)$, not $(G,*)$.
    The identity is given by $u^{-1}$.
    We have $a * u^{-1} = a \circ u \circ u^{-1} = a \circ (u \circ u^{-1}) = a \circ e = a$,
    and similarly vice versa. Closure is obvious.
    Associativity too is obviously derived from the associativity of $\circ$.
    For inverse, we have
    $(u^{-1} \circ a^{-1} \circ u^{-1}) * a = u^{-1} \circ a^{-1} \circ u^{-1} \circ u \circ a = u^{-1}$,
    $a * (u^{-1} \circ a^{-1} \circ u^{-1}) = a \circ u \circ u^{-1} \circ a^{-1} \circ u^{-1} = a \circ a^{-1} \circ u^{-1} = u^{-1}$, and
    which is the identity in $(G,*)$. Thus, $(G,*)$ satisfies all group axioms and is a group.

    Define $\phi : G \to G$ by $\phi(a) = u^{-1} \circ a$, which has
    $\phi(a)*\phi(b) = (u^{-1}\circ a) * (u^{-1} \circ b)
    = u^{-1} \circ a \circ u \circ u^{-1} \circ b = u^{-1} \circ a \circ b = \phi(ab) $.
    Thus, $\phi$ is an isomorphism from $(G,\circ)$ to $(G,*)$.
\end{proof}

As an aside, the method I used to find the isomorphism in the last part 
was as follows:

If $\phi : G \to G$ is an isomorphism from $(G,\circ)$ to $(G,*)$, 
it must send inverses to inverses, so $\phi(a^{-1}) = u^{-1} \circ \phi(a)^{-1} \circ u^{-1}$,
Rearranging gives $(u \circ \phi(a^{-1})) = (u \circ \phi(a))^{-1}$,
Therefore, $\psi(a) = u \circ \phi(a)$ satisifies $\psi(a^{-1}) = \psi(a)^{-1}$.
This seems to be an automorphism of $(G,\circ)$?
If we let $\psi(a) = a$, the trivial automorphism,
we have $\phi(a) = u^{-1} \circ a$.

\begin{question}
    Let $G$ be a non-empty set with an associative binary operation,
    written as multiplication, such that 
    for all $a \in G$, there exists a unique $a^{*} \in G$ such that $aa^{*}a = a$.
    Prove that $G$ forms a group under this binary operation.
\end{question}
\begin{nonproof} \textit{(Partial)}
    With it being an associative binary operation, 
    closure and associativity are covered.
    It suffices to show an identity exists and that
    an inverse exists for all $a \in G$.

    Let $a \in G$.
    We have unique $a^{*}$ and $aa^{*}a = a$.
    Applying the equation again yields $aa^{*}aa^{*}a = a(a^{*}aa^{*})a = a$.
    Thus, since $a^{*}$ is unique, $a^{*}aa^{*} = a^{*}$.
    Applying uniqueness again yields $a = a^{**}$.

    Let $a,b \in G$. Then we have that if $a^{*} = b^{*}$,
    then $a^{**} = b^{**}$, so $a = b$.
    Therefore, $f(a) = a^{*}$ is injective.
    For any $a$, we have $(a^{*})^{*} = a$, so $f$ is also surjective.
    Thus, $f$ is a bijection.

    Consider the sequence $aa^{*}aa^{*}aa^{*}$.
    Using $a^{*}aa^{*} = a^{*}$ yields $aa^{*}(aa^{*})aa^{*} = aa^{*}$.
    By uniqueness, we have $(aa^{*})^{*} = aa^{*}$.
    By a similar argument, $(a^{*}a)^{*} = a^{*}a$.

    Define $R_{a} = a^{*}a$ and $L_{a} = aa^{*}$.
    Rewriting things from before gives
    $L_{a}a = a$, $aR_{a} = a$.
    $L_a L_a = L_a$, and $R_a R_a = R_a$.
    $L_a^{*} = L_a$, and $R_a^{*} = R_a$.
\end{nonproof}

\end{document}
